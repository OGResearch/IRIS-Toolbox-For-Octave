

    \filetitle{plotmat}{Visualise 2D matrix}{grfun/plotmat}

	\paragraph{Syntax}

\begin{verbatim}
[HPos,HNeg,HNanInf,HMax] = grfun.plotmat(X,...)
\end{verbatim}

\paragraph{Short-cut syntax}

\begin{verbatim}
[HPos,HNeg,HNanInf,HMax] = plotmat(X,...)
\end{verbatim}

\paragraph{Input arguments}

\begin{itemize}
\itemsep1pt\parskip0pt\parsep0pt
\item
  \texttt{X} {[} numeric {]} - 2D matrix that will be visualised; ND
  matrices will be unfolded in 2nd dimension before plotting.
\end{itemize}

\paragraph{Output arguments}

\begin{itemize}
\item
  \texttt{HPos} {[} numeric {]} - Handles to discs displaying
  non-negative entries.
\item
  \texttt{HNeg} {[} numeric {]} - Handles to discs displeying negative
  entries.
\item
  \texttt{HNanInf} {[} numeric {]} - Handles to NaN or Inf marks.
\item
  \texttt{HMax} {[} numeric {]} - Handles to circles displaying maximum
  value.
\end{itemize}

\paragraph{Options}

\begin{itemize}
\item
  \texttt{'colNames='} {[} char \textbar{} cellstr \textbar{} empty
  \textbar{} \emph{\texttt{'auto'}} {]} - Names that will be given to
  the columns of the matrix.
\item
  \texttt{'rowNames='} {[} char \textbar{} cellstr \textbar{} empty
  \textbar{} \emph{\texttt{'auto'}} {]} - Names that will be give to the
  row of the matrix.
\item
  \texttt{'maxCircle='} {[} \texttt{true} \textbar{}
  \emph{\texttt{false}} {]} - If \texttt{true},display a circle denoting
  the maximum value around each entry.
\item
  \texttt{'nanInf='} {[} char \textbar{} \texttt{X} {]} - Appearance of
  \texttt{NaN} and \texttt{Inf} entries.
\item
  \texttt{'showDiag='} {[} \emph{\texttt{true}} \textbar{}
  \texttt{false} {]} - If \texttt{false}, hide the entries on the main
  diagonal by setting them to \texttt{NaN}.
\item
  \texttt{'scale='} {[} numeric \textbar{} \emph{\texttt{'auto'}} {]} -
  Maximum value (positive) relative to which all matrix entries will be
  scaled; by default the scale is the maximum entry in the input matrix,
  \texttt{max(max(abs(X(isfinite(X))))}.
\end{itemize}

\paragraph{Description}

\paragraph{Example}


