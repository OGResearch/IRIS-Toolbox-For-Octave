

    \filetitle{plotneigh}{Plot local behaviour of objective function after estimation}{grfun/plotneigh}

	\paragraph{Syntax}

\begin{verbatim}
H = grfun.plotneigh(D,...)
\end{verbatim}

\paragraph{Input arguments}

\begin{itemize}
\itemsep1pt\parskip0pt\parsep0pt
\item
  \texttt{D} {[} struct {]} - Structure describing the local behaviour
  of the objective function returned by the
  \href{model/neighbourhood}{\texttt{neighbourhood}} function.
\end{itemize}

\paragraph{Output arguments}

\begin{itemize}
\itemsep1pt\parskip0pt\parsep0pt
\item
  \texttt{H} {[} struct {]} - Struct with handles to the graphics
  objects plotted by \texttt{plotpp}; the struct has the following
  fields with vectors of handles: \texttt{figure}, \texttt{axes},
  \texttt{obj}, \texttt{est}, \texttt{lik}, \texttt{bounds}.
\end{itemize}

\paragraph{Options}

\begin{itemize}
\item
  \texttt{'caption='} {[} \emph{empty} \textbar{} cellstr {]} -
  User-supplied graph titles; if empty, default captions will be
  automatically created.
\item
  \texttt{'model='} {[} model \textbar{} \emph{empty} {]} - Model object
  used to create graph captions if the option \texttt{'caption='} is
  \texttt{'descript'} or \texttt{'alias'}.
\item
  \texttt{'plotObj='} {[} \emph{\texttt{true}} \textbar{} \texttt{false}
  {]} - Plot the local behaviour of the overall objective function; a
  cell array can be specified to control graphics options.
\item
  \texttt{'plotLik='} {[} \emph{\texttt{true}} \textbar{} \texttt{false}
  \textbar{} cell {]} - Plot the local behaviour of the data likelihood
  component; a cell array can be specified to control graphics options.
\item
  \texttt{'plotEst='} {[} \emph{\texttt{true}} \textbar{} \texttt{false}
  \textbar{} cell {]} - Mark the actual parameter estimate; a cell array
  can be specified to control graphics options.
\item
  \texttt{'plotBounds='} {[} \emph{\texttt{true}} \textbar{}
  \texttt{false} \textbar{} cell {]} - Draw the lower and/or upper
  bounds if they fall within the graph range; a cell array can be
  specified to control graphics options.
\item
  \texttt{'subplot='} {[} \emph{\texttt{'auto'}} \textbar{} numeric {]}
  - Subplot division of the figure when plotting the results.
\item
  \texttt{'title='} {[} \texttt{\{'interpreter=','none'\}} \textbar{}
  cell {]} - Display graph titles, and specify graphics options for the
  titles.
\item
  \texttt{'linkAxes='} {[} \texttt{true} \textbar{}
  \emph{\texttt{false}} {]} - Make the vertical axes identical for all
  graphs.
\end{itemize}

\paragraph{Description}

The data log-likelihood curves are shifted up or down by an arbitrary
constant to make them fit in the graph; their curvature is preserved.

\paragraph{Example}


