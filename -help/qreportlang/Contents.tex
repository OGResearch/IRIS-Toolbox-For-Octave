
    \foldertitle{qreportlang}{Quick-report file language}{qreportlang/Contents}

	\paragraph{Figures}
 
 \begin{itemize}
 \item
   \href{qreportlang/figure}{\texttt{!++}} - Create new figure window.
 \item
   \href{qreportlang/subplot}{\texttt{\#MxN}} - Specify the subplot
   division of the figure windows. figure windows.
 \end{itemize}
 
 \paragraph{Graphs}
 
 \begin{itemize}
 \item
   \href{qreportlang/linegraph}{\texttt{!-{}-}} - Create new line graph.
 \item
   \href{qreportlang/bargraph}{\texttt{!::}} - Create new bar graph.
 \item
   \href{qreportlang/stemgraph}{\texttt{!ii}} - Create new stem graph.
 \item
   \href{qreportlang/errorbargraph}{\texttt{!II}} - Create new errorbar
   graph.
 \item
   \href{qreportlang/histgraph}{\texttt{!\^{}\^{}}} - Create new
   histogram.
 \item
   \href{qreportlang/blank}{\texttt{!..}} - Skip the current subplot
   position leaving it blank.
 \end{itemize}
 
 \paragraph{Formatting graph titles}
 
 \begin{itemize}
 \item
   \href{qreportlang/linebreak}{\texttt{//}} - Line break in graph title.
 \item
   \href{qreportlang/subtitle}{\texttt{\_\_}} - Subtitle in graph title.
 \end{itemize}
 
 \paragraph{Getting on-line help on qreport file language}
 
 \begin{verbatim}
 help qreportlang
 help qreportlang/keyword
 \end{verbatim}



