

    \filetitle{userfigure}{Insert existing figure window}{report/userfigure}

	\paragraph{Syntax}
 
 \begin{verbatim}
 P.userfigure(Caption,H,...)
 \end{verbatim}
 
 \paragraph{Input arguments}
 
 \begin{itemize}
 \item
   \texttt{P} {[} struct {]} - Report object created by the
   \href{report/new}{\texttt{report.new}} function.
 \item
   \texttt{Caption} {[} char \textbar{} cellstr {]} - Title or a cell
   array with title and subtitle displayed at the top of the figure; see
   Description for splitting the title or subtitle into multiple lines.
 \item
   \texttt{H} {[} numeric {]} - Handle to a graphics figure created by
   the user that will be captured and inserted in the report.
 \end{itemize}
 
 \paragraph{Options}
 
 See help on \url{report/figure} for options available.
 
 \paragraph{Generic options}
 
 See help on \href{report/Contents}{generic options} in report objects.
 
 \paragraph{Description}
 
 The function \texttt{report/userfigure} inserts an existing figure
 window (created by the user by standard Matlab commands, and referenced
 by its handle, \texttt{H}) into a report:
 
 \begin{itemize}
 \item
   The figure and the graphs in it must be created \emph{before} you call
   \texttt{report/figure}: any changes or additions to the figure or its
   graphs made after you call the function will not show in the report.
 \item
   The report figure cannot have any children; in other words, you cannot
   call \url{report/graph} after a call to \texttt{report/figure} with a
   graphics handle, \texttt{H}.
 \end{itemize}
 
 \subparagraph{Titles and subtitles}
 
 The input argument \texttt{Caption} can be either a text string, or a
 1-by-2 cell array of strings. In the latter case, the first cell will be
 printed as a title, and the second cell will be printed as a subtitle.
 
 To split the title or subtitle into multiple lines, use the following
 LaTeX commands wrapped in curly brackets:
 \texttt{\{\textbackslash{}\textbackslash{}\}} or
 \texttt{\{\textbackslash{}\textbackslash{}{[}Xpt{]}\}}, where \texttt{X}
 is the width of an extra vertical space (in points) added between the
 respective lines.
 
 \paragraph{Example}


