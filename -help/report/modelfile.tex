

    \filetitle{modelfile}{Write formatted model file}{report/modelfile}

	\paragraph{Syntax}\label{syntax}

\begin{verbatim}
P.modelfile(Caption,FileName,...)
P.modelfile(Caption,FileName,M,...)
\end{verbatim}

\paragraph{Input arguments}\label{input-arguments}

\begin{itemize}
\item
  \texttt{P} {[} report {]} - Report object created by the
  \href{report/new}{\texttt{report.new}} function.
\item
  \texttt{Caption} {[} char \textbar{} cellstr {]} - Title and subtitle
  displayed at the top of the table.
\item
  \texttt{FileName} {[} char {]} - Model file name.
\item
  \texttt{M} {[} model {]} - Model object from which the values of
  parameters and std devs of shocks will be read; if missing no
  parameter values or std devs will be printed.
\end{itemize}

\paragraph{Options}\label{options}

\begin{itemize}
\item
  \texttt{'latexAlias='} {[} \texttt{true} \textbar{}
  \emph{\texttt{false}} {]} - Treat alias in labels as LaTeX code and
  typeset it that way.
\item
  \texttt{'lines='} {[} numeric \textbar{} \emph{\texttt{Inf}} {]} -
  Print only selected lines of the model file \texttt{FileName};
  \texttt{Inf} means all lines will be printed.
\item
  \texttt{'lineNumbers='} {[} \emph{\texttt{true}} \textbar{}
  \texttt{false} {]} - Display line numbers.
\item
  \texttt{'footnote='} {[} char \textbar{} \emph{empty} {]} - Footnote
  at the model file title; only shows if the title is non-empty.
\item
  \texttt{'paramValues='} {[} \emph{\texttt{true}} \textbar{}
  \texttt{false} {]} - Display the values of parameters and std devs of
  shocks next to each occurence of a parameter or a shock; this option
  works only if a model object \texttt{M} is entered as the 3rd input
  argument.
\item
  \texttt{'syntax='} {[} \emph{\texttt{true}} \textbar{} \texttt{false}
  {]} - Highlight model file syntax; this includes model language
  keywords, descriptions of variables, shocks and parameters, and
  equation labels.
\item
  \texttt{'typeface='} {[} char \textbar{} \emph{empty} {]} - (Not
  inheritable from parent objects) LaTeX code specifying the typeface
  for the model file as a whole; it must use the declarative forms (such
  as \texttt{\textbackslash{}itshape}) and not the command forms (such
  as \texttt{\textbackslash{}textit\{...\}}).
\end{itemize}

\paragraph{Description}\label{description}

If you enter a model object with multiple parameterisations, only the
first parameterisation will get reported.

At the moment, the syntax highlighting in model file reports does not
handle correctly comment blocks, i.e. \texttt{\%\{ ... \%\}}.

\paragraph{Example}\label{example}


