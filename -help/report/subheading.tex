

    \filetitle{subheading}{Enter subheading in table}{report/subheading}

	\paragraph{Syntax}

\begin{verbatim}
P.subheading(CAP,...)
\end{verbatim}

\paragraph{Input arguments}

\begin{itemize}
\item
  \texttt{P} {[} struct {]} - Report object created by the
  \href{report/new}{\texttt{report.new}} function.
\item
  \texttt{CAP} {[} char {]} - Text displayed as a subheading on a
  separate line in the table.
\end{itemize}

\paragraph{Options}

\begin{itemize}
\item
  \texttt{'justify='} {[} \texttt{'c'} \textbar{} \emph{\texttt{'l'}}
  \textbar{} \texttt{'r'} {]} - (Inheritable from parent objects)
  Horizontal alignment of the subheading (centre, left, right).
\item
  \texttt{'separator='} {[} char \textbar{} \emph{empty} {]} - (Not
  inheritable from parent objects) LaTeX commands that will be inserted
  immediately after the end of the table row, i.e.~appended to
  \textbackslash{}, within a tabular mode.
\item
  \texttt{'stretch='} {[} \emph{\texttt{true}} \textbar{} \texttt{false}
  {]} - (Inheritable from parent objects) Stretch the subheading text
  also across the data part of the table; if not the text will be
  contained within the initial descriptive columns.
\item
  \texttt{'typeface='} {[} char \textbar{}
  \emph{\texttt{'\textbackslash{}itshape\textbackslash{}bfseries'}} {]}
  - (Not inheritable from parent objects) LaTeX code specifying the
  typeface for the subheading; it must use the declarative forms (such
  as \texttt{\textbackslash{}itshape}) and not the command forms (such
  as \texttt{\textbackslash{}textit\{...\}}).
\end{itemize}

\paragraph{Generic options}

See help on \href{report/Contents}{generic options} in report objects.

\paragraph{Description}

\paragraph{Example}


