

    \filetitle{series}{Add new data to graph or table}{report/series}

	\paragraph{Syntax}
 
 \begin{verbatim}
 P.series(Cap,X,...)
 \end{verbatim}
 
 \paragraph{Input arguments}
 
 \begin{itemize}
 \item
   \texttt{P} {[} struct {]} - Report object created by the
   \href{report/new}{\texttt{report.new}} function.
 \item
   \texttt{Cap} {[} char {]} - Caption used as a default legend entry in
   a graph, or in the leading column in a table.
 \item
   \texttt{X} {[} tseries {]} - Input data that will be added to the
   current table or graph.
 \end{itemize}
 
 \paragraph{Options for both table series and graph series}
 
 \begin{itemize}
 \item
   \texttt{'marks='} {[} cellstr \textbar{} \emph{empty} {]} -
   (Inheritable from parent objects) Marks that will be added to the
   legend entries in graphs, or printed in a third column in tables, to
   distinguish the individual columns of possibly multivariate input
   tseries objects.
 \item
   \texttt{'showMarks='} {[} \emph{\texttt{true}} \textbar{}
   \texttt{false} {]} - (Inheritable from parent objects) Use the marks
   defined in the \texttt{'marks='} option to label the individual rows
   when input data is a multivariate tseries object.
 \end{itemize}
 
 \paragraph{Options for table series}
 
 \begin{itemize}
 \item
   \texttt{'autoData='} {[} function\_handle \textbar{} cell \textbar{}
   \emph{empty} {]} - Function, or a cell array of functions, that will
   be used to produce new columns in the input tseries object (i.e.~new
   rows of ouput in the report).
 \item
   \texttt{'condFormat='} {[} struct \textbar{} \emph{empty} {]} -
   (Inheritable from parent objects) Structure with .test and .format
   fields describing conditional formatting of individual table entries.
 \item
   \texttt{'decimal='} {[} numeric \textbar{} \emph{\texttt{NaN}} {]} -
   (Inheritable from parent objects) Number of decimals that will be
   displayed; if NaN the \texttt{'format='} option is used instead.
 \item
   \texttt{'format='} {[} char \textbar{} \emph{\texttt{'\%.2f'}} {]} -
   (Inheritable from parent objects) Numeric format string; see help on
   the built-in \texttt{sprintf} function.
 \item
   \texttt{'footnote='} {[} char \textbar{} \emph{empty} {]} - Footnote
   at the series text.
 \item
   \texttt{'highlight='} {[} numeric \textbar{} \emph{empty} {]} -
   (Inheritable from parent objects) Periods that will get highlighted in
   tables; to highlight date ranges in graphs, use the
   \texttt{'highlight='} option in the parent graph object.
 \item
   \texttt{'inf='} {[} char \textbar{}
   \emph{\texttt{'\textbackslash{}ensuremath\{\textbackslash{}infty\}'}}
   {]} - (Inheritable from parent objects) LaTeX string that will be used
   to typeset Inf entries.
 \item
   \texttt{'nan='} {[} char \textbar{}
   \emph{\texttt{'\textbackslash{}ensuremath\{\textbackslash{}cdot\}'}}
   {]} - (Inheritable from parent objects) LaTeX string that will be used
   to typeset NaN entries.
 \item
   \texttt{'pureZero='} {[} char \textbar{} \emph{empty} {]} -
   (Inheritable from parent objects) LaTeX string that will be used to
   typeset pure zero entries; if empty the zeros will be printed using
   the current numeric format.
 \item
   \texttt{'printedZero='} {[} char \textbar{} \emph{empty} {]} -
   (Inheritable from parent objects) LaTeX string that will be used to
   typeset the entries that would appear as zero under the current
   numeric format used; if empty these numbers will be printed using the
   current numeric format.
 \item
   \texttt{'separator='} {[} char \textbar{} \emph{empty} {]} - (Not
   inheritable from parent objects) LaTeX commands that will be inserted
   immediately after the end of the table row, i.e.~appended to
   \textbackslash{}, within a tabular mode.
 \item
   \texttt{'units='} {[} char {]} - (Inheritable from parent objects)
   Description of input data units that will be displayed in the second
   column of tables.
 \end{itemize}
 
 \paragraph{Options for graph series}
 
 \begin{itemize}
 \item
   \texttt{'legend='} {[} char \textbar{} cellstr \textbar{} \texttt{NaN}
   \textbar{} \emph{\texttt{Inf}} {]} - (Not inheritable from parent
   objects) Legend entries used instead of the series caption and marks;
   Inf means the caption and marks will be used to construct legend
   entries; NaN means the series will be exluded from legend.
 \item
   \texttt{'plotFunc='} {[} \texttt{@area} \textbar{} \texttt{@bar}
   \textbar{} \texttt{@barcon} \textbar{} \emph{\texttt{@plot}}
   \textbar{} \texttt{@plotcmp} \textbar{} \texttt{@plotpred} \textbar{}
   \texttt{@stem} {]} - (Inheritable from parent objects) Plot function
   that will be used to create graphs.
 \item
   \texttt{'plotOptions='} {[} cell \textbar{} \emph{empty} {]} - Options
   passed as the last input arguments to the plot function.
 \item
   \texttt{yAxis='} {[} \emph{\texttt{'left'}} \textbar{}
   *\texttt{'right'} {]} - Choose the LHS or RHS axis to plot this
   series; see also comments on LHS-RHS plots in Description.
 \end{itemize}
 
 \paragraph{Generic options}
 
 See help on \href{report/Contents}{generic options} in report objects.
 
 \paragraph{Description}
 
 \subparagraph{Using the \texttt{'nan='}, \texttt{'inf='},
 \texttt{'pureZero='} and \texttt{'printedZero='} options}
 
 When specifying the LaTeX string for these options, bear in mind that
 the table entries are printed in the math model. This means that
 whenever you wish to print a normal text, you need to use an appropriate
 text formatting command allowed within a math mode. Most frequently, it
 would be \texttt{'\textbackslash{}textnormal\{...\}'}.
 
 \subparagraph{Using the \texttt{'plotFunc='} option}
 
 When you set the option to \texttt{'plotpred'}, the input data
 \texttt{X} (second input argument) must be a multicolumn tseries object
 where the first column is the time series observations, and the second
 and further columns are its Kalman filter predictions as returned by the
 \texttt{filter} function.
 
 \subparagraph{Conditional formatting}
 
 The conditional format struct (or an array of structs) specified through
 the \texttt{'condFormat='} option must have two fields, \texttt{.test}
 and \texttt{.format}.
 
 The \texttt{.test} field is a text string with a Matlab expression. The
 expression must evaluate to a scalar true or false, and can refer to the
 following attributes associated with each entry in the data part of the
 table:
 
 \begin{itemize}
 \item
   \texttt{value} - the numerical value of the entry,
 \item
   \texttt{date} - the date under which the entry appears,
 \item
   \texttt{year} - the year under which the entry appears,
 \item
   \texttt{period} - the period within the year (e.g.~month or quarter)
   under which the entry appears,
 \item
   \texttt{freq} - the frequency of the date under which the entry
   appears,
 \item
   \texttt{text} - the text label on the left,
 \item
   \texttt{mark} - the text mark on the left used to describe the
   individual rows reported for multivariate series,
 \item
   \texttt{row} - the row number within a multivariate series.
 \item
   \texttt{rowvalues} - a row vector of all values on the current row.
 \end{itemize}
 
 If the table is based on user-defined structure of columns (option
 \texttt{'colstruct='} in \href{report/table}{\texttt{table}}), the
 following additional attributes are available
 
 \begin{itemize}
 \item
   \texttt{colname} - descriptor of the column (text in the headline).
 \end{itemize}
 
 You can combine a number of attribues within one test, using the logical
 operators, e.g.
 
 \begin{verbatim}
 'value > 0 && year > 2010'
 \end{verbatim}
 
 The \texttt{.format} fields of the conditional format structure consist
 of LaTeX commands that will be use to typeset the corresponding entry.
 The reference to the entry itself is through a question mark. The
 entries are typeset in math mode; this for instance meanse that for bold
 or italic typface, you must use the
 \texttt{\textbackslash{}mathbf\{...\}} and
 \texttt{\textbackslash{}mathit\{...\}} commands.
 
 In addition to standard LaTeX commands, you can use the following
 IRIS-specific commands in the format strings:
 
 \begin{itemize}
 \item
   \texttt{\textbackslash{}sprintf\{FFFF\}} - to modify the way each
   numeric entry that passes the test is printed by the \texttt{sprintf}
   function; \texttt{FFFF} is one of the standard sprintf formattting
   strings.
 \item
   \texttt{\textbackslash{}hide\{?\}} - to hide the actual entry when it
   is supposed to be replaced with something else.
 \end{itemize}
 
 You can combine multiple tests and their correponding formats in one
 structure; they will be all applied to each entry in the specified
 order.
 
 \subparagraph{LHS-RHS plots}
 
 The LHS-RHS report graphs are still an experimental feature.
 
 When the option \texttt{'yAxis='} is used to plot on both the LHS and
 the RHS y-axis, the plot functions are restricted to \texttt{@plot},
 \texttt{@bar}, \texttt{@area} and \texttt{@stem}. Also, because of a bug
 in Matlab, always control the color of the lines, bars and areas in all
 LHS-RHS graphs: use either the option \texttt{'plotOptions='} in this
 command, or \texttt{'style='} in the respective
 \href{report/graph}{\texttt{graph}} command.
 
 \paragraph{Example of a conditional format structure}
 
 \begin{verbatim}
 cf = struct();
 cf(1).test = 'value < 0';
 cf(1).format = '\mathit{?}';
 cf(2).test = 'date < qq(2010,1)';
 cf(2).format = '\color{blue}';
 \end{verbatim}


