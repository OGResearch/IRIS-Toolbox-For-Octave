

    \filetitle{vline}{Add vertical line to graph}{report/vline}

	\paragraph{Syntax}\label{syntax}

\begin{verbatim}
P.vline(Caption,Date,...)
\end{verbatim}

\paragraph{Input arguments}\label{input-arguments}

\begin{itemize}
\item
  \texttt{P} {[} struct {]} - Report object created by the
  \href{report/new}{\texttt{report.new}} function.
\item
  \texttt{Caption} {[} char {]} - Caption used to annotate the vertical
  line.
\item
  \texttt{Date} {[} numeric {]} - Date at which the vertical line will
  be plotted.
\end{itemize}

\paragraph{Options}\label{options}

\begin{itemize}
\item
  \texttt{'hPosition='} {[} \texttt{'bottom'} \textbar{}
  \texttt{'middle'} \textbar{} \emph{\texttt{'top'}} {]} - (Inheritable
  from parent objects) Horizontal position of the caption.
\item
  \texttt{'vPosition='} {[} \texttt{'centre'} \textbar{} \texttt{'left'}
  \textbar{} \emph{\texttt{'right'}} {]} - (Inheritable from parent
  objects) Vertical position of the caption relative to the line.
\item
  \texttt{'timePosition='} {[} \texttt{'after'} \textbar{}
  \texttt{'before'} \textbar{} \texttt{'middle'} {]} - Placement of the
  vertical line on the time axis: in the middle of the specified period,
  immediately before it (between the specified period and the previous
  one), or immediately after it (between the specified period and the
  next one).
\end{itemize}

\paragraph{Generic options}\label{generic-options}

See help on \href{report/Contents}{generic options} in report objects.

\paragraph{Description}\label{description}

\paragraph{Example}\label{example}


