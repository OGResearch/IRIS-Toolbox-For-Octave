

    \filetitle{tex}{Include \LaTeX\ code or verbatim text input in report}{report/tex}

	\paragraph{Syntax with input specified in comment block}
 
 \begin{verbatim}
 P.tex(Cap,...)
 
 %{
 Write text or \LaTeX\ code as a block comment
 right after the P.tex(...) command.
 %}
 \end{verbatim}
 
 \paragraph{Syntax with input specified as char argument}
 
 \begin{verbatim}
 P.tex(Cap,Code,...)
 \end{verbatim}
 
 \paragraph{Input arguments}
 
 \begin{itemize}
 \item
   \texttt{P} {[} struct {]} - Report object created by the
   \href{report/new}{\texttt{report.new}} function.
 \item
   \texttt{Cap} {[} char {]} - Caption displayed at the top of the text.
 \item
   \texttt{Code} {[} char {]} - \LaTeX~code or text input that will be
   included in the report.
 \end{itemize}
 
 \paragraph{Options}
 
 \begin{itemize}
 \item
   \texttt{'centering='} {[} \texttt{true} \textbar{}
   \emph{\texttt{false}} {]} - (Inheritable from parent objects) Centre
   the \LaTeX~code or text input on the page.
 \item
   \texttt{'footnote='} {[} char \textbar{} \emph{empty} {]} - Footnote
   at the tex block title; only shows if the title is non-empty.
 \item
   \texttt{'separator='} {[} char \textbar{}
   \emph{\texttt{'\textbackslash{}medskip\textbackslash{}par'}} {]} -
   (Inheritable from parent objects) LaTeX commands that will be inserted
   after the text.
 \item
   \texttt{'verbatim='} {[} \texttt{true} \textbar{}
   \emph{\texttt{false}} {]} - If true the text will be typeset verbatim
   in monospaced font; if false the text will be treated as \LaTeX\\code
   included in the report.
 \end{itemize}
 
 \paragraph{Generic options}
 
 See help on \href{report/Contents}{generic options} in report objects.
 
 \paragraph{Description}
 
 \paragraph{Example}


