

    \filetitle{align}{Vertically align the following K objects}{report/align}

	\paragraph{Syntax}
 
 \begin{verbatim}
 P.align(Caption,K,NCol,...)
 \end{verbatim}
 
 \paragraph{Input arguments}
 
 \begin{itemize}
 \item
   \texttt{P} {[} struct {]} - Report object created by the
   \href{report/new}{\texttt{report.new}} function.
 \item
   \texttt{Caption} {[} char {]} - Caption displayed only when describing
   the structure of the report on the screen, but not in the actual PDF
   report.
 \item
   \texttt{K} {[} numeric {]} - Number of objects following this
   \texttt{align} that will be vertically aligned.
 \item
   \texttt{NCol} {[} numeric {]} - Number of columns in which the objects
   will vertically aligned.
 \end{itemize}
 
 \paragraph{Options}
 
 \begin{itemize}
 \item
   \texttt{'hspace='} {[} numeric \textbar{} \emph{\texttt{2}} {]} -
   Horizontal space (in em units) inserted between two neighbouring
   objects.
 \item
   \texttt{'separator='} {[} char \textbar{}
   \emph{\texttt{'\textbackslash{}medskip\textbackslash{}par}} {]} -
   (Inheritable from parent objects) \LaTeX~commands that will be
   inserted after the aligned objects.
 \item
   \texttt{'shareCaption='} {[} \emph{\texttt{'auto'}} \textbar{} true
   \textbar{} false {]} - (Inheritable from parent objects) Place a
   shared caption (title and subtitle) over each row of objects; the
   title of the first object in each row is used; \texttt{'auto'} means
   that the caption will be shared if they are identical for all objects
   in a row.
 \item
   \texttt{'typeface='} {[} char \textbar{} \emph{empty} {]} - (Not
   inheritable from parent objects) \LaTeX~code specifying the typeface
   for the align element as a whole; it must use the declarative forms
   (such as \texttt{\textbackslash{}itshape}) and not the command forms
   (such as \texttt{\textbackslash{}textit\{...\}}).
 \end{itemize}
 
 \paragraph{Description}
 
 Vertically aligned can be the following types of objects:
 
 \begin{itemize}
 \item
   \href{report/figure}{\texttt{figure}}
 \item
   \href{report/table}{\texttt{table}}
 \item
   \href{report/matrix}{\texttt{matrix}}
 \item
   \href{report/array}{\texttt{array}}
 \end{itemize}
 
 Note that the \texttt{align} object itself has no caption (even if you
 specify one it will not be used). Only the objects within \texttt{align}
 will be given captions. If the objects aligned on one row have identical
 captions (i.e.~both titles and subtitles), only one caption will be
 displayed centred above the objects.
 
 Because \href{report/empty}{\texttt{empty}} objects count in the total
 number of objects inluded in \texttt{align}, you can use
 \href{report/empty}{\texttt{empty}} in to create blank space in a
 particular position.
 
 \paragraph{Example}


