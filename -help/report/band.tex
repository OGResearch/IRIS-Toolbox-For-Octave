

    \filetitle{band}{Add new data with lower and upper bounds to graph or table}{report/band}

	\paragraph{Syntax}
 
 \begin{verbatim}
 P.series(Caption,X,Low,High,...)
 \end{verbatim}
 
 \paragraph{Input arguments}
 
 \begin{itemize}
 \item
   \texttt{P} {[} struct {]} - Report object created by the
   \href{report/new}{\texttt{report.new}} function.
 \item
   \texttt{Caption} {[} char {]} - Caption used as a default legend entry
   in a graph, or in the leading column in a table.
 \item
   \texttt{X} {[} tseries {]} - Input data with the centre of the band.
 \item
   \texttt{Low} {[} tseries {]} - Input data with lower bounds; can be
   specified either relative to the centre or absolute, see the option
   \texttt{'relative='}.
 \item
   \texttt{High} {[} tseries {]} - Input data with upper bounds; can be
   specified either relative to the centre or absolute, see the option
   \texttt{'relative='}.
 \end{itemize}
 
 \paragraph{Options for table and graph bands}
 
 \begin{itemize}
 \item
   \texttt{'low='} {[} char \textbar{} \emph{`Low'} {]} - (Inheritable
   from parent objects) Mark used to denote the lower bound.
 \item
   \texttt{'high='} {[} char \textbar{} \emph{`High'} {]} - (Inheritable
   from parent objects) Mark used to denote the upper bound.
 \item
   \texttt{'relative='} {[} \emph{\texttt{true}} \textbar{}
   \texttt{false} {]} - (Inheritable from parent objects) If true, the
   data for the lower and upper bounds are relative to the centre,
   i.e.~the bounds will be added to the centre (in this case,
   \texttt{LOW} must be negative numbers and \texttt{HIGH} must be
   positive numbers). If false, the bounds are absolute data (in this
   case \texttt{LOW} must be lower than \texttt{X}, and \texttt{HIGH}
   must be higher than \texttt{X}).
 \end{itemize}
 
 \paragraph{Options for table bands}
 
 \begin{itemize}
 \item
   \texttt{'bandTypeface='} {[} char \textbar{}
   \emph{\texttt{'\textbackslash{}footnotesize'}} {]} - (Inheritable from
   parent objects) LaTeX format string used to typeset the lower and
   upper bounds.\%
 \end{itemize}
 
 \paragraph{Options for graph bands}
 
 \begin{itemize}
 \item
   \texttt{'plotType='} {[} \texttt{'errorbar'} \textbar{}
   \emph{\texttt{'patch'}} {]} - Type of plot used to draw the band.
 \item
   \texttt{'relative='} {[} \emph{\texttt{true}} \textbar{}
   \texttt{false} {]} - (Inheritable from parent objects) If true the
   lower and upper bounds will be, respectively, subtracted from and
   added to to the middle line.
 \item
   \texttt{'white='} {[} numeric \textbar{} \emph{\texttt{0.8}} {]} -
   (Inheritable from parent objects) Proportion of white colour mixed
   with the line colour and used to fill the area that depicts the band.
 \end{itemize}
 
 See help on \url{report/series} for other options available.
 
 \paragraph{Generic options}
 
 See help on \href{report/Contents}{generic options} in report objects.
 
 \paragraph{Description}
 
 \paragraph{Example}


