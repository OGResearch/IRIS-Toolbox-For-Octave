

    \filetitle{new}{Create new, empty report object}{report/new}

	\paragraph{Syntax}\label{syntax}

\begin{verbatim}
P = report.new(Caption,...)
\end{verbatim}

\paragraph{Output arguments}\label{output-arguments}

\begin{itemize}
\item
  \texttt{P} {[} struct {]} - Report object with function handles
  through wich the individual report elements can be created.
\item
  \texttt{Caption} {[} char {]} - Report caption; the caption will also
  be printed on the title page of the report if published with the
  option \texttt{'makeTitle='} \texttt{true}.
\end{itemize}

\paragraph{Options}\label{options}

\begin{itemize}
\item
  \texttt{'centering='} {[} \emph{\texttt{true}} \textbar{}
  \texttt{false} {]} - All report elements, except
  \href{report/tex}{\texttt{tex}}, will be centered on the page.
\item
  \texttt{'orientation='} {[} \emph{\texttt{'landscape'}} \textbar{}
  `\texttt{portrait}' {]} - Paper orientation of the published report.
\end{itemize}

Report options are cascading. You can specify any of an object's options
in any of his parent (or ascendant) objects.


