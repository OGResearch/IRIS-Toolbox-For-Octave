

    \filetitle{publish}{Compile PDF from report object}{report/publish}

	\paragraph{Syntax}\label{syntax}

\begin{verbatim}
[OutpFile,Rerun] = P.publish(InpFile,...)
\end{verbatim}

\paragraph{Input arguments}\label{input-arguments}

\begin{itemize}
\item
  \texttt{P} {[} struct {]} - Report object created by the
  \texttt{report.new} function.
\item
  \texttt{InpFile} {[} char {]} - File name under which the compiled PDF
  will be saved.
\end{itemize}

\paragraph{Output arguments}\label{output-arguments}

\begin{itemize}
\item
  \texttt{OutpFile} {[} char {]} - Name of the resulting PDF.
\item
  \texttt{Rerun} {[} numeric {]} - Number of times the LaTeX compiler
  was run to compile the PDF.
\end{itemize}

\paragraph{Options}\label{options}

\begin{itemize}
\item
  \texttt{'abstract='} {[} char \textbar{} \emph{empty} {]} - Abstract
  that will displayed on the title page.
\item
  \texttt{'abstractWidth='} {[} numeric \textbar{} \emph{\texttt{1}} {]}
  - Width of the abstract on the page as a percentage of the full
  default width (between \texttt{0} and \texttt{1}).
\item
  \texttt{'author='} {[} char \textbar{} \emph{empty} {]} - List of
  authors on the title page separated with \texttt{\textbackslash{}and}
  or \texttt{\textbackslash{}\textbackslash{}}.
\item
  \texttt{'cleanup='} {[} \emph{\texttt{true}} \textbar{} \texttt{false}
  {]} - Delete all temporary files created when compiling the report.
\item
  \texttt{'compile='} {[} \emph{\texttt{true}} \textbar{} \texttt{false}
  {]} - Compile the source files to an actual PDF; if \texttt{false}
  only the source files are created.
\item
  \texttt{'date='} {[} char \textbar{}
  \emph{\texttt{'\textbackslash{}today'}} {]} - Date on the title page.
\item
  \texttt{'display='} {[} \emph{\texttt{true}} \textbar{} \texttt{false}
  {]} - Display the \LaTeX compiler report on the final iteration.
\item
  \texttt{'echo='} {[} \texttt{true} \textbar{} \emph{\texttt{false}}
  {]} - If \texttt{true}, the optional flag \texttt{'-echo'} will be
  used in the Matlab function \texttt{system} when compiling the PDF;
  this causes the screen output and all prompts to be displayed for each
  run of the compiler.
\item
  \texttt{'epsToPdf='} {[} char \textbar{} \emph{\texttt{Inf}} {]} -
  Command line arguments for EPSTOPDF; \texttt{Inf} means OS-specific
  arguments are used.
\item
  \texttt{'fontEnc='} {[} char \textbar{} \emph{\texttt{'T1'}} {]} -
  \LaTeX~font encoding.
\item
  \texttt{'makeTitle='} {[} \texttt{true} \textbar{}
  \emph{\texttt{false}} {]} - Produce title page (with title, author,
  date, and abstract).
\item
  \texttt{'package='} {[} char \textbar{} cellstr \textbar{}
  \emph{empty} {]} - Package or list of packages that will be imported
  in the preamble of the LaTeX file.
\item
  \texttt{'paperSize='} {[} \texttt{'a4paper'} \textbar{}
  \emph{\texttt{'letterpaper'}} {]} - Paper size.
\item
  \texttt{'orientation='} {[} \emph{\texttt{'landscape'}} \textbar{}
  \texttt{'portrait'} {]} - Paper orientation.
\item
  \texttt{'preamble='} {[} char \textbar{} \emph{empty} {]} -
  \LaTeX~commands that will be placed in the \LaTeX~file preamble.
\item
  \texttt{'timeStamp='} {[} char \textbar{}
  \emph{\texttt{'datestr(now())'}} {]} - String printed in the top-left
  corner of each page.
\item
  \texttt{'tempDir='} {[} char \textbar{} \emph{\texttt{tempname(cd())}}
  {]} - Directory for storing temporary files; the directory is deleted
  at the end of the execution if it's empty.
\item
  \texttt{'maxRerun='} {[} numeric \textbar{} \emph{\texttt{5}} {]} -
  Maximum number of times the \LaTeX\\compiler will be run to resolve
  cross-references, etc.
\item
  \texttt{'minRerun='} {[} numeric \textbar{} \emph{\texttt{1}} {]} -
  Minimum number of times the \LaTeX\\compiler will be run to resolve
  cross-references, etc.
\item
  \texttt{'textScale='} {[} numeric \textbar{} \emph{\texttt{0.8}} {]} -
  Percentage of the total page area that will be used.
\end{itemize}

\paragraph{Description}\label{description}

\subparagraph{Difference between \texttt{'display='} and
\texttt{'echo='}}\label{difference-between-display-and-echo}

There are two differences between these otherwise similar options:

\begin{itemize}
\item
  When publishing the final PDF, the PDFLaTeX compiler may be called
  more than once to resolve cross-references, the table of contents, and
  so on. Setting \texttt{'display=' true} only displays the screen
  output from the final iteration only, while \texttt{'echo=' true}
  displays the screen outputs from all iterations.
\item
  In the case of a compiler error unrelated to the \LaTeX~code, the
  compiler may stop and prompt the user to respond. The prompt only
  appears on the screen when \texttt{'echo=' true}. Otherwise, Matlab
  may remain in a busy state with no on-screen information, and
  \texttt{Ctrl+C} may be needed to regain control.
\end{itemize}

\paragraph{Example}\label{example}


