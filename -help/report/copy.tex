

    \filetitle{copy}{Create a copy of a report object}{report/copy}

	\paragraph{Syntax}
 
 \begin{verbatim}
 Q = copy(P)
 \end{verbatim}
 
 \paragraph{Input arguments}
 
 \begin{itemize}
 \item
   \texttt{P} {[} report {]} - Report object whose copy will be created.
 \end{itemize}
 
 \paragraph{Output arguments}
 
 \begin{itemize}
 \item
   \texttt{Q} {[} report {]} - Copy of the input report object.
 \end{itemize}
 
 \paragraph{Description}
 
 Because \texttt{report} is a handle class object, a plain assignment
 
 \begin{verbatim}
 Q = P;
 \end{verbatim}
 
 creates a handle to the same copy of a report object. In other words,
 changes in \texttt{Q} will also change \texttt{P} and vice versa. To
 make a new, independent copy of an existing report object, you need to
 run
 
 \begin{verbatim}
 Q = copy(P);
 \end{verbatim}


