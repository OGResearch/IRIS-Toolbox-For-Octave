

    \filetitle{array}{Insert array with user data}{report/array}

	\paragraph{Syntax}\label{syntax}

\begin{verbatim}
P.array(Caption,Data)
\end{verbatim}

\paragraph{Input arguments}\label{input-arguments}

\begin{itemize}
\item
  \texttt{P} {[} struct {]} - Report object created by the
  \href{report/new}{\texttt{report.new}} function.
\item
  \texttt{Caption} {[} char \textbar{} cellstr {]} - Title or a cell
  array with title and subtitle displayed at the top of the array; see
  Description for splitting the title or subtitle into multiple lines.
\item
  \texttt{Data} {[} cell {]} - Cell array with input data; numeric and
  text entries are allowed.
\end{itemize}

\paragraph{Options}\label{options}

\begin{itemize}
\item
  \texttt{'arrayStretch='} {[} numeric \textbar{} \emph{\texttt{1.15}}
  {]} - (Inherited) Stretch between lines in the array (in pts).
\item
  \texttt{'captionTypeface='} {[} cellstr \textbar{} char \textbar{}
  \emph{\texttt{'\textbackslash{}large\textbackslash{}bfseries'}} {]} -
  (Inherited) \LaTeX~format commands for typesetting the array caption;
  the subcaption format can be entered as the second cell in a cell
  array.
\item
  \texttt{'colWidth='} {[} numeric \textbar{} \emph{\texttt{NaN}} {]} -
  (Inheritable from parent objects) Width, or a vector of widhts, of the
  array columns in \texttt{em}units; \texttt{NaN} means the width of the
  column will adjust automatically.
\item
  \texttt{'format='} {[} char \textbar{} \emph{\texttt{'\%.2f'}} {]} -
  (Inherited) Numeric format string; see help on the built-in
  \texttt{sprintf} function.
\item
  \texttt{'footnote='} {[} char \textbar{} \emph{empty} {]} -
  (Inherited) Footnote at the array title; only shows if the title is
  non-empty.
\item
  \texttt{'heading='} {[} char \textbar{} cellstr \textbar{}
  \emph{empty} {]} - (Inherited) User-supplied heading, i.e.~an extra
  row or rows at the top of the array. The heading can be either a
  \LaTeX~code, or a cell array whose size is consistent with
  \texttt{Data}. The heading is repeated at the top of each new page
  when used with \texttt{'long=' true}.
\item
  \texttt{'inf='} {[} char \textbar{}
  \emph{\texttt{'\$\textbackslash{}infty\$'}} {]} - (Inherited)
  \LaTeX~string that will be used to typeset \texttt{Inf}s.
\item
  \texttt{'long='} {[} \texttt{true} \textbar{} \emph{\texttt{false}}
  {]} - (Inherited) If \texttt{true}, the array may stretch over more
  than one page.
\item
  \texttt{'longFoot='} {[} char \textbar{} \emph{empty} {]} -
  (Inherited) Footnote that appears at the bottom of the array (if it is
  longer than one page) on each page except the last one; works only
  with \texttt{'long='} \texttt{true}.
\item
  \texttt{'longFootPosition='} {[} \texttt{'centre'} \textbar{}
  \emph{\texttt{'left'}} \textbar{} \texttt{'right'} {]} - (Inherited)
  Horizontal alignment of the footnote in long arrays; works only with
  \texttt{'long='} \texttt{true}.
\item
  \texttt{'nan='} {[} char \textbar{}
  \emph{\texttt{'\$\textbackslash{}cdots\$'}} {]} - (Inherited)
  \LaTeX~string that will be used to typeset \texttt{NaN}s.
\item
  \texttt{'pureZero='} {[} char \textbar{} \emph{empty} {]} -
  (Inherited) \LaTeX~string that will be used to typeset pure zero
  entries; if empty the zeros will be printed using the current numeric
  format.
\item
  \texttt{'printedZero='} {[} char \textbar{} \emph{empty} {]} -
  (Inherited) \LaTeX~string that will be used to typeset the entries
  that would appear as zero under the current numeric format used; if
  empty these numbers will be printed using the current numeric format.
\item
  \texttt{'separator='} {[} char \textbar{}
  \emph{\texttt{'\textbackslash{}medskip\textbackslash{}par'}} {]} -
  (Inherited) \LaTeX\\commands that will be inserted after the array.
\item
  \texttt{'sideways='} {[} \texttt{true} \textbar{}
  \emph{\texttt{false}} {]} - (Inherited) Print the array rotated by 90
  degrees.
\item
  \texttt{'tabcolsep='} {[} \texttt{NaN} \textbar{} numeric {]} -
  (Inherited) Space between columns in the array, measured in em units;
  \texttt{NaN} means the \LaTeX~default.
\item
  \texttt{'typeface='} {[} char \textbar{} \emph{empty} {]} - (Not
  inherited) \LaTeX~code specifying the typeface for the array as a
  whole; it must use the declarative forms (such as
  \texttt{\textbackslash{}itshape}) and not the command forms (such as
  \texttt{\textbackslash{}textit\{...\}}).
\end{itemize}

\paragraph{Generic options}\label{generic-options}

See help on \href{report/Contents}{generic options} in report objects.

\paragraph{Description}\label{description}

The input cell array \texttt{Data} can contain either strings or numeric
values, or horizontal rules. Numeric values are printed using the
standard \texttt{sprintf} function and formatted using the
\texttt{'format='} option. Horizontal rules must be entered as a string
of five (or more) dashes, \texttt{'-{}-{}-{}-{}-'}, in the first cell of
the respective row, with all other cells empty in that row. If you wish
to include a \LaTeX~command or a piece of \LaTeX~code, you must enclose
it in curly brackets.

\subparagraph{Titles and subtitles}\label{titles-and-subtitles}

The input argument \texttt{Caption} can be either a text string, or a
1-by-2 cell array of strings. In the latter case, the first cell will be
printed as a title, and the second cell will be printed as a subtitle.

To split the title or subtitle into multiple lines, use the following
LaTeX commands wrapped in curly brackets:
\texttt{\{\textbackslash{}\textbackslash{}\}} or
\texttt{\{\textbackslash{}\textbackslash{}{[}Xpt{]}\}}, where \texttt{X}
is the width of an extra vertical space (in points) added between the
respective lines.

\paragraph{Example}\label{example}

These commands create a table with two rows separated by a horizontal
rule, and three columns in each of them. The middle columns will have
Greek letters printed in \LaTeX~math mode.

\begin{verbatim}
A = { ...
    'First row','{\$\textbackslash alpha\$}',10000; ...
    '-----','',''; ...
    'Second row,'{\$\textbackslash beta\$}',20000; ...
};

P.array('My table',A);
\end{verbatim}


