

    \filetitle{fanchart}{Add fanchart to graph}{report/fanchart}

	\paragraph{Syntax}
 
 \begin{verbatim}
 P.fanchart(Cap,X,Std,Prob,...)
 \end{verbatim}
 
 \paragraph{Input arguments}
 
 \begin{itemize}
 \item
   \texttt{P} {[} struct {]} - Report object created by the
   \href{report/new}{\texttt{report.new}} function.
 \item
   \texttt{Cap} {[} char {]} - Caption used as a legend entry for the
   line (mean of fanchart)
 \item
   \texttt{X} {[} tseries {]} - Tseries object with input data to be
   displayed.
 \item
   \texttt{Std} {[} tseries {]} - Tseries object with standard deviations
   of input data.
 \item
   \texttt{Prob} {[} numeric {]} - Confidence porbabilities of intervals
   to be displayed.
 \end{itemize}
 
 \paragraph{Options for fancharts}
 
 \begin{itemize}
 \item
   \texttt{'asym='} {[} numeric \textbar{} tseries \textbar{}
   \emph{\texttt{1}} {]} - Ratio of asymmetry (area of upper part to one
   of lower part).
 \item
   \texttt{'exclude='} {[} numeric \textbar{} true \textbar{}
   \emph{\texttt{false}} {]} - Exclude some of the confidence intervals.
 \item
   \texttt{'factor='} {[} numeric \textbar{} \emph{\texttt{1}} {]} -
   factor to increase or decrease input standard deviations
 \item
   \texttt{'fanLegend='} {[} cell \textbar{} \texttt{NaN} \textbar{}
   \emph{\texttt{Inf}} {]} - Legend entries used instead of confidence
   interval values; Inf means all confidence intervals values will be
   used to construct legend entries; NaN means the intervals will be
   exluded from legend; \texttt{NaN} in cellstr means the intervals of
   respective fancharts will be exluded from legend.
 \end{itemize}
 
 See help on \url{report/series} for other options available.
 
 \paragraph{Description}
 
 The confidence intervals are based on normal distributions with standard
 deviations supplied by the user. Optionally, the user can also specify
 assumptions about asymmetry and/or common correction factors.
 
 \paragraph{Example}


