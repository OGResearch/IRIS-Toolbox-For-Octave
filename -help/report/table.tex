

    \filetitle{table}{Start new table}{report/table}

	\paragraph{Syntax}\label{syntax}

\begin{verbatim}
P.table(Caption,...)
\end{verbatim}

\paragraph{Input arguments}\label{input-arguments}

\begin{itemize}
\item
  \texttt{P} {[} report {]} - Report object created by the
  \href{report/new}{\texttt{report.new}} function.
\item
  \texttt{Caption} {[} char \textbar{} cellstr {]} - Title or a cell
  array with title and subtitle displayed at the top of the table; see
  Description for splitting the title or subtitle into multiple lines.
\end{itemize}

\paragraph{Options}\label{options}

\begin{itemize}
\item
  \texttt{'arrayStretch='} {[} numeric \textbar{} \emph{\texttt{1.15}}
  {]} - (Inheritable from parent objects) Stretch between lines in the
  table (in pts).
\item
  \texttt{'captionTypeface='} {[} cell \textbar{}
  \emph{\texttt{'\textbackslash{}large\textbackslash{}bfseries'}} {]} -
  LaTeX format commands for typesetting the table caption and
  subcaption; you can use Inf for either to indicate the default format.
\item
  \texttt{'colStruct='} {[} struct \textbar{} \emph{empty} {]} - (Not
  inheritable from parent objects) User-defined structure of the table
  columns; use of this option disables \texttt{'range='}.
\item
  \texttt{'colWidth='} {[} numeric \textbar{} \emph{\texttt{NaN}} {]} -
  (Inheritable from parent objects) Width, or a vector of widhts, of the
  table columns in \texttt{em}units; \texttt{NaN} means the width of the
  column will adjust automatically.
\item
  \texttt{'headlineJustify='} {[} \emph{\texttt{'c'}} \textbar{}
  \texttt{'l'} \textbar{} \texttt{'r'} {]} - Horizontal alignment of the
  headline entries (individual dates or user-defined text): Centre,
  Left, Right.
\item
  \texttt{'dateFormat='} {[} char \textbar{} cellstr \textbar{}
  \emph{\texttt{irisget('dateformat')}} {]} - (Inheritable from parent
  objects) Format string for the date row.
\item
  \texttt{'footnote='} {[} char \textbar{} \emph{empty} {]} - Footnote
  at the table title; only shows if the title is non-empty.
\item
  \texttt{'long='} {[} true \textbar{} \emph{\texttt{false}} {]} -
  (Inheritable from parent objects) If true, the table may stretch over
  more than one page.
\item
  \texttt{'longFoot='} {[} char \textbar{} \emph{empty} {]} -
  (Inheritable from parent objects) Works only with
  \texttt{'long='}=true: Footnote that appears at the bottom of the
  table (if it is longer than one page) on each page except the last
  one.
\item
  \texttt{'longFootPosition='} {[} \texttt{'centre'} \textbar{}
  \emph{\texttt{'left'}} \textbar{} \texttt{'right'} {]} - (Inheritable
  from parent objects) Works only with \texttt{'long='} \texttt{true}:
  Horizontal alignment of the footnote in long tables.
\item
  \texttt{'range='} {[} numeric \textbar{} \emph{empty} {]} -
  (Inheritable from parent objects) Date range or vector of dates that
  will appear as columns of the table.
\item
  \texttt{'separator='} {[} char \textbar{}
  \emph{\texttt{'\textbackslash{}medskip\textbackslash{}par'}} {]} -
  (Inheritable from parent objects) \LaTeX~commands that will be
  inserted after the table.
\item
  \texttt{'sideways='} {[} \texttt{true} \textbar{}
  \emph{\texttt{false}} {]} - (Inheritable from parent objects) Print
  the table rotated by 90 degrees.
\item
  \texttt{'tabcolsep='} {[} \texttt{NaN} \textbar{} numeric {]} -
  (Inheritable from parent objects) Space between columns in the table,
  measured in em units; NaN means the \LaTeX~default.
\item
  \texttt{'typeface='} {[} char \textbar{} \emph{empty} {]} - (Not
  inheritable from parent objects) \LaTeX~code specifying the typeface
  for the table as a whole; it must use the declarative forms (such as
  \texttt{\textbackslash{}itshape}) and not the command forms (such as
  \texttt{\textbackslash{}textit\{...\}}).
\item
  \texttt{'vline='} {[} numeric \textbar{} \emph{empty} {]} -
  (Inheritable from parent objects) Vector of dates after which a
  vertical line (divider) will be placed.
\end{itemize}

\paragraph{Generic options}\label{generic-options}

See help on \href{report/Contents}{generic options} in report objects.

\paragraph{Description}\label{description}

Tables are top-level report objects and cannot be nested within other
report objects, except \href{report/align}{\texttt{align}}. Table
objects can have the following children:

\begin{itemize}
\itemsep1pt\parskip0pt\parsep0pt
\item
  \href{report/series}{\texttt{series}};
\item
  \href{report/subheading}{\texttt{subheading}}.
\end{itemize}

By default, the date row is printed as a leading row with dates formated
using the option \texttt{'dateFormat='}. Alternatively, you can specify
this option as a cell array of two strings. In that case, the dates will
be printed in two rows. The first row will have a date string displayed
and centred for every year, and the first cell of the
\texttt{'dateFormat='} option will be used for formatting. The second
row will have a date displayed for every period (i.e.~every column), and
the second cell of the \texttt{'dateFormat='} option will be used for
formatting.

\subparagraph{User-defined structure of the table
columns}\label{user-defined-structure-of-the-table-columns}

Use can use the \texttt{'columnStruct='} option to define your own table
columns. This gives you more flexibility than when using the
\texttt{'range='} option in defining the content of the table.

The option \texttt{'columnStruct='} must be a 1-by-N struct, where N is
the number of columns you want in the table, with the following fields:

\begin{itemize}
\item
  \texttt{'name='} - specifies the descriptor of the column that will be
  displayed in the headline;
\item
  \texttt{'func='} - specifies a function that will be applied to the
  input series; if \texttt{'func='} is empty, no function will be
  applied. The function must evaluate to a tseries or a numeric scalar.
\item
  \texttt{'date='} - specifies the date at which a number will be taken
  from the series unless the function \texttt{'func='} applied before
  resulted in a numeric scalar.
\end{itemize}

\subparagraph{Titles and subtitles}\label{titles-and-subtitles}

The input argument \texttt{Caption} can be either a text string, or a
1-by-2 cell array of strings. In the latter case, the first cell will be
printed as a title, and the second cell will be printed as a subtitle.

To split the title or subtitle into multiple lines, use the following
LaTeX commands wrapped in curly brackets:
\texttt{\{\textbackslash{}\textbackslash{}\}} or
\texttt{\{\textbackslash{}\textbackslash{}{[}Xpt{]}\}}, where \texttt{X}
is the width of an extra vertical space (in points) added between the
respective lines.

\paragraph{Example}\label{example}

Compare the headers of these two tables:

\begin{verbatim}
x = report.new();

x.table('First table', ...
    'range',qq(2010,1):qq(2012,4), ...
    'dateformat','YYYYFP');
% You can add series or subheadings here.

x.table('Second table', ...
    'range,qq(2010,1):qq(2012,4), ...
    'dateformat',{'YYYY','FP'});
% You can add series or subheadings here.

x.publish('myreport.pdf');
\end{verbatim}


