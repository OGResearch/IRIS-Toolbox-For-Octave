
    \foldertitle{report}{Report functions}{report/Contents}

	\paragraph{New report}\label{new-report}

\begin{itemize}
\itemsep1pt\parskip0pt\parsep0pt
\item
  \href{report/new}{\texttt{new}} - Create new, empty report object.
\item
  \href{report/copy}{\texttt{copy}} - Create a copy of a report object.
\end{itemize}

\paragraph{Compiling PDF report}\label{compiling-pdf-report}

\begin{itemize}
\itemsep1pt\parskip0pt\parsep0pt
\item
  \href{report/publish}{\texttt{publish}} - Compile PDF from report
  object.
\end{itemize}

\paragraph{Top-level objects}\label{top-level-objects}

\begin{itemize}
\itemsep1pt\parskip0pt\parsep0pt
\item
  \href{report/table}{\texttt{table}} - Start new table.
\item
  \href{report/figure}{\texttt{figure}} - Start new figure.
\item
  \href{report/userfigure}{\texttt{usefigure}} - Insert existing figure
  window.
\item
  \href{report/matrix}{\texttt{matrix}} - Insert matrix or numeric
  array.
\item
  \href{report/modelfile}{\texttt{modelfile}} - Write formatted model
  file.
\item
  \href{report/array}{\texttt{array}} - Insert array with user data.
\item
  \href{report/tex}{\texttt{tex}} - Include \LaTeX~code or verbatim text
  input in report.
\end{itemize}

\paragraph{Inspecting and maninpulating report
objects}\label{inspecting-and-maninpulating-report-objects}

\begin{itemize}
\itemsep1pt\parskip0pt\parsep0pt
\item
  \href{report/disp}{\texttt{disp}} - Display the structure of report
  object.
\item
  \href{report/display}{\texttt{display}} - Display the structure of
  report object.
\item
  \href{report/findall}{\texttt{findall}} - Find all objects of a given
  type within report object.
\end{itemize}

\paragraph{Figure objects}\label{figure-objects}

\begin{itemize}
\itemsep1pt\parskip0pt\parsep0pt
\item
  \href{report/graph}{\texttt{graph}} - Add graph to figure.
\end{itemize}

\paragraph{Table and graph objects}\label{table-and-graph-objects}

\begin{itemize}
\itemsep1pt\parskip0pt\parsep0pt
\item
  \href{report/band}{\texttt{band}} - Add new data with lower and upper
  bounds to graph or table.
\item
  \href{report/fanchart}{\texttt{fanchart}} - Add fanchart to graph.
\item
  \href{report/series}{\texttt{series}} - Add new data to graph or
  table.
\item
  \href{report/subheading}{\texttt{subheading}} - Enter subheading in
  table.
\item
  \href{report/vline}{\texttt{vline}} - Add vertical line to graph.
\item
  \href{report/highlight}{\texttt{highlight}} - Highlight range in
  graph.
\end{itemize}

\paragraph{Structuring reports}\label{structuring-reports}

\begin{itemize}
\itemsep1pt\parskip0pt\parsep0pt
\item
  \href{report/align}{\texttt{align}} - Vertically align the following K
  objects.
\item
  \href{report/empty}{\texttt{empty}} - Empty report object.
\item
  \href{report/include}{\texttt{include}} - Include text or LaTeX input
  file in the report.
\item
  \href{report/merge}{\texttt{merge}} - Merge the content of two or more
  report objects.
\item
  \href{report/pagebreak}{\texttt{pagebreak}} - Force page break.
\item
  \href{report/section}{\texttt{section}} - Start new section in report.
\end{itemize}

\paragraph{Getting on-line help on report
functions}\label{getting-on-line-help-on-report-functions}

\begin{verbatim}
help report
help report/function_name
\end{verbatim}

\paragraph{Generic options}\label{generic-options}

The following generic options can be used on any of the report objects.

\begin{itemize}
\itemsep1pt\parskip0pt\parsep0pt
\item
  \texttt{'saveAs='} {[} char \textbar{} \emph{empty} {]} - (Not
  inheritable from parent objects) Save the LaTeX code generated for the
  respective report element in a text file under the specified name.
\end{itemize}



