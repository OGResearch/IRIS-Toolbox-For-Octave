

    \filetitle{qplot}{Quick report}{qreport/qplot}

	\paragraph{Syntax}
 
 \begin{verbatim}
 [FF,AA,PDb] = qplot(QFile,D,Range,...)
 \end{verbatim}
 
 \paragraph{Input arguments}
 
 \begin{itemize}
 \item
   \texttt{QFile} {[} char {]} - Name of the q-file that defines the
   contents of the individual graphs.
 \item
   \texttt{D} {[} struct {]} - Database with input data.
 \item
   \texttt{Range} {[} numeric {]} - Date range.
 \end{itemize}
 
 \paragraph{Output arguments}
 
 \begin{itemize}
 \item
   \texttt{FF} {[} numeric {]} - Handles to figures created by
   \texttt{qplot}.
 \item
   \texttt{AA} {[} cell {]} - Handles to axes created by \texttt{qplot}.
 \item
   \texttt{PDb} {[} struct {]} - Database with actually plotted series.
 \end{itemize}
 
 \paragraph{Options}
 
 \begin{itemize}
 \item
   \texttt{'addClick='} {[} \emph{\texttt{true}} \textbar{}
   \texttt{false} {]} - Make axes expand in a new graphics figure upon
   mouse click.
 \item
   \texttt{'caption='} {[} cellstr \textbar{} @comment \textbar{}
   \emph{empty} {]} - Strings that will be used for titles in the graphs
   that have no title in the q-file.
 \item
   \texttt{'clear='} {[} numeric \textbar{} \emph{empty} {]} - Serial
   numbers of graphs (axes objects) that will not be displayed.
 \item
   \texttt{'dbsave='} {[} cellstr \textbar{} \emph{empty} {]} - Options
   passed to \texttt{dbsave} when \texttt{'saveAs='} is used.
 \item
   \texttt{'drawNow='} {[} \texttt{true} \textbar{} \emph{\texttt{false}}
   {]} - Call Matlab \texttt{drawnow} function upon completion of all
   figures.
 \item
   \texttt{'grid='} {[} \emph{\texttt{true}} \textbar{} \texttt{false}
   {]} - Add grid lines to all graphs.
 \item
   \texttt{'highlight='} {[} numeric \textbar{} cell \textbar{}
   \emph{empty} {]} - Date range or ranges that will be highlighted.
 \item
   \texttt{'interpreter='} {[} \emph{\texttt{'latex'}} \textbar{} `none'
   {]} - Interpreter used in graph titles.
 \item
   \texttt{'mark='} {[} cellstr \textbar{} \emph{empty} {]} - Marks that
   will be added to each legend entry to distinguish individual columns
   of multivariated tseries objects plotted.
 \item
   \texttt{'overflow='} {[} \texttt{true} \textbar{}
   \emph{\texttt{false}} {]} - Open automatically a new figure window if
   the number of subplots exceeds the available total;
   \texttt{'overflow' = false} means an error will occur instead.
 \item
   \texttt{'prefix='} {[} char \textbar{} \emph{\texttt{'P\%g\_'}} {]} -
   Prefix (a \texttt{sprintf} format string) that will be used to precede
   the name of each entry in the \texttt{PDb} database.
 \item
   \texttt{'round='} {[} numeric \textbar{} \emph{\texttt{Inf}} {]} -
   Round the input data to this number of decimals before plotting.
 \item
   \texttt{'saveAs='} {[} char \textbar{} \emph{empty} {]} - File name
   under which the plotted data will be saved either in a CSV data file
   or a PDF; you can use the \texttt{'dbsave='} option to control the
   options used when saving CSV.
 \item
   \texttt{'style='} {[} struct \textbar{} \emph{empty} {]} - Style
   structure that will be applied to all figures and their children
   created by the \texttt{qplot} function.
 \item
   \texttt{'subplot='} {[} \emph{`auto'} \textbar{} numeric {]} - Default
   subplot division of figures, can be modified in the q-file.
 \item
   \texttt{'sstate='} {[} struct \textbar{} model \textbar{} \emph{empty}
   {]} - Database or model object from which the steady-state values
   referenced to in the quick-report file will be taken.
 \item
   \texttt{'style='} {[} struct \textbar{} \emph{empty} {]} - Style
   structure that will be applied to all created figures upon completion.
 \item
   \texttt{'transform='} {[} function\_handle \textbar{} \emph{empty} {]}
   - Function that will be used to trans
 \item
   \texttt{'tight='} {[} \texttt{true} \textbar{} \emph{\texttt{false}}
   {]} - Make the y-axis in each graph tight.
 \item
   \texttt{'vLine='} {[} numeric \textbar{} \emph{empty} {]} - Dates at
   which vertical lines will be plotted.
 \item
   \texttt{'zeroLine='} {[} \texttt{true} \textbar{}
   \emph{\texttt{false}} {]} - Add a horizontal zero line to graphs whose
   y-axis includes zero.
 \end{itemize}
 
 \paragraph{Description}
 
 \paragraph{Example}


