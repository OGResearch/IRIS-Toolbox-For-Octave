

    \filetitle{qstyle}{Apply styles to graphics object and its descandants}{qreport/qstyle}

	\paragraph{Syntax}\label{syntax}

\begin{verbatim}
qstyle(H,S,...)
\end{verbatim}

\paragraph{Input arguments}\label{input-arguments}

\begin{itemize}
\item
  \texttt{H} {[} numeric {]} - Handle to a figure or axes object that
  will be styled together with its descandants (unless
  \texttt{'cascade='} is false).
\item
  \texttt{S} {[} struct {]} - Struct each field of which refers to an
  object-dot-property; the value of the field will be applied to the the
  respective property of the respective object; see below the list of
  graphics objects allowed.
\end{itemize}

\paragraph{Options}\label{options}

\begin{itemize}
\item
  \texttt{'cascade='} {[} \emph{\texttt{true}} \textbar{} \texttt{false}
  {]} - Cascade through all descendants of the object \texttt{H}; if
  false only the object \texttt{H} itself will be styled.
\item
  \texttt{'warning='} {[} \emph{\texttt{true}} \textbar{} \texttt{false}
  {]} - Display warnings produced by this function.
\end{itemize}

\paragraph{Description}\label{description}

The style structure, \texttt{S}, is constructed of any number of nested
object-property fields:

\begin{verbatim}
S.object.property = value;
\end{verbatim}

The following is the list of standard Matlab grahics objects the
top-level fields can refer to:

\begin{itemize}
\itemsep1pt\parskip0pt\parsep0pt
\item
  \texttt{figure};
\item
  \texttt{axes};
\item
  \texttt{title};
\item
  \texttt{xlabel};
\item
  \texttt{ylabel};
\item
  \texttt{zlabel};
\item
  \texttt{line};
\item
  \texttt{bar};
\item
  \texttt{patch};
\item
  \texttt{text}.
\end{itemize}

In addition, you can also refer to the following special instances of
objects created by IRIS functions:

\begin{itemize}
\itemsep1pt\parskip0pt\parsep0pt
\item
  \texttt{rhsaxes} (an RHS axes object created by \texttt{plotyy})
\item
  \texttt{legend} (represented by an axes object);
\item
  \texttt{plotpred} (line objects with prediction data created by
  \texttt{plotpred});
\item
  \texttt{highlight} (a patch object created by \texttt{highlight});
\item
  \texttt{highlightcaption} (a text object created by
  \texttt{highlight});
\item
  \texttt{vline} (a line object created by \texttt{vline});
\item
  \texttt{vlinecaption} (a text object created by \texttt{vline});
\item
  \texttt{zeroline} (a line object created by \texttt{zeroline}).
\end{itemize}

The property used as the second-level field is simply any regular Matlab
property of the respective object (see Matlab help on graphics).

The value assigned to a particular property can be either of the
following:

\begin{itemize}
\item
  a single proper valid value (i.e.~a value you would be able to assign
  using the standard Matlab \texttt{set} function);
\item
  a cell array of multiple different values that will be assigned to the
  objects of the same type in order of their creation;
\item
  a text string starting with a double exclamation point, \texttt{!!},
  followed by Matlab commands. The commands are expected to eventually
  create a variable named \texttt{SET} whose value will then assigned to
  the respective property. The commands have access to variable
  \texttt{H}, a handle to the current object.
\end{itemize}

\paragraph{Example}\label{example}


