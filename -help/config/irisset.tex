

    \filetitle{irisset}{Change configurable IRIS options}{config/irisset}

	\paragraph{Syntax}

\begin{verbatim}
irisset(Option,Value)
irisset(Option,Value,Option,Value,...)
\end{verbatim}

\paragraph{Input arguments}

\begin{itemize}
\item
  \texttt{Option} {[} char {]} - Name of the IRIS configuration option
  that will be modified.
\item
  \texttt{Value} {[} \ldots{} {]} - New value that will be assigned to
  the option.
\end{itemize}

\paragraph{Modifiable IRIS configuration
options}

\subparagraph{Dates and formats}

\begin{itemize}
\item
  \texttt{'dateFormat='} {[} char \textbar{} \emph{\texttt{'YPF'}} {]} -
  Date format used to display dates in the command window, CSV
  databases, and reports. Note that the default date format for graphs
  is controlled by the \texttt{'plotdateformat='} option. The default
  `YFP' means that the year, frequency letter, and period is displayed.
  See also help on \href{dates/dat2str}{\texttt{dat2str}} for more date
  formatting details. The \texttt{'dateformat='} option is also found in
  many IRIS functions whenever it is relevant, and can be used to
  overwrite the \texttt{'irisset='} settings.
\item
  \texttt{'freqLetters='} {[} char \textbar{} \emph{\texttt{'YHQBMW'}}
  {]} - Six letters used to represent the six possible frequencies of
  the IRIS dates: yearly, half-yearly, quarterly, bi-monthly, monthly,
  and weekly, such as the \texttt{'Q'} in \texttt{'2010Q1'} denoting a
  quarter.
\item
  \texttt{'months='} {[} cellstr \textbar{}
  \emph{\texttt{\{'January',...,'December'\}}} {]} - Twelve strings
  representing the names of the twelve months; this option can be used
  whenever you want to replace the default English names with your local
  language. .
\item
  \texttt{'plotDateFormat='} {[} char \textbar{}
  \emph{\texttt{\{'Y','Y:P','Y:P','Y:P','Y:P','Y:P'\}}} {]}
\item
  Default date formats used to display dates in graphs including graphs
  in reports. The default date formats are 6 strings, one for each of
  the date frequencies, in the following order: yearly (1), half-yearly
  (2), quarterly (4), bimonthly (6), monthly (12), and weekly (52).
  Dates with indeterminate frequency are printed as plain numbers.
\item
  \texttt{'tseriesFormat='} {[} char \textbar{} \emph{empty} {]} -
  Format string for displaying time series data on the screen. See help
  on the Matlab \texttt{sprintf} function for how to set up format
  strings. If empty the default format of the \texttt{num2str} function
  is used.
\item
  \texttt{tseriesMaxWSpace='} {[} numeric \textbar{} \emph{\texttt{5}}
  {]} - Maximum number of white spaces printed between individual
  columns of a multivariate tseries object on the screen.
\item
  \texttt{'standinMonth='} {[} \emph{\texttt{'first'}} \textbar{}
  \texttt{'last'} \textbar{} numeric {]} - This option specifies which
  month will be used to represent lower-frequency periods (such as a
  quarters) when a month-displaying format is used in
  \texttt{'dateformat='}.
\end{itemize}

\subparagraph{External tools used by
IRIS}

\begin{itemize}
\item
  \texttt{'pdflatexPath='} {[} char {]} - Location of the
  \texttt{pdflatex.exe} program. This program is called to compile
  report and publish m-files. By default, IRIS attempts to locate
  \texttt{pdflatex.exe} by running TeX's \texttt{kpsewhich}, and
  \texttt{which} on Unix platforms.
\item
  \texttt{'epstopdfPath='} {[} char {]} - Location of the
  \texttt{epstopdf.exe} program. This program is called to convert EPS
  graphics files to PDFs in reports.
\end{itemize}

\paragraph{Description}

\paragraph{Example}


