

    \filetitle{irisset}{Change configurable IRIS options}{config/irisset}

	\paragraph{Syntax}\label{syntax}

\begin{verbatim}
irisset(Option,Value)
irisset(Option,Value,Option,Value,...)
\end{verbatim}

\paragraph{Input arguments}\label{input-arguments}

\begin{itemize}
\item
  \texttt{Option} {[} char {]} - Name of the IRIS configuration option
  that will be modified.
\item
  \texttt{Value} {[} \ldots{} {]} - New value that will be assigned to
  the option.
\end{itemize}

\paragraph{Modifiable IRIS configuration
options}\label{modifiable-iris-configuration-options}

\subparagraph{Dates and formats}\label{dates-and-formats}

\begin{itemize}
\item
  \texttt{'dateFormat='} {[} char \textbar{} \emph{\texttt{'YPF'}} {]} -
  Date format used to display dates in the command window, CSV
  databases, and reports. Note that the default date format for graphs
  is controlled by the \texttt{'plotdateformat='} option. The default
  `YFP' means that the year, frequency letter, and period is displayed.
  See also help on \href{dates/dat2str}{\texttt{dat2str}} for more date
  formatting details. The \texttt{'dateformat='} option is also found in
  many IRIS functions whenever it is relevant, and can be used to
  overwrite the \texttt{'irisset='} settings.
\item
  \texttt{'freqLetters='} {[} char \textbar{} \emph{\texttt{'YHQBMW'}}
  {]} - Six letters used to represent the six possible frequencies of
  IRIS dates, in this order: yearly, half-yearly, quarterly, bi-monthly,
  monthly, and weekly (such as the \texttt{'Q'} in \texttt{'2010Q1'}).
\item
  \texttt{'months='} {[} cellstr \textbar{}
  \emph{\texttt{\{'January',...,'December'\}}} {]} - Twelve strings
  representing the names of the twelve months; this option can be used
  whenever you want to replace the default English names with your local
  language.
\item
  \texttt{'plotDateFormat='} {[} char \textbar{}
  \emph{\texttt{struct('yy','Y','hh','Y:P','qq','Y:P','bb','Y:P','mm','Y:P','ww','Y:P')}}
  {]} - Default date formats used to display dates in graphs including
  graphs in reports. The default date formats are set individually for
  each of the 6 ate frequencies in a struct with the following fields:
  \texttt{.yy}, \texttt{.hh}, \texttt{.qq}, \texttt{.bb}, \texttt{.mm},
  \texttt{.ww}. Dates with indeterminate frequency are printed as plain
  numbers.
\item
  \texttt{'tseriesFormat='} {[} char \textbar{} \emph{empty} {]} -
  Format string for displaying time series data on the screen. See help
  on the Matlab \texttt{sprintf} function for how to set up format
  strings. If empty the default format of the \texttt{num2str} function
  is used.
\item
  \texttt{tseriesMaxWSpace='} {[} numeric \textbar{} \emph{\texttt{5}}
  {]} - Maximum number of white spaces printed between individual
  columns of a multivariate tseries object on the screen.
\item
  \texttt{'standinMonth='} {[} numeric \textbar{} \texttt{'last'}
  \textbar{} \texttt{*1*} {]} - Month that will represent a
  lower-than-monthly-frequency date if the month is part of the date
  format string.
\end{itemize}

\subparagraph{External tools used by
IRIS}\label{external-tools-used-by-iris}

\begin{itemize}
\item
  \texttt{'pdflatexPath='} {[} char {]} - Location of the
  \texttt{pdflatex.exe} program. This program is called to compile
  report and publish m-files. By default, IRIS attempts to locate
  \texttt{pdflatex.exe} by running TeX's \texttt{kpsewhich}, and
  \texttt{which} on Unix platforms.
\item
  \texttt{'epstopdfPath='} {[} char {]} - Location of the
  \texttt{epstopdf.exe} program. This program is called to convert EPS
  graphics files to PDFs in reports.
\end{itemize}

\paragraph{Description}\label{description}

\paragraph{Example}\label{example}


