
    \foldertitle{config}{Starting, quitting, and configuring IRIS}{config/Contents}

	This section describes how to start and quit an IRIS session, and how to
customise some of the IRIS configuration options.

The most common way of starting an IRIS session (after you have
installed the IRIS files on your disk) is to run the following line in
the Matlab command window:

\begin{verbatim}
addpath C:\IRIS_Tbx; irisstartup();
\end{verbatim}

The first command, \texttt{addpath}, adds the IRIS root folder to the
Matlab search path. The second command, \texttt{irisstartup},
initialises IRIS and puts the other necessary IRIS subfolders, classes,
and internal packages on the search path. \emph{Never} add these other
subfolders, classes and packages to the search path by yourself.

\paragraph{Starting and quitting IRIS}

\begin{itemize}
\itemsep1pt\parskip0pt\parsep0pt
\item
  \href{config/irisstartup}{\texttt{irisstartup}} - Start an IRIS
  session.
\item
  \href{config/irisfinish}{\texttt{irisfinish}} - Close the current IRIS
  session.
\item
  \href{config/iriscleanup}{\texttt{iriscleanup}} - Remove IRIS from
  Matlab and clean up.
\end{itemize}

\paragraph{Getting information about
IRIS}

\begin{itemize}
\itemsep1pt\parskip0pt\parsep0pt
\item
  \href{config/irisget}{\texttt{irisget}} - Query current IRIS config
  options.
\item
  \href{config/irisroot}{\texttt{irisroot}} - Current IRIS root folder.
\item
  \href{config/irisrequired}{\texttt{irisrequired}} - Throw error if the
  installed version of IRIS fails to comply with the required minimum.
\item
  \href{config/irisversion}{\texttt{irisversion}} - Current IRIS
  version.
\end{itemize}

\paragraph{Changes in configuration}

\begin{itemize}
\itemsep1pt\parskip0pt\parsep0pt
\item
  \href{config/irisset}{\texttt{irisset}} - Change configurable IRIS
  options.
\item
  \href{config/irisreset}{\texttt{irisreset}} - Reset IRIS configuration
  options to start-up values.
\item
  \href{config/irisuserconfighelp}{\texttt{irisuserconfig}} - User
  configuration file called at the IRIS start-up.
\end{itemize}

\paragraph{Getting on-line help on configuration
functions}

\begin{verbatim}
help config
help function_name
\end{verbatim}



