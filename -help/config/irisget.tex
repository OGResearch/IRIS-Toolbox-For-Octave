

    \filetitle{irisget}{Query current IRIS config options}{config/irisget}

	\paragraph{Syntax}
 
 \begin{verbatim}
 Value = irisget(Option)
 S = irisget()
 \end{verbatim}
 
 \paragraph{Input arguments}
 
 \begin{itemize}
 \item
   \texttt{Option} {[} char {]} - Name of the queried IRIS configuration
   option.
 \end{itemize}
 
 \paragraph{Output arguments}
 
 \begin{itemize}
 \item
   \texttt{Value} {[} \ldots{} {]} - Current value of the queried
   configuration option.
 \item
   \texttt{S} {[} struct {]} - Structure with all configuration options
   and their current values.
 \end{itemize}
 
 \paragraph{Description}
 
 You can view any of the modifiable options listed in
 \href{config/irisset}{\texttt{irisset}}, plus the following
 non-modifiable ones (these cannot be changed by the user):
 
 \begin{itemize}
 \item
   \texttt{'userConfigPath='} {[} char {]} - The path to the user
   configuration file called by the last executed
   \href{config/irisstartup}{\texttt{irisstartup}}.
 \item
   \texttt{'irisRoot='} {[} char {]} - The current IRIS root directory.
 \item
   \texttt{'version='} {[} char {]} - The current IRIS version string.
 \end{itemize}
 
 When called without any input arguments, the \texttt{irisget} function
 returns a struct with all options and their current values.
 
 \paragraph{Example}
 
 \begin{verbatim}
 irisget('dateformat')
 
 ans =
 
 YFP
 
 g = irisget();
 g.dateformat
 
 ans =
 
 YFP
 \end{verbatim}


