

    \filetitle{estimate}{Estimate a reduced-form VAR or BVAR}{VAR/estimate}

	\paragraph{Syntax}\label{syntax}

\begin{verbatim}
[V,VData,Fitted] = estimate(V,Inp,Range,...)
\end{verbatim}

\paragraph{Input arguments}\label{input-arguments}

\begin{itemize}
\item
  \texttt{V} {[} VAR {]} - Empty VAR object.
\item
  \texttt{Inp} {[} struct {]} - Input database.
\item
  \texttt{Range} {[} numeric {]} - Estimation range, including
  \texttt{P} pre-sample periods, where \texttt{P} is the order of the
  VAR.
\end{itemize}

\paragraph{Output arguments}\label{output-arguments}

\begin{itemize}
\item
  \texttt{V} {[} VAR {]} - Estimated reduced-form VAR object.
\item
  \texttt{VData} {[} struct {]} - Output database with the endogenous
  variables and the estimated residuals.
\item
  \texttt{Fitted} {[} numeric {]} - Periods in which fitted values have
  been calculated.
\end{itemize}

\paragraph{Options}\label{options}

\begin{itemize}
\item
  \texttt{'A='} {[} numeric \textbar{} \emph{empty} {]} - Restrictions
  on the individual values in the transition matrix, \texttt{A}.
\item
  \texttt{'BVAR='} {[} numeric {]} - Prior dummy observations for
  estimating a BVAR; construct the dummy observations using the one of
  the \texttt{BVAR} functions.
\item
  \texttt{'C='} {[} numeric \textbar{} \emph{empty} {]} - Restrictions
  on the individual values in the constant vector, \texttt{C}.
\item
  \texttt{'diff='} {[} \texttt{true} \textbar{} \emph{\texttt{false}}
  {]} - Difference the series before estimating the VAR; integrate the
  series back afterwards.
\item
  \texttt{'G='} {[} numeric \textbar{} \emph{empty} {]} - Restrictions
  on the individual values in the matrix at the co-integrating vector,
  \texttt{G}.
\item
  \texttt{'cointeg='} {[} numeric \textbar{} \emph{empty} {]} -
  Co-integrating vectors (in rows) that will be imposed on the estimated
  VAR.
\item
  \texttt{'comment='} {[} char \textbar{} \texttt{Inf} {]} - Assign
  comment to the estimated VAR object; \texttt{Inf} means the existing
  comment will be preserved.
\item
  \texttt{'constraints='} {[} char {]} - General linear constraints on
  the VAR parameters.
\item
  \texttt{'constant='} {[} \emph{\texttt{true}} \textbar{}
  \texttt{false} {]} - Include a constant vector in the VAR.
\item
  \texttt{'covParameters='} {[} \texttt{true} \textbar{}
  \emph{\texttt{false}} {]} - Calculate the covariance matrix of
  estimated parameters.
\item
  \texttt{'eqtnByEqtn='} {[} \texttt{true} \textbar{}
  \emph{\texttt{false}} {]} - Estimate the VAR equation by equation.
\item
  \texttt{'maxIter='} {[} numeric \textbar{} \emph{\texttt{1}} {]} -
  Maximum number of iterations when generalised least squares algorithm
  is involved.
\item
  \texttt{'mean='} {[} numeric \textbar{} \emph{empty} {]} - Impose a
  particular asymptotic mean on the VAR process.
\item
  \texttt{'order='} {[} numeric \textbar{} \emph{\texttt{1}} {]} - Order
  of the VAR.
\item
  \texttt{'progress='} {[} \texttt{true} \textbar{}
  \emph{\texttt{false}} {]} - Display progress bar in the command
  window.
\item
  \texttt{'schur='} {[} \emph{\texttt{true}} \textbar{} \texttt{false}
  {]} - Calculate triangular (Schur) representation of the estimated VAR
  straight away.
\item
  \texttt{'stdize='} {[} \texttt{true} \textbar{} \emph{\texttt{false}}
  {]} - Adjust the prior dummy observations by the std dev of the
  observations.
\item
  \texttt{'timeWeights=}' {[} tseries \textbar{} empty {]} - Time series
  of weights applied to individual periods in the estimation range.
\item
  \texttt{'tolerance='} {[} numeric \textbar{} \emph{\texttt{1e-5}} {]}
  - Convergence tolerance when generalised least squares algorithm is
  involved.
\item
  \texttt{'yNames='} {[} cellstr \textbar{} function\_handle \textbar{}
  \emph{\texttt{@(n) sprintf('y\%g',n)}} {]} - Use these names for the
  VAR variables.
\item
  \texttt{'eNames='} {[} cellstr \textbar{} function\_handle \textbar{}
  \emph{\texttt{@(yname,n) {[}'res\_',yname{]}}} {]} - Use these names
  for the VAR residuals.
\item
  \texttt{'warning='} {[} \emph{\texttt{true}} \textbar{} \texttt{false}
  {]} - Display warnings produced by this function.
\end{itemize}

\paragraph{Options for panel VAR}\label{options-for-panel-var}

\begin{itemize}
\item
  \texttt{'fixedEffect='} {[} \texttt{true} \textbar{}
  \emph{\texttt{false}} {]} - Include constant dummies for fixed effect
  in panel estimation; applies only if \texttt{'constant=' true}.
\item
  \texttt{'groupWeights='} {[} numeric \textbar{} \emph{empty} {]} - A
  1-by-NGrp vector of weights applied to groups in panel estimation,
  where NGrp is the number of groups; the weights will be rescaled so as
  to sum up to \texttt{1}.
\end{itemize}

\paragraph{Description}\label{description}

\subparagraph{Estimating a panel VAR}\label{estimating-a-panel-var}

Panel VAR objects are created by calling the function
\href{VAR/VAR}{\texttt{VAR}} with two input arguments: the list of
variables, and the list of group names. To estimate a panel VAR, the
input data, \texttt{Inp}, must be organised a super-database with
sub-databases for each group, and time series for each variables within
each group:

\begin{verbatim}
d.Group1_Name.Var1_Name
d.Group1_Name.Var2_Name
...
d.Group2_Name.Var1_Name
d.Group2_Name.Var2_Name
...
\end{verbatim}

\paragraph{Example}\label{example}


