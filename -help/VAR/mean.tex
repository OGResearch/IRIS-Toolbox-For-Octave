

    \filetitle{mean}{Mean of VAR process}{VAR/mean}

	\paragraph{Syntax}
 
 \begin{verbatim}
 M = mean(V)
 \end{verbatim}
 
 \paragraph{Input arguments}
 
 \begin{itemize}
 \item
   \texttt{V} {[} VAR {]} - VAR object.
 \end{itemize}
 
 \paragraph{Output arguments}
 
 \begin{itemize}
 \item
   \texttt{M} {[} numeric {]} - Asymptotic mean of the VAR variables.
 \end{itemize}
 
 \paragraph{Description}
 
 For plain VAR objects, the output argument \texttt{X} is a column vector
 where the k-th number is the asymptotic mean of the k-th variable, or
 \texttt{NaN} if the k-th variable is non-stationary (contains a unit
 root).
 
 In panel VAR objects (with a total of Ng groups) and/or VAR objects with
 multiple alternative parameterisations (with a total of Na
 parameterisations), \texttt{X} is an Ny-by-Ng-by-Na matrix in which the
 column \texttt{X(:,g,a)} is the asyptotic mean of the VAR variables in
 the g-th group and the a-th parameterisation.
 
 In VAR objects with exogenous inputs, the mean will be computed based on
 the asymptotic assumptions of exogenous inputs assigned by the function
 \href{VAR/xasymptote}{\texttt{xasymptote}}.
 
 \paragraph{Example}


