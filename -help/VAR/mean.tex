

    \filetitle{mean}{Mean of VAR process}{VAR/mean}

	\paragraph{Syntax}

\begin{verbatim}
X = mean(V)
\end{verbatim}

\paragraph{Input arguments}

\begin{itemize}
\itemsep1pt\parskip0pt\parsep0pt
\item
  \texttt{V} {[} VAR {]} - VAR object.
\end{itemize}

\paragraph{Output arguments}

\begin{itemize}
\itemsep1pt\parskip0pt\parsep0pt
\item
  \texttt{X} {[} numeric {]} - Asymptotic mean of the VAR variables.
\end{itemize}

\paragraph{Description}

For plain VAR objects, the output argument \texttt{X} is a column vector
where the k-th number is the asymptotic mean of the k-th variable, or
\texttt{NaN} if the k-th variable is non-stationary (contains a unit
root).

In panel VAR objects (with a total of Ng groups) and/or VAR objects with
multiple alternative parameterisations (with a total of Na
parameterisations), \texttt{X} is an Ny-by-Ng-by-Na matrix in which the
column \texttt{X(:,g,a)} is the asyptotic mean of the VAR variables in
the g-th group and the a-th parameterisation.

\paragraph{Example}


