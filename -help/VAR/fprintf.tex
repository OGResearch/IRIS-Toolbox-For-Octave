

    \filetitle{fprintf}{Write VAR model as formatted model code to text file}{VAR/fprintf}

	\paragraph{Syntax}\label{syntax}

\begin{verbatim}
[C,D] = fprintf(V,File,...)
\end{verbatim}

\paragraph{Input arguments}\label{input-arguments}

\begin{itemize}
\item
  \texttt{V} {[} VAR {]} - VAR object that will be printed to a model
  file.
\item
  \texttt{File} {[} char \textbar{} cellstr {]} - Filename, or filename
  format string, under which the model code will be saved.
\item
  Output arguments
\item
  \texttt{C} {[} cellstr {]} - Text string with the model code for each
  parameterisation.
\item
  \texttt{D} {[} cell {]} - Parameter databases for each
  parameterisation; if \texttt{'hardParameters='} true, the database
  will be empty.
\end{itemize}

\paragraph{Options}\label{options}

See help on \href{VAR/sprintf}{\texttt{sprintf}} for options available.

\paragraph{Description}\label{description}

For VAR objects with Na multiple alternative parameterisations, the
filename \texttt{File} must be either a 1-by-Na cell array of string
with a filename for each parameterisation, or a \texttt{sprintf} format
string where a single occurence of \texttt{'\%g'} will be replaced with
the parameterisation number.

\paragraph{Example}\label{example}


