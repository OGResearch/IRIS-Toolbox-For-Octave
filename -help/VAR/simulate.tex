

    \filetitle{simulate}{Simulate VAR model}{VAR/simulate}

	\paragraph{Syntax}
 
 \begin{verbatim}
 Outp = simulate(V,Inp,Range,...)
 \end{verbatim}
 
 \paragraph{Input arguments}
 
 \begin{itemize}
 \item
   \texttt{V} {[} VAR {]} - VAR object that will be simulated.
 \item
   \texttt{Inp} {[} tseries \textbar{} struct {]} - Input data from which
   the initial condtions and residuals will be taken.
 \item
   \texttt{Range} {[} numeric {]} - Simulation range; must not refer to
   \texttt{Inf}.
 \end{itemize}
 
 \paragraph{Output arguments}
 
 \begin{itemize}
 \item
   \texttt{Outp} {[} tseries {]} - Simulated output data.
 \end{itemize}
 
 \paragraph{Options}
 
 \begin{itemize}
 \item
   \texttt{'contributions='} {[} \texttt{true} \textbar{}
   \emph{\texttt{false}} {]} - Decompose the simulated paths into
   contributions of individual residuals.
 \item
   \texttt{'deviation='} {[} \texttt{true} \textbar{}
   \emph{\texttt{false}} {]} - Treat input and output data as deviations
   from unconditional mean.
 \item
   \texttt{'output='} {[} \emph{\texttt{'auto'}} \textbar{}
   \texttt{'dbase'} \textbar{} \texttt{'tseries'} {]} - Format of output
   data.
 \end{itemize}
 
 \paragraph{Description}
 
 \subparagraph{Backward simulation (backcast)}
 
 If the \texttt{Range} is a vector of decreasing dates, the simulation is
 performed backward. The VAR object is first converted to its backward
 representation using the function
 \href{VAR/backward}{\texttt{backward}}, and then the data are simulated
 from the latest date to the earliest date.
 
 \paragraph{Example}


