

    \filetitle{VAR}{Create new, empty reduced-form VAR object}{VAR/VAR}

	\paragraph{Syntax for plain VAR and VARX}
 
 \begin{verbatim}
 V = VAR(YNames)
 V = VAR(YNames,'exogenous=',XNames)
 \end{verbatim}
 
 \paragraph{Syntax for panel VAR and VARX}
 
 \begin{verbatim}
 V = VAR(YNames,'groups=',GroupNames)
 V = VAR(YNames,'exogenous=',XNames,'groups=',GroupNames)
 \end{verbatim}
 
 \paragraph{Output arguments}
 
 \begin{itemize}
 \item
   \texttt{V} {[} VAR {]} - New empty VAR object.
 \item
   \texttt{YNames} {[} cellstr \textbar{} char \textbar{}
   function\_handle {]} - Names of endogenous variables.
 \item
   \texttt{XNames} {[} cellstr \textbar{} char \textbar{}
   function\_handle {]} - Names of exogenous inputs.
 \item
   \texttt{GroupNames} {[} cellstr \textbar{} char \textbar{}
   function\_handle {]} - Names of groups for panel VAR estimation.
 \end{itemize}
 
 \paragraph{Options}
 
 \begin{itemize}
 \item
   \texttt{'exogenous='} {[} cellstr \textbar{} \emph{empty} {]} - Names
   of exogenous inputs; one of the names can be \texttt{!ttrend}, a
   linear time trend, which will be created automatically each time input
   data are required, and then included in the output database under the
   name \texttt{ttrend}.
 \item
   \texttt{'groups='} {[} cellstr \textbar{} \emph{empty} {]} - Names of
   groups for panel VAR estimation.
 \end{itemize}
 
 \paragraph{Description}
 
 This function creates a new empty VAR object. It is usually followed by
 an \href{VAR/estimate}{\texttt{estimate}} command to estimate the VAR
 parameters on the data.
 
 \paragraph{Example}
 
 To estimate a VAR, you first need to create an empty VAR object
 specifying the variable names, and then run the
 \href{VAR/estimate}{VAR/estimate} function on it, e.g.
 
 \begin{verbatim}
 v = VAR({'x','y','z'});
 [v,d] = estimate(v,d,range);
 \end{verbatim}
 
 where the input database \texttt{d} ought to contain time series
 \texttt{d.x}, \texttt{d.y}, \texttt{d.z}.


