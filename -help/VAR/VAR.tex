

    \filetitle{VAR}{Create new, empty reduced-form VAR object}{VAR/VAR}

	\paragraph{Syntax for plain VAR}\label{syntax-for-plain-var}

\begin{verbatim}
V = VAR(YNames)
\end{verbatim}

\paragraph{Syntax for panel VAR}\label{syntax-for-panel-var}

\begin{verbatim}
V = VAR(YNames,GroupNames)
\end{verbatim}

\paragraph{Output arguments}\label{output-arguments}

\begin{itemize}
\item
  \texttt{V} {[} VAR {]} - New empty VAR object.
\item
  \texttt{YNames} {[} cellstr \textbar{} char \textbar{}
  function\_handle {]} - Names of VAR variables.
\item
  \texttt{GroupNames} {[} cellstr \textbar{} char \textbar{}
  function\_handle {]} - Names of groups of data for panel estimation.
\end{itemize}

\paragraph{Description}\label{description}

This function creates a new empty VAR object. It is usually followed by
the \href{VAR/estimate}{\texttt{estimate}} function to estimate the VAR
parameters on data.

\paragraph{Example}\label{example}

To estimate a VAR, you first need to create an empty VAR object
specifying the variable names, and then run the \url{VAR/estimate}
function on it, e.g.

\begin{verbatim}
v = VAR({'x','y','z'});
[v,d] = estimate(v,d,range);
\end{verbatim}

where the input database \texttt{d} ought to contain time series
\texttt{d.x}, \texttt{d.y}, \texttt{d.z}.


