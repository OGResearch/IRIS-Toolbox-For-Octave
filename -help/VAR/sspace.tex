

    \filetitle{sspace}{Quasi-triangular state-space representation of VAR}{VAR/sspace}

	\paragraph{Syntax}\label{syntax}

\begin{verbatim}
[T,R,K,Z,H,D,Omg] = sspace(V,...)
\end{verbatim}

\paragraph{Input arguments}\label{input-arguments}

\begin{itemize}
\itemsep1pt\parskip0pt\parsep0pt
\item
  \texttt{V} {[} VAR {]} - VAR object.
\end{itemize}

\paragraph{Output arguments}\label{output-arguments}

\begin{itemize}
\item
  \texttt{T} {[} numeric {]} - Transition matrix.
\item
  \texttt{R} {[} numeric {]} - Matrix at the shock vector in transition
  equations.
\item
  \texttt{K} {[} numeric {]} - Constant vector in transition equations.
\item
  \texttt{Z} {[} numeric {]} - Matrix mapping transition variables to
  measurement variables.
\item
  \texttt{H} {[} numeric {]} - Matrix at the shock vector in measurement
  equations.
\item
  \texttt{D} {[} numeric {]} - Constant vector in measurement equations.
\item
  \texttt{U} {[} numeric {]} - Transformation matrix for predetermined
  variables.
\item
  \texttt{Omega} {[} numeric {]} - Covariance matrix of shocks.
\end{itemize}

\paragraph{Description}\label{description}

\paragraph{Example}\label{example}


