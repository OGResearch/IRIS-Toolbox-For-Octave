

    \filetitle{acf}{Autocovariance and autocorrelation functions for VAR variables}{VAR/acf}

	\paragraph{Syntax}
 
 \begin{verbatim}
 [C,R] = acf(V,...)
 \end{verbatim}
 
 \paragraph{Input arguments}
 
 \begin{itemize}
 \item
   \texttt{V} {[} VAR {]} - VAR object for which the ACF will be
   computed.
 \end{itemize}
 
 \paragraph{Output arguments}
 
 \begin{itemize}
 \item
   \texttt{C} {[} numeric {]} - Auto/cross-covariance matrices.
 \item
   \texttt{R} {[} numeric {]} - Auto/cross-correlation matrices.
 \end{itemize}
 
 \paragraph{Options}
 
 \begin{itemize}
 \item
   \texttt{'applyTo='} {[} logical \textbar{} \emph{\texttt{Inf}} {]} -
   Logical index of variables to which the \texttt{'filter='} will be
   applied; the default Inf means all variables.
 \item
   \texttt{'filter='} {[} char \textbar{} \emph{empty} {]} - Linear
   filter that is applied to variables specified by `applyto'.
 \item
   \texttt{'nfreq='} {[} numeric \textbar{} \emph{\texttt{256}} {]} -
   Number of equally spaced frequencies over which the `filter' is
   numerically integrated.
 \item
   \texttt{'order='} {[} numeric \textbar{} \emph{\texttt{0}} {]} - Order
   up to which ACF will be computed.
 \item
   \texttt{'progress='} {[} \texttt{true} \textbar{}
   \emph{\texttt{false}} {]} - Display progress bar in the command
   window.
 \end{itemize}
 
 \paragraph{Description}
 
 \paragraph{Example}


