

    \filetitle{forecast}{Unconditional or conditional VAR forecasts}{VAR/forecast}

	\paragraph{Syntax}\label{syntax}

\begin{verbatim}
Outp = forecast(V,Inp,Range,...)
Outp = forecast(V,Inp,Range,Cond,...)
\end{verbatim}

\paragraph{Input arguments}\label{input-arguments}

\begin{itemize}
\item
  \texttt{V} {[} VAR {]} - VAR object.
\item
  \texttt{Inp} {[} struct {]} - Input database from which initial
  condition will be read.
\item
  \texttt{Range} {[} numeric {]} - Forecast range; must not refer to
  \texttt{Inf}.
\item
  \texttt{Cond} {[} struct \textbar{} tseries {]} - Conditioning
  database with the mean values of residuals, reduced-form conditions on
  endogenous variables, and conditioning instruments.
\end{itemize}

\paragraph{Output arguments}\label{output-arguments}

\begin{itemize}
\itemsep1pt\parskip0pt\parsep0pt
\item
  \texttt{Outp} {[} struct {]} - Output database with forecasts of
  endogenous variables, residuals, and conditioning instruments.
\end{itemize}

\paragraph{Options}\label{options}

\begin{itemize}
\item
  \texttt{'cross='} {[} numeric \textbar{} \emph{\texttt{1}} {]} -
  Multiply the off-diagonal elements of the covariance matrix
  (cross-covariances) by this factor; \texttt{'cross='} must be equal to
  or smaller than \texttt{1}.
\item
  \texttt{'dbOverlay='} {[} \texttt{true} \textbar{}
  \emph{\texttt{false}} {]} - Combine the output data with the input
  data; works only if the input data is a database.
\item
  \texttt{'deviation='} {[} \texttt{true} \textbar{}
  \emph{\texttt{false}} {]} - Both input and output data are deviations
  from the unconditional mean.
\item
  \texttt{'meanOnly='} {[} \texttt{true} \textbar{}
  \emph{\texttt{false}} {]} - Return a plain database with mean
  forecasts only.
\item
  \texttt{'omega='} {[} numeric \textbar{} \emph{empty} {]} - Modify the
  covariance matrix of residuals for this forecast.
\end{itemize}

\paragraph{Description}\label{description}

\paragraph{Example}\label{example}


