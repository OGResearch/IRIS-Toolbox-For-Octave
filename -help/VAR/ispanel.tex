

    \filetitle{ispanel}{True for panel VAR based objects}{VAR/ispanel}

	\paragraph{Syntax}
 
 \begin{verbatim}
 Flag = ispanel(X)
 \end{verbatim}
 
 \paragraph{Input arguments}
 
 \begin{itemize}
 \item
   \texttt{X} {[} VAR \textbar{} SVAR \textbar{} FAVAR {]} - VAR based
   object.
 \end{itemize}
 
 \paragraph{Output arguments}
 
 \begin{itemize}
 \item
   \texttt{Flag} {[} \texttt{true} \textbar{} \texttt{false} {]} - True
   if the VAR based object, \texttt{X}, is based on a panel of data.
 \end{itemize}
 
 \paragraph{Description}
 
 Plain, i.e.~non-panel, VAR based objects are created by calling the
 constructor with one input argument: the list of variables. Panel VAR
 based objects are created by calling the constructor with two input
 arguments: the list of variables, and the names of groups of data.
 
 \paragraph{Example}


