

    \filetitle{assign}{Manually assign system matrices to VAR object}{VAR/assign}

	\paragraph{Syntax}

\begin{verbatim}
V = assign(V,A,K,Omg)
V = assign(V,A,[],Omg)
V = assign(V,A,K,Omg,Dates)
V = assign(V,A,[],Omg,Dates)
\end{verbatim}

\paragraph{Input arguments}

\begin{itemize}
\item
  \texttt{V} {[} VAR {]} - VAR object with variable names.
\item
  \texttt{A} {[} numeric {]} - Transition matrices; see Description.
\item
  \texttt{K} {[} numeric \textbar{} empty {]} - Constant vector or
  matrix; if omitted, the constant vector will be set to zeros, and will
  not be included in the number of free parameters.
\item
  \texttt{Omg} {[} numeric {]} - Covariance matrix of forecast errors
  (reduced-form residuals).
\item
  \texttt{Dates} {[} numeric {]} - Vector of dates of (hypothetical)
  fitted observations; may be omitted.
\end{itemize}

\paragraph{Output arguments}

\begin{itemize}
\itemsep1pt\parskip0pt\parsep0pt
\item
  \texttt{V} {[} VAR {]} - VAR object with system matrices assigned.
\end{itemize}

\paragraph{Description}

To assign matrices for a p-th order VAR, stack the transition matrices
for individual lags horizontally,

\begin{verbatim}
A = [A1,...,Ap]
\end{verbatim}

where \texttt{A1} is the coefficient matrix on the first lag, and
\texttt{Ap} is the coefficient matrix on the last, p-th, lag.

\paragraph{Example}


