

    \filetitle{infocrit}{Populate information criteria for a parameterised VAR}{VAR/infocrit}

	\paragraph{Syntax}
 
 \begin{verbatim}
 V = infocrit(V)
 \end{verbatim}
 
 \paragraph{Input arguments}
 
 \begin{itemize}
 \item
   \texttt{V} {[} VAR {]} - VAR object.
 \end{itemize}
 
 \paragraph{Output arguments}
 
 \begin{itemize}
 \item
   \texttt{V} {[} VAR {]} - VAR object with the AIC and SBC information
   criteria re-calculated.
 \end{itemize}
 
 \paragraph{Description}
 
 In most cases, you don't have to run the function \texttt{infocrit} as
 it is called from within \texttt{estimate} immediately after a new
 parameterisation is created.
 
 \paragraph{Example}


