

    \filetitle{get}{Query tseries object property}{tseries/get}

	\paragraph{Syntax}
 
 \begin{verbatim}
 Value = get(x,Query)
 [Value,Value,...] = get(x,Query,Query,...)
 \end{verbatim}
 
 \paragraph{Input arguments}
 
 \begin{itemize}
 \item
   \texttt{x} {[} model {]} - Tseries object.
 \item
   \texttt{Query} {[} char {]} - Name of the queried property.
 \end{itemize}
 
 \paragraph{Output arguments}
 
 \begin{itemize}
 \item
   \texttt{Value} {[} \ldots{} {]} - Value of the queried property.
 \end{itemize}
 
 \paragraph{Valid queries on tseries objects}
 
 \begin{itemize}
 \item
   \texttt{'end='} Returns {[} numeric {]} the date of the last
   observation.
 \item
   \texttt{'freq='} Returns {[} numeric {]} the frequency (periodicity)
   of the time series.
 \item
   \texttt{'nanEnd='} Returns {[} numeric {]} the last date at which
   observations are available in all columns; for scalar tseries, this
   query always returns the same as \texttt{'end'}.
 \item
   \texttt{'nanRange='} Returns {[} numeric {]} the date range from
   \texttt{'nanstart'} to \texttt{'nanend'}; for scalar time series, this
   query always returns the same as \texttt{'range'}.
 \item
   \texttt{'nanStart='} Returns {[} numeric {]} the first date at which
   observations are available in all columns; for scalar tseries, this
   query always returns the same as \texttt{'start'}.
 \item
   \texttt{'range='} Returns {[} numeric {]} the date range from the
   first observation to the last observation.
 \item
   \texttt{'start='} Returns {[} numeric {]} the date of the first
   observation.
 \end{itemize}
 
 \paragraph{Description}


