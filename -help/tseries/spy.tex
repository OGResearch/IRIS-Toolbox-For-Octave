

    \filetitle{spy}{Visualise tseries observations that pass a test}{tseries/spy}

	\paragraph{Syntax}\label{syntax}

\begin{verbatim}
[AA,LL] = spy(X,...)
[AA,LL] = spy(RANGE,X,...)
\end{verbatim}

\paragraph{Input arguments}\label{input-arguments}

\begin{itemize}
\item
  \texttt{X} {[} tseries {]} - Input tseries object whose non-NaN
  observations will be plotted as markers.
\item
  \texttt{RANGE} {[} tseries {]} - Date range on which the tseries
  observations will be visualised; if not specified the entire available
  range will be used.
\end{itemize}

\paragraph{Output arguments}\label{output-arguments}

\begin{itemize}
\item
  \texttt{AA} {[} tseries {]} - Handle to the axes created.
\item
  \texttt{LL} {[} tseries {]} - Handle to the marks plotted.
\end{itemize}

\paragraph{Options}\label{options}

\begin{itemize}
\item
  \texttt{'dateformat='} {[} char \textbar{}
  \emph{irisget(`plotdateformat')} {]} - Date format for the tick marks
  on the x-axis.
\item
  \texttt{'datetick='} {[} numeric \textbar{} \emph{\texttt{Inf}} {]} -
  Vector of dates locating tick marks on the x-axis; Inf means they will
  be created automatically.
\item
  \texttt{'names='} {[} cellstr {]} - Names that will be used to
  annotate individual columns of the input tseries object.
\item
  \texttt{'test='} {[} function\_handle \textbar{}
  \emph{@(x)\textasciitilde{}isnan(x)} {]} - Test applied to each
  observations; only the values returning a true will be displayed.
\end{itemize}

\paragraph{Description}\label{description}

\paragraph{Example}\label{example}


