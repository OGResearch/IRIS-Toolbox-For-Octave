

    \filetitle{barcon}{Contribution bar graph for tseries objects}{tseries/barcon}

	\paragraph{Syntax}
 
 \begin{verbatim}
 [H,Range] = barcon(X,...)
 [H,Range] = barcon(Range,X,...)
 [H,Range] = barcon(Ax,Range,X,...)
 \end{verbatim}
 
 \paragraph{Input arguments}
 
 \begin{itemize}
 \item
   \texttt{Ax} {[} numeric {]} - Handle to axes in which the graph will
   be plotted; if not specified, the current axes will used.
 \item
   \texttt{Range} {[} numeric {]} - Date range; if not specified the
   entire range of the input tseries object will be plotted.
 \item
   \texttt{X} {[} tseries {]} - Input tseries object whose columns will
   be ploted as a contribution bar graph.
 \end{itemize}
 
 \paragraph{Output arguments}
 
 \begin{itemize}
 \item
   \texttt{H} {[} numeric {]} - Handle(s) to the bars plotted.
 \item
   \texttt{Range} {[} numeric {]} - Actually plotted date range.
 \end{itemize}
 
 \paragraph{Options}
 
 \begin{itemize}
 \item
   \texttt{'barWidth='} {[} numeric \textbar{} \emph{0.8} {]} - Width of
   bars as a percentage of the space each period occupies on the x-axis.
 \item
   \texttt{'colorMap='} {[} numeric \textbar{}
   \emph{get(gcf(),`colorMap')} {]} - Color map used to fill the
   contribution bars.
 \item
   \texttt{'dateFormat='} {[} char \textbar{}
   \emph{irisget(`plotdateformat')} {]} - Date format for the tick marks
   on the x-axis.
 \item
   \texttt{'dateTick='} {[} numeric \textbar{} \emph{\texttt{Inf}} {]} -
   Vector of dates locating tick marks on the x-axis; Inf means they will
   be created automatically.
 \item
   \texttt{'evenlySpread='} {[} \emph{\texttt{true}} \textbar{}
   \texttt{false} {]} - Colors picked for the contribution bars are
   evenly spread across the color map.
 \item
   \texttt{'ordering='} {[} `ascend' \textbar{} `descend' \textbar{}
   \emph{`preserve'} \textbar{} numeric {]} - Ordering of contributions
   with the same sign withinin each period; \texttt{'preserve'} means the
   original order will be preserved.
 \end{itemize}
 
 \paragraph{Description}
 
 \paragraph{Example}


