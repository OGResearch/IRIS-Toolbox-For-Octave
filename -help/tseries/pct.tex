

    \filetitle{pct}{Percent rate of change}{tseries/pct}

	\paragraph{Syntax}
 
 \begin{verbatim}
 X = pct(X)
 X = pct(X,K,...)
 \end{verbatim}
 
 \paragraph{Input arguments}
 
 \begin{itemize}
 \item
   \texttt{X} {[} tseries {]} - Input tseries object.
 \item
   \texttt{K} {[} numeric {]} - Time shift over which the rate of change
   will be computed, i.e.~between time t and t+k; if not specified
   \texttt{K} will be set to \texttt{-1}.
 \end{itemize}
 
 \paragraph{Output arguments}
 
 \begin{itemize}
 \item
   \texttt{X} {[} tseries {]} - Percentage rate of change in the input
   data.
 \end{itemize}
 
 \paragraph{Options}
 
 \begin{itemize}
 \item
   \texttt{'outputFreq='} {[} \texttt{1} \textbar{} \texttt{2} \textbar{}
   \texttt{4} \textbar{} \texttt{6} \textbar{} \texttt{12} \textbar{}
   \emph{\texttt{Inf}} {]} - Convert the rate of change to the requested
   date frequency; \texttt{Inf} means plain rate of change with no
   conversion.
 \end{itemize}
 
 \paragraph{Description}
 
 \paragraph{Example}
 
 In this example, \texttt{x} is a monthly time series. The following
 command computes the annualised rate of change between month t and t-1:
 
 \begin{verbatim}
 pct(x,-1,'outputfreq=',1)
 \end{verbatim}
 
 while the following line computes the annualised rate of change between
 month t and t-3:
 
 \begin{verbatim}
 pct(x,-3,'outputFreq=',1)
 \end{verbatim}


