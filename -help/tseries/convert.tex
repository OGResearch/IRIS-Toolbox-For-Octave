

    \filetitle{convert}{Convert tseries object to a different frequency}{tseries/convert}

	\paragraph{Syntax}
 
 \begin{verbatim}
 Y = convert(X,NewFreq)
 Y = convert(X,NewFreq,Range,...)
 \end{verbatim}
 
 \paragraph{Input arguments}
 
 \begin{itemize}
 \item
   \texttt{X} {[} tseries {]} - Input tseries object that will be
   converted to a new frequency, \texttt{freq}, aggregating or
   intrapolating the data.
 \item
   \texttt{NewFreq} {[} numeric \textbar{} char {]} - New frequency to
   which the input data will be converted: \texttt{1} or \texttt{'A'} for
   annual, \texttt{2} or \texttt{'H'} for half-yearly, \texttt{4} or
   \texttt{'Q'} for quarterly, \texttt{6} or \texttt{'B'} for bi-monthly,
   and \texttt{12} or \texttt{'M'} for monthly.
 \item
   \texttt{Range} {[} numeric {]} - Date range on which the input data
   will be converted.
 \end{itemize}
 
 \paragraph{Output arguments}
 
 \begin{itemize}
 \item
   \texttt{Y} {[} tseries {]} - Output tseries created by converting
   \texttt{X} to the new frequency.
 \end{itemize}
 
 \paragraph{Options}
 
 \begin{itemize}
 \item
   \texttt{'ignoreNaN='} {[} \texttt{true} \textbar{}
   \emph{\texttt{false}} {]} - Exclude NaNs from agreggation.
 \item
   \texttt{'missing='} {[} numeric \textbar{} \emph{\texttt{NaN}}
   \textbar{} \texttt{'last'} {]} - Replace missing observations with
   this value.
 \end{itemize}
 
 \paragraph{Options for high- to low-frequency conversion (aggregation)}
 
 \begin{itemize}
 \item
   \texttt{'method='} {[} function\_handle \textbar{} \texttt{'first'}
   \textbar{} \texttt{'last'} \textbar{} \emph{\texttt{@mean}} {]} -
   Method that will be used to aggregate the high frequency data.
 \item
   \texttt{'select='} {[} numeric \textbar{} \emph{\texttt{Inf}} {]} -
   Select only these high-frequency observations within each
   low-frequency period; Inf means all observations will be used.
 \end{itemize}
 
 \paragraph{Options for low- to high-frequency conversion
 (interpolation)}
 
 \begin{itemize}
 \item
   \texttt{'method='} {[} char \textbar{} \emph{\texttt{'cubic'}}
   \textbar{} \texttt{'quadsum'} \textbar{} \texttt{'quadavg'} {]} -
   Interpolation method; any option available in the built-in
   \texttt{interp1} function can be used.
 \item
   \texttt{'position='} {[} \emph{\texttt{'centre'}} \textbar{}
   \texttt{'start'} \textbar{} \texttt{'end'} {]} - Position of the
   low-frequency date grid.
 \end{itemize}
 
 \paragraph{Description}
 
 The function handle that you pass in through the `method' option when
 you aggregate the data (convert higher frequency to lower frequency)
 should behave like the built-in functions \texttt{mean}, \texttt{sum}
 etc. In other words, it is expected to accept two input arguments:
 
 \begin{itemize}
 \item
   the data to be aggregated,
 \item
   the dimension along which the aggregation is calculated.
 \end{itemize}
 
 The function will be called with the second input argument set to 1, as
 the data are processed en block columnwise. If this call fails,
 \texttt{convert} will attempt to call the function with just one input
 argument, the data, but this is not a safe option under some
 circumstances since dimension mismatch may occur.
 
 \paragraph{Example}


