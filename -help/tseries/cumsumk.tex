

    \filetitle{cumsumk}{Cumulative sum with a k-period leap}{tseries/cumsumk}

	\paragraph{Syntax}\label{syntax}

\begin{verbatim}
Y = cumsumk(X,K,Rho,Range)
Y = cumsumk(X,K,Rho)
Y = cumsumk(X,K)
Y = cumsumk(X)
\end{verbatim}

\paragraph{Input arguments}\label{input-arguments}

\begin{itemize}
\item
  \texttt{X} {[} tseries {]} - Input data.
\item
  \texttt{K} {[} numeric {]} - Number of periods that will be leapt the
  cumulative sum will be taken; if not specified, \texttt{K} is chosen
  to match the frequency of the input data (e.g. \texttt{K = -4} for
  quarterly data), or \texttt{K = -1} for indeterminate frequency.
\item
  \texttt{Rho} {[} numeric {]} - Autoregressive coefficient; if not
  specified, \texttt{Rho = 1}.
\item
  \texttt{Range} {[} numeric {]} - Range on which the cumulative sum
  will be computed and the output series returned.
\end{itemize}

\paragraph{Output arguments}\label{output-arguments}

\begin{itemize}
\itemsep1pt\parskip0pt\parsep0pt
\item
  \texttt{Y} {[} tseries {]} - Output data constructed as described
  below.
\end{itemize}

\paragraph{Options}\label{options}

\begin{itemize}
\itemsep1pt\parskip0pt\parsep0pt
\item
  \texttt{'log='} {[} \texttt{true} \textbar{} \emph{\texttt{false}} {]}
  - Logarithmise the input data before, and de-logarithmise the output
  data back after, running \texttt{x12}.
\end{itemize}

\paragraph{Description}\label{description}

If \texttt{K \textless{} 0}, the first \texttt{K} observations in the
output series \texttt{Y} are copied from \texttt{X}, and the new
observations are given recursively by

\begin{verbatim}
Y{t} = Rho*Y{t-K} + X{t}.
\end{verbatim}

If \texttt{K \textgreater{} 0}, the last \texttt{K} observations in the
output series \texttt{Y} are copied from \texttt{X}, and the new
observations are given recursively by

\begin{verbatim}
Y{t} = Rho*Y{t+K} + X{t},
\end{verbatim}

going backwards in time.

If \texttt{K == 0}, the input data are returned.

\paragraph{Example}\label{example}

Construct random data with seasonal pattern, and run X12 to seasonally
adjust these series.

\begin{verbatim}
x = tseries(qq(1990,1):qq(2020,4),@randn);
x1 = cumsumk(x,-4,1);
x2 = cumsumk(x,-4,0.7);
x1sa = x12(x1);
x2sa = x12(x2);
\end{verbatim}

The new series \texttt{x1} will be a unit-root process while \texttt{x2}
will be stationary. Note that the command on the second line could be
replaced with \texttt{x1 = cumsumk(x)}.


