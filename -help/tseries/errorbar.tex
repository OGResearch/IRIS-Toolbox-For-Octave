

    \filetitle{errorbar}{Line plot with error bars}{tseries/errorbar}

	\paragraph{Syntax}\label{syntax}

\begin{verbatim}
[LL,EE,Range] = errorbar(X,B,...)
[LL,EE,Range] = errorbar(Range,X,B,...)
[LL,EE,Range] = errorbar(AA,Range,X,B,...)
[LL,EE,Range] = errorbar(X,Lo,Hi,...)
[LL,EE,Range] = errorbar(Range,X,Lo,Hi,...)
[LL,EE,Range] = errorbar(AA,Range,X,Lo,Hi,...)
\end{verbatim}

\paragraph{Input arguments}\label{input-arguments}

\begin{itemize}
\item
  \texttt{AA} {[} numeric {]} - Handle to axes in which the graph will
  be plotted; if not specified, the current axes will used.
\item
  \texttt{Range} {[} numeric {]} - Date range; if not specified the
  entire range of the input tseries object will be plotted.
\item
  \texttt{X} {[} tseries {]} - Tseries object whose data will be plotted
  as a line graph.
\item
  \texttt{B} {[} tseries {]} - Width of the bands that will be plotted
  around the lines.
\item
  \texttt{Lo} {[} tseries {]} - Width of the band below the line.
\item
  \texttt{Hi} {[} tseries {]} - Width of the band above the line.
\end{itemize}

\paragraph{Output arguments}\label{output-arguments}

\begin{itemize}
\item
  \texttt{LL} {[} numeric {]} - Handles to lines plotted.
\item
  \texttt{EE} {[} numeric {]} - Handles to error bars plotted.
\item
  \texttt{Range} {[} numeric {]} - Actually plotted date range.
\end{itemize}

\paragraph{Options}\label{options}

\begin{itemize}
\itemsep1pt\parskip0pt\parsep0pt
\item
  \texttt{'relative='} {[} \emph{\texttt{true}} \textbar{}
  \texttt{false} {]} - If \texttt{true}, the data for the lower and
  upper bounds are relative to the centre, i.e.~the bounds will be added
  to the centre (in this case, \texttt{Lo} must be negative numbers and
  \texttt{Hi} must be positive numbers). If \texttt{false}, the bounds
  are absolute data (in this case \texttt{Lo} must be lower than
  \texttt{X}, and \texttt{Hi} must be higher than \texttt{X}).
\end{itemize}

See help on \href{tseries/plot}{\texttt{tseries/plot}}.


