

    \filetitle{tseries}{Create new time series (tseries) object}{tseries/tseries}

	\paragraph{Syntax}\label{syntax}

\begin{verbatim}
X = tseries()
X = tseries(DATES,VALUES)
X = tseries(DATES,VALUES,COMMENTS)
\end{verbatim}

\paragraph{Input arguments}\label{input-arguments}

\begin{itemize}
\item
  \texttt{DATES} {[} numeric {]} - Dates for which observations will be
  supplied; \texttt{dates} do not have to be sorted in ascending order.
  If \texttt{dates} is scalar and \texttt{values} have multiple rows,
  then the date in \texttt{dates} is interpreted as a startdate for the
  time series.
\item
  \texttt{VALUES} {[} numeric \textbar{} function\_handle {]} -
  Numerical values (observations) arranged columnwise, or a function
  that will be used to create an N-by-1 array of values, where N is the
  number of \texttt{dates}.
\item
  \texttt{COMMENTS} {[} char \textbar{} cellstr {]} - Comment or
  comments attached to each column of observations.
\end{itemize}

\paragraph{Output arguments}\label{output-arguments}

\begin{itemize}
\itemsep1pt\parskip0pt\parsep0pt
\item
  \texttt{X} {[} tseries {]} - New tseries object.
\end{itemize}

\paragraph{Description}\label{description}

\paragraph{Example}\label{example}


