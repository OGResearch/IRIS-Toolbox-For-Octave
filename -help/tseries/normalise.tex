

    \filetitle{normalise}{Normalise (or rebase) data to particular date}{tseries/normalise}

	\paragraph{Syntax}
 
 \begin{verbatim}
 x = normalise(x)
 x = normalise(x,normdate,...)
 \end{verbatim}
 
 \paragraph{Input arguments}
 
 \begin{itemize}
 \item
   \texttt{x} {[} tseries {]} - Input tseries object that will be
   normalised.
 \item
   \texttt{normdate} {[} numeric \textbar{} `start' \textbar{} `end'
   \textbar{} `nanstart' \textbar{} `nanend' {]} - Date relative to which
   the input data will be normalised; if not specified, `nanstart' (the
   first date for which all columns have an observation) will be used.
 \end{itemize}
 
 \paragraph{Output arguments}
 
 \begin{itemize}
 \item
   \texttt{x} {[} tseries {]} - Normalised tseries object.
 \end{itemize}
 
 \paragraph{Options}
 
 \begin{itemize}
 \item
   \texttt{'mode='} {[} `add' \textbar{} \emph{`mult'} {]} - Additive or
   multiplicative normalisation. Description ============
 \end{itemize}
 
 \paragraph{Example}


