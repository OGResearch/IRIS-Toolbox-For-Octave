

    \filetitle{normalise}{Normalise (or rebase) data to particular date}{tseries/normalise}

	\paragraph{Syntax}
 
 \begin{verbatim}
 X = normalise(X)
 X = normalise(X,NormDate,...)
 \end{verbatim}
 
 \paragraph{Input arguments}
 
 \begin{itemize}
 \item
   \texttt{x} {[} tseries {]} - Input time series that will be
   normalised.
 \item
   \texttt{NormDate} {[} numeric \textbar{} \texttt{'start'} \textbar{}
   \texttt{'end'} \textbar{} \texttt{'nanStart'} \textbar{}
   \texttt{'nanEnd'} {]} - Date relative to which the input data will be
   normalised; if not specified, \texttt{'nanStart'} (the first date for
   which all columns have an observation) will be used.
 \end{itemize}
 
 \paragraph{Output arguments}
 
 \begin{itemize}
 \item
   \texttt{X} {[} tseries {]} - Normalised time series.
 \end{itemize}
 
 \paragraph{Options}
 
 \begin{itemize}
 \item
   \texttt{'mode='} {[} \texttt{'add'} \textbar{} \emph{\texttt{'mult'}}
   {]} - Additive or multiplicative normalisation.
 \end{itemize}
 
 \paragraph{Description}
 
 \paragraph{Example}


