

    \filetitle{startdate}{Date of the first available observation in a tseries object}{tseries/startdate}

	\paragraph{Syntax}\label{syntax}

\begin{verbatim}
D = startdate(X)
\end{verbatim}

\paragraph{Input arguments}\label{input-arguments}

\begin{itemize}
\itemsep1pt\parskip0pt\parsep0pt
\item
  \texttt{X} {[} tseries {]} - Tseries object.
\end{itemize}

\paragraph{Output arguments}\label{output-arguments}

\begin{itemize}
\itemsep1pt\parskip0pt\parsep0pt
\item
  \texttt{D} {[} numeric {]} - IRIS serial date number representing the
  date of the first observation available in the input tseries.
\end{itemize}

\paragraph{Description}\label{description}

The \texttt{startdate} function is equivalent to calling

\begin{verbatim}
get(X,'startDate')
\end{verbatim}

\paragraph{Example}\label{example}


