

    \filetitle{plotyy}{Line plot function with LHS and RHS axes for time series}{tseries/plotyy}

	\paragraph{Syntax}\label{syntax}

\begin{verbatim}
[Ax,Lhs,Rhs,Range] = plotyy(X,Y,...)
[Ax,Lhs,Rhs,Range] = plotyy(Range,X,Y,...)
\end{verbatim}

\paragraph{Input arguments}\label{input-arguments}

\begin{itemize}
\item
  \texttt{Range} {[} numeric {]} - Date range; if not specified the
  entire range of the input tseries object will be plotted.
\item
  \texttt{X} {[} tseries {]} - Input tseries object whose columns will
  be plotted and labelled on the LHS.
\item
  \texttt{Y} {[} tseries {]} - Input tseries object whose columns will
  be plotted and labelled on the RHS.
\end{itemize}

\paragraph{Output arguments}\label{output-arguments}

\begin{itemize}
\item
  \texttt{Ax} {[} numeric {]} - Handles to the LHS and RHS axes.
\item
  \texttt{Lhs} {[} numeric {]} - Handles to series plotted on the LHS
  axis.
\item
  \texttt{Rhs} {[} numeric {]} - Handles to series plotted on the RHS
  axis.
\item
  \texttt{Range} {[} numeric {]} - Actually plotted date range.
\end{itemize}

\paragraph{Options}\label{options}

\begin{itemize}
\item
  \texttt{'conincident='} {[} \texttt{true} \textbar{}
  \emph{\texttt{false}} {]} - Make the LHS and RHS y-axis grids
  coincident.
\item
  \texttt{'dateFormat='} {[} char \textbar{}
  \emph{irisget(`plotdateformat')} {]} - Date format for the tick marks
  on the x-axis.
\item
  \texttt{'dateTick='} {[} numeric \textbar{} \emph{\texttt{Inf}} {]} -
  Vector of dates locating tick marks on the x-axis; Inf means they will
  be created automatically.
\item
  \texttt{'freqLetters='} {[} char \textbar{} \emph{`YHQBM'} {]} - Five
  letters to represent the five date frequencies (yearly, half-yearly,
  quarterly, bi-monthly, and monthly).
\item
  \texttt{'lhsPlotFunc='} {[} @area \textbar{} @bar \textbar{} *@plot*
  \textbar{} @stem {]} - Function that will be used to plot the LHS
  data.
\item
  \texttt{'lhsTight='} {[} \texttt{true} \textbar{}
  \emph{\texttt{false}} {]} - Make the LHS y-axis tight.
\item
  \texttt{'rhsPlotFunc='} {[} @area \textbar{} @bar \textbar{} *@plot*
  \textbar{} @stem {]} - Function that will be used to plot the RHS
  data.
\item
  \texttt{'rhsTight='} {[} \texttt{true} \textbar{}
  \emph{\texttt{false}} {]} - Make the RHS y-axis tight.
\end{itemize}

\paragraph{Description}\label{description}

\paragraph{Example}\label{example}


