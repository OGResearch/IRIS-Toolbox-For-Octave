

    \filetitle{evalr}{Evaluate tseries expression recursively}{tseries/evalr}

	\paragraph{Syntax}\label{syntax}

\begin{verbatim}
evalr(Expr,Range)
\end{verbatim}

\paragraph{Input arguments}\label{input-arguments}

\begin{itemize}
\item
  \texttt{Eqtn} {[} char {]} - Equation that will be evaluated
  recursively ; the expression must be \texttt{'LHS\_Name = RHS\_Expr'}
  where \texttt{LHS\_Name} is the name of the LHS variable (which must
  be a tseries object), and \texttt{RHS\_Expr} is the RHS expression
  that will be evaluated for each period in \texttt{Range}, and whose
  result will be immediately assigned to the variables named
  \texttt{LHS\_Name}.
\item
  \texttt{Range} {[} numeric {]} - Date range or vector of dates over
  which the expression will be evaluated.
\end{itemize}

\paragraph{Description}\label{description}

\paragraph{Example}\label{example}

The following commands create an autoregressive process with random
shocks:

\begin{verbatim}
x = tseries(qq(2000,1):qq(2015,4),0);
e = tseries(qq(2000,1):qq(2015,4),@randn);
evalr('x = 0.8*x{-1} + e',qq(2000,2):qq(2015,4));
\end{verbatim}


