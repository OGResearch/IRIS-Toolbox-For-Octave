

    \filetitle{bsxfunc}{Standard BSXFUN implemented for tseries objects}{tseries/bsxfun}

	\paragraph{Syntax}\label{syntax}

\begin{verbatim}
Z = bsxfun(Func,X,Y)
\end{verbatim}

\paragraph{Input arguments}\label{input-arguments}

\begin{itemize}
\item
  \texttt{Func} {[} function\_handle {]} - Function that will be applied
  to the input series, \texttt{FUN(X,Y)}.
\item
  \texttt{X} {[} tseries \textbar{} numeric {]} - Input time series or
  numeric array.
\item
  \texttt{Y} {[} tseries \textbar{} numeric {]} - Input time series or
  numeric array.
\end{itemize}

\paragraph{Output arguments}\label{output-arguments}

\begin{itemize}
\itemsep1pt\parskip0pt\parsep0pt
\item
  \texttt{Z} {[} tseries {]} - Result of \texttt{Func(X,Y)} with
  \texttt{X} and/or \texttt{Y} expanded properly in singleton
  dimensions.
\end{itemize}

\paragraph{Description}\label{description}

See help on built-in \texttt{bsxfun} for more help.

\paragraph{Example}\label{example}

We create a multivariate time series and subtract mean from its
individual columns.

\begin{verbatim}
x = tseries(1:10,rand(10,4));
xx = bsxfun(@minus,x,mean(x));
\end{verbatim}


