

    \filetitle{plot}{Line graph for tseries objects}{tseries/plot}

	\paragraph{Syntax}
 
 \begin{verbatim}
 [h,range] = plot(x,...)
 [h,range] = plot(range,x,...)
 [h,range] = plot(a,range,x,...)
 \end{verbatim}
 
 \paragraph{Input arguments}
 
 \begin{itemize}
 \item
   \texttt{a} {[} numeric {]} - Handle to axes in which the graph will be
   plotted; if not specified, the current axes will used.
 \item
   \texttt{range} {[} numeric {]} - Date range; if not specified the
   entire range of the input tseries object will be plotted.
 \item
   \texttt{x} {[} tseries {]} - Input tseries object whose columns will
   be ploted as a line graph.
 \end{itemize}
 
 \paragraph{Output arguments}
 
 \begin{itemize}
 \item
   \texttt{h} {[} numeric {]} - Handles to the lines plotted.
 \item
   \texttt{range} {[} numeric {]} - Actually plotted date range.
 \end{itemize}
 
 \paragraph{Options}
 
 \begin{itemize}
 \item
   \texttt{'dateFormat='} {[} char \textbar{}
   \emph{irisget(`plotdateformat')} {]} - Date format for the tick marks
   on the x-axis.
 \item
   \texttt{'datePosition='} {[} \emph{`centre'} \textbar{} `end'
   \textbar{} `start' {]} - Position of each date point within a given
   period span.
 \item
   \texttt{'datetick='} {[} numeric \textbar{} \emph{\texttt{Inf}} {]} -
   Vector of dates locating tick marks on the x-axis; Inf means they will
   be created automatically.
 \item
   \texttt{'tight='} {[} \texttt{true} \textbar{} \emph{\texttt{false}}
   {]} - Make the y-axis tight.
 \end{itemize}
 
 See help on built-in \texttt{plot} function for other options available.
 
 \paragraph{Description}
 
 \paragraph{Example}


