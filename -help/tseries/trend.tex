

    \filetitle{trend}{Estimate a time trend}{tseries/trend}

	\paragraph{Syntax}
 
 \begin{verbatim}
 X = trend(X,range)
 \end{verbatim}
 
 \paragraph{Input arguments}
 
 \begin{itemize}
 \item
   \texttt{X} {[} tseries {]} - Input time series.
 \item
   \texttt{Range} {[} tseries {]} - Range for which the trend will be
   computed.
 \end{itemize}
 
 \paragraph{Output arguments}
 
 \begin{itemize}
 \item
   \texttt{X} {[} tseries {]} - Output trend time series.
 \end{itemize}
 
 \paragraph{Options}
 
 \begin{itemize}
 \item
   \texttt{'break='} {[} numeric \textbar{} \emph{empty} {]} - Vector of
   breaking points at which the trend may change its slope.
 \item
   \texttt{'connect='} {[} \emph{\texttt{true}} \textbar{} \texttt{false}
   {]} - Calculate the trend by connecting the first and the last
   observations.
 \item
   \texttt{'diff='} {[} \texttt{true} \textbar{} \emph{\texttt{false}}
   {]} - Estimate the trend on differenced data.
 \item
   \texttt{'log='} {[} \texttt{true} \textbar{} \emph{\texttt{false}} {]}
   - Logarithmise the input data, de-logarithmise the output data.
 \item
   \texttt{'season='} {[} \texttt{true} \textbar{} \emph{\texttt{false}}
   \textbar{} \texttt{2} \textbar{} \texttt{4} \textbar{} \texttt{6}
   \textbar{} \texttt{12} {]} - Include deterministic seasonal factors in
   the trend.
 \end{itemize}
 
 \paragraph{Description}
 
 \paragraph{Example}


