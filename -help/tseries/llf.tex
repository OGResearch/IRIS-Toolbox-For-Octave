

    \filetitle{llf}{Local level filter (aka random walk plus white noise) with tunes}{tseries/llf}

	\paragraph{Syntax}\label{syntax}

\begin{verbatim}
[T,C,CutOff,Lambda] = llf(X)
[T,C,CutOff,Lambda] = llf(X,Range,...)
\end{verbatim}

\paragraph{Syntax with output arguments
swapped}\label{syntax-with-output-arguments-swapped}

\begin{verbatim}
[C,T,CutOff,Lambda] = llf2(X)
[C,T,CutOff,Lambda] = llf2(X,Range,...)
\end{verbatim}

\paragraph{Input arguments}\label{input-arguments}

\begin{itemize}
\item
  \texttt{X} {[} tseries {]} - Input tseries object that will be
  filtered.
\item
  \texttt{Range} {[} numeric {]} - Date range on which the input data
  will be filtered; \texttt{Range} can be \texttt{Inf},
  \texttt{{[}startdata,Inf{]}}, or \texttt{{[}-Inf,enddate{]}}; if not
  specifired, \texttt{Inf} (i.e.~the entire available range of the input
  series) is used.
\end{itemize}

\paragraph{Output arguments}\label{output-arguments}

\begin{itemize}
\item
  \texttt{T} {[} tseries {]} - Low-frequency (trend) component.
\item
  \texttt{C} {[} tseries {]} - High-frequency (cyclical or gap)
  component.
\item
  \texttt{CutOff} {[} numeric {]} - Cut-off periodicity; periodicities
  above the cut-off are attributed to trends, periodicities below the
  cut-off are attributed to gaps.
\item
  \texttt{Lambda} {[} numeric {]} - Smoothing parameter actually used;
  this output argument is useful when the option \texttt{'cutoff='} is
  used instead of \texttt{'lambda='}.
\end{itemize}

\paragraph{Options}\label{options}

\begin{itemize}
\item
  \texttt{'cutoff='} {[} numeric \textbar{} \emph{empty} {]} - Cut-off
  periodicity in periods (depending on the time series frequency); this
  option can be specified instead of \texttt{'lambda='}; the smoothing
  parameter will be then determined based on the cut-off periodicity.
\item
  \texttt{'cutoffYear='} {[} numeric \textbar{} \emph{empty} {]} -
  Cut-off periodicity in years; this option can be specified instead of
  \texttt{'lambda='}; the smoothing parameter will be then determined
  based on the cut-off periodicity.
\item
  \texttt{'gamma='} {[} numeric \textbar{} tseries \textbar{} \emph{1}
  {]} - Weight or weights on the deviations of the trend from
  observations; it only makes sense to use this option to make the
  signal-to-noise ratio time-varying; see the optimisation problem
  below.
\item
  \texttt{'drift='} {[} numeric \textbar{} tseries \textbar{}
  \emph{\texttt{0}} {]} - Deterministic drift in the trend.
\end{itemize}

\texttt{'infoSet='} {[} \texttt{1} \textbar{} \emph{\texttt{2}} {]} -
Information set assumption used in the filter: \texttt{1} runs a
one-sided filter, \texttt{2} runs a two-sided filter.

\begin{itemize}
\item
  \texttt{'lambda='} {[} numeric \textbar{} \emph{\texttt{10 freq}} {]}
  - Smoothing parameter; needs to be specified for tseries objects with
  indeterminate frequency.
\item
  \texttt{'level='} {[} tseries {]} - Time series with soft and hard
  tunes on the level of the trend.
\item
  \texttt{'change='} {[} tseries {]} - Time series with soft and hard
  tunes on the change in the trend.
\item
  \texttt{'log='} {[} \texttt{true} \textbar{} \emph{\texttt{false}} {]}
  - Logarithmise the data before filtering, de-logarithmise afterwards.
\end{itemize}

\paragraph{Description}\label{description}

\subparagraph{The underlying optimisation
problem}\label{the-underlying-optimisation-problem}

The function \texttt{llf} solves a constrained optimisation problem
described by the following Lagrangian

\[\min_{\bar y_t, \omega_t, \sigma_t}
\underbrace{
\tsum \lambda \left( \Delta \bar y_t - \delta_t \right)^2
+ \tsum \gamma_t \left( \bar y_t - y_t \right)^2}_\text{Plain local level
filter with time-varying signal-to-noise ratio} + \cdots\] \[\cdots +
\underbrace{\tsum u_t \left( \bar y_t - a_t \right)^2}_\text{Soft level tunes}
+ \underbrace{\tsum v_t \left( \Delta \bar y_t - b_t
\right)^2}_\text{Soft growth tunes} +
\underbrace{\tsum \omega_t \left( \bar y_t - c_t \right)}_\text{Hard level tunes}
+ \underbrace{\tsum \sigma_t \left( \Delta \bar y_t - d_t
\right)}_\text{Hard growth tunes},\]

where

\begin{itemize}
\itemsep1pt\parskip0pt\parsep0pt
\item
  $\Delta$ is the first-difference operator;
\item
  $\lambda$ is a (scalar) smoothing parameter;
\item
  $y_t$ are user-supplied observations;
\item
  $\bar y_t$ is the fitted trend;
\item
  $\delta_t$ is a user-supplied drift, either constant or time-varying,
  enetered in the option \texttt{'drift='};
\item
  $\gamma_t$ are user-supplied weights to modify the basic
  signal-to-noise ratio over time (the default setting is $\gamma_t=1$),
  entered in the option \texttt{'gamma='};
\item
  $a_t$ and $u_t$ are soft tunes on the level of the trend and the
  weights associated with these soft level tunes, respectively, entered
  together as complex numbers in the option \texttt{'level='};
\item
  $b_t$ and $v_t$ are soft tunes on the change in the level of the trend
  and the weights associated with these soft growth tunes, respectively,
  entered together as complex numbers in the option \texttt{'growth='};
\item
  $c_t$ are hard tunes on the level of the trend, entered as real
  numbers in the option \texttt{'level='};
\item
  $d_t$ are hard tunes on the change in the level of the trend, entered
  as real numbers in the option \texttt{'growth='};
\item
  $\omega_t$ are lagrange multipliers on the hard level tunes (note that
  these are computed as part of the optimisation problem, not entered by
  the user);
\item
  $\sigma_t$ are lagrange multipliers on the hard growth tunes (note
  that these are computed as part of the optimisation problem, not
  entered by the user).
\end{itemize}

Each of the summations in the above Lagrangian goes over those periods
in which the respective bracketed terms are defined (observations or
tunes exist). You can combine any number of any tunes in one run of
\texttt{llf}, including out-of-sample tunes (see below).

\subparagraph{How to enter the tunes}\label{how-to-enter-the-tunes}

\begin{itemize}
\item
  The soft and hard tunes on the level of the trend are entered as time
  series through the option \texttt{'level='}.
\item
  The soft and hard tunes on the change in the trend are entered as time
  series through the option \texttt{'change='}.
\item
  In the tseries objects entered through \texttt{'level='} and/or
  \texttt{'change='}, you can combine any number of hard and soft tune.
  In each particular period, you can obviously specify only a hard tune
  or only a soft tune. You can think of hard tunes as a special case of
  soft tunes with infinitely large weights.
\item
  A hard tune is specified as a plain real number (i.e.~a number with a
  zero complex part).
\item
  A soft tune must be entered as a complex number whose real part
  specifies the tune itself, and the imaginary part specifies the
  \emph{inverse} of the weight, i.e. $1/v_t$ or $1/u_t$, on that tune in
  that period. Note that if the weight goes to infinity, the imaginary
  part becomes zero and the tune becomes a hard tune.
\end{itemize}

\subparagraph{Out-of-sample tunes}\label{out-of-sample-tunes}

Tunes can be imposed also at dates before the first observation of the
input series, or after the last observation. In other words, the time
series in \texttt{'level='} and/or \texttt{'growth='} can have a more
extended range (at either side) than the filtered input series.

\subparagraph{Default smoothing
parameters}\label{default-smoothing-parameters}

If the user does not specify the smoothing parameter using the
\texttt{'lambda='} option, a default value is used. The default value is
based on common practice and can be calculated using the date frequency
of the input time series as
\texttt{\textbackslash{}lambda = 10 \textbackslash{}cdot f}, where
\texttt{f} is the frequency (yearly=1, half-yearly=2, quarterly=4,
bi-monthly=6, monthly=12). This gives the following default values:

\begin{itemize}
\itemsep1pt\parskip0pt\parsep0pt
\item
  10 for yearly time series (cut-off periodicity of 19.79 years);
\item
  20 for half-yearly time series (cut-off periodicity of 14.02 years);
\item
  40 for quarterly time series (cut-off periodicity of 9.92 years);
\item
  60 for bi-monthly time series (cut-off periodicity of 8.11 years);
\item
  120 for monthly time series (cut-off periodicity of 5.73 years).
\end{itemize}

Note that there is no default value for data with indeterminate or daily
frequency: for these types of time series, you must always use the
option ``lambda=''.

\paragraph{Example}\label{example}


