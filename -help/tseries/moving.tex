

    \filetitle{moving}{Apply function to moving window of observations}{tseries/moving}

	\paragraph{Syntax}
 
 \begin{verbatim}
 X = moving(X)
 X = moving(X,Range,...)
 \end{verbatim}
 
 \paragraph{Input arguments}
 
 \begin{itemize}
 \item
   \texttt{X} {[} tseries {]} - Tseries object on whose observations the
   function will be applied.
 \item
   \texttt{Range} {[} numeric \textbar{} Inf {]} - Range on which the
   moving function will be applied; \texttt{Inf} means the entire range
   on which the time series is defined.
 \end{itemize}
 
 \paragraph{Output arguments}
 
 \begin{itemize}
 \item
   \texttt{X} {[} tseries {]} - Output time series.
 \end{itemize}
 
 \paragraph{Options}
 
 \begin{itemize}
 \item
   \texttt{'function='} {[} function\_handle \textbar{} \texttt{@mean}
   {]} - Function to be applied to a moving window of observations.
 \item
   \texttt{'window='} {[} numeric \textbar{} \emph{\texttt{Inf}} {]} -
   The window of observations where 0 means the current date, -1 means
   one period lag, etc. Inf means that the last n observations (including
   the current one) are used, where n is the frequency of the input data.
 \end{itemize}
 
 \paragraph{Description}
 
 \paragraph{Example}


