

    \filetitle{fft}{Discrete Fourier transform of tseries object}{tseries/fft}

	\paragraph{Syntax}\label{syntax}

\begin{verbatim}
[y,range,freq,per] = fft(x)
[y,range,freq,per] = fft(x,range,...)
\end{verbatim}

\paragraph{Input arguments}\label{input-arguments}

\begin{itemize}
\item
  \texttt{x} {[} tseries {]} - Input tseries object that will be
  transformed.
\item
  \texttt{range} {[} numeric \textbar{} Inf {]} - Date range.
\end{itemize}

\paragraph{Output arguments}\label{output-arguments}

\begin{itemize}
\item
  \texttt{y} {[} numeric {]} - Fourier transform with data organised in
  columns.
\item
  \texttt{range} {[} numeric {]} - Actually used date range.
\item
  \texttt{freq} {[} numeric {]} - Frequencies corresponding to FFT
  vector elements.
\item
  \texttt{per} {[} numeric {]} - Periodicities corresponding to FFT
  vector elements.
\end{itemize}

\paragraph{Options}\label{options}

\begin{itemize}
\itemsep1pt\parskip0pt\parsep0pt
\item
  \texttt{'full='} {[} \texttt{true} \textbar{} \emph{\texttt{false}}
  {]} - Return Fourier transform on the whole interval {[}0,2*pi{]}; if
  false only the interval {[}0,pi{]} is returned.
\end{itemize}

\paragraph{Description}\label{description}

\paragraph{Example}\label{example}

\}


