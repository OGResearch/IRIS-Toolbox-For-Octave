

    \filetitle{bar}{Bar graph for tseries objects}{tseries/bar}

	\paragraph{Syntax}

\begin{verbatim}
[H,Range] = bar(X,...)
[H,Range] = bar(Range,X,...)
[H,Range] = bar(Ax,Range,X,...)
\end{verbatim}

\paragraph{Input arguments}

\begin{itemize}
\item
  \texttt{Ax} {[} handle \textbar{} numeric {]} - Handle to axes in
  which the graph will be plotted; if not specified, the current axes
  will used.
\item
  \texttt{Range} {[} numeric {]} - Date Range; if not specified the
  entire Range of the input tseries object will be plotted.
\item
  \texttt{X} {[} tseries {]} - Input tseries object whose columns will
  be ploted as a bar graph.
\end{itemize}

\paragraph{Output arguments}

\begin{itemize}
\item
  \texttt{H} {[} handle \textbar{} numeric {]} - Handles to bars
  plotted.
\item
  \texttt{Range} {[} numeric {]} - Actually plotted date Range.
\end{itemize}

\paragraph{Options}

\begin{itemize}
\item
  \texttt{'dateFormat='} {[} char \textbar{}
  \emph{\texttt{irisget('plotdateformat')}} {]} - Date format for the
  tick marks on the x-axis.
\item
  \texttt{'dateTick='} {[} numeric \textbar{} \emph{\texttt{Inf}} {]} -
  Vector of dates locating tick marks on the x-axis; Inf means they will
  be created automatically.
\item
  \texttt{'tight='} {[} \texttt{true} \textbar{} \emph{\texttt{false}}
  {]} - Make the y-axis tight.
\end{itemize}

See help on built-in \texttt{bar} function for other options available.

\paragraph{Description}

\paragraph{Example}


