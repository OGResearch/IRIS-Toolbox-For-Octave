

    \filetitle{bar}{Bar graph for tseries objects}{tseries/bar}

	\paragraph{Syntax}\label{syntax}

\begin{verbatim}
[H,Range] = bar(X,...)
[H,Range] = bar(Range,X,...)
[H,Range] = bar(Ax,Range,X,...)
\end{verbatim}

\paragraph{Input arguments}\label{input-arguments}

\begin{itemize}
\item
  \texttt{Ax} {[} handle \textbar{} numeric {]} - Handle to axes in
  which the graph will be plotted; if not specified, the current axes
  will used.
\item
  \texttt{Range} {[} numeric {]} - Date Range; if not specified the
  entire Range of the input tseries object will be plotted.
\item
  \texttt{X} {[} tseries {]} - Input tseries object whose columns will
  be ploted as a bar graph.
\end{itemize}

\paragraph{Output arguments}\label{output-arguments}

\begin{itemize}
\item
  \texttt{H} {[} handle \textbar{} numeric {]} - Handles to bars
  plotted.
\item
  \texttt{Range} {[} numeric {]} - Actually plotted date Range.
\end{itemize}

\paragraph{Options}\label{options}

See help on \href{tseries/bar}{\texttt{tseries/bar}} and the built-in
function \texttt{bar} for all options available.

\paragraph{Description}\label{description}

\paragraph{Example}\label{example}


