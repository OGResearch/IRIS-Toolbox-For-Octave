
    \foldertitle{BVAR}{Bayesian VAR prior dummies: BVAR package}{BVAR/Contents}

	The BVAR package can be used to create the basic types of prior dummy
 observations when estimating Bayesian VAR models. The dummy observations
 are passed in the \url{VAR/estimate} function through the
 \texttt{'BVAR='} option.
 
 \paragraph{Constructing dummy observations}
 
 \begin{itemize}
 \item
   \href{BVAR/covmat}{\texttt{covmat}} - Covariance matrix prior dummy
   observations for BVARs.
 \item
   \href{BVAR/litterman}{\texttt{litterman}} - Litterman's prior dummy
   observations for BVARs.
 \item
   \href{BVAR/sumofcoeff}{\texttt{sumofcoeff}} - Doan et al
   sum-of-coefficient prior dummy observations for BVARs.
 \item
   \href{BVAR/uncmean}{\texttt{uncmean}} - Unconditional-mean dummy (or
   Sims' initial dummy) observations for BVARs.
 \item
   \href{BVAR/user}{\texttt{user}} - User-supplied prior dummy
   observations for BVARs.
 \end{itemize}
 
 \paragraph{Weights on prior dummy observations}
 
 The prior dummies produced by \href{BVAR/litterman}{\texttt{litterman}},
 \href{BVAR/uncmean}{\texttt{uncmean}},
 \href{BVAR/sumofcoeff}{\texttt{sumofcoeff}} can be weighted up or down
 using the input argument \texttt{Mu}. To give the weight a clear
 interpretation, use the option \texttt{'stdize=' true} when estimating
 the VAR. In that case, setting \texttt{Mu} to \texttt{sqrt(N)} means the
 prior dummies are worth a total of extra \texttt{N} artifical
 observations; the weight can be related to the actual number of
 observations used in estimation.
 
 \paragraph{Getting help on BVAR functions}
 
 \begin{verbatim}
 help BVAR
 help BVAR/function_name
 \end{verbatim}



