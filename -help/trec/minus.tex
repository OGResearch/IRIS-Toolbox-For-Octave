

    \filetitle{minus}{Create time-recursive lag of tseries object}{trec/minus}

	\paragraph{Syntax}\label{syntax}

\begin{verbatim}
X{T-K}
\end{verbatim}

\paragraph{Input arguments}\label{input-arguments}

\begin{itemize}
\item
  \texttt{X} {[} tseries {]} - Tseries object whose time-recursive lag
  will be created.
\item
  \texttt{T} {[} trec {]} - Initialized trec object.
\item
  \texttt{K} {[} numeric {]} - Integer scalar specifying the lag.
\end{itemize}

\paragraph{Description}\label{description}

The tseries object, \texttt{X}, referenced by \texttt{T-K} in a
time-recursive expression will, in each iteration, return a value that
corresponds to period \texttt{t-K}, where \texttt{t} is the currently
processed date from the vector of dates (or date range) associated with
the trec object, \texttt{T}.


