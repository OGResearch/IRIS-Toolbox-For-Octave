

    \filetitle{trec}{Create new recursive time subscript object}{trec/trec}

	\paragraph{Syntax}\label{syntax}

\begin{verbatim}
T = trec(Dates)
\end{verbatim}

\paragraph{Input arguments}\label{input-arguments}

\begin{itemize}
\itemsep1pt\parskip0pt\parsep0pt
\item
  \texttt{Dates} {[} numeric {]} - Vector of dates or date range on
  which the final time-recursive expression will be evaluated.
\end{itemize}

\paragraph{Output arguments}\label{output-arguments}

\begin{itemize}
\itemsep1pt\parskip0pt\parsep0pt
\item
  \texttt{T} {[} trec {]} - New time-recursive subscript object.
\end{itemize}

\paragraph{Description}\label{description}

Time-recursive subscript objects are used to reference tseries objects
on both the left-hand side and the right-hand side of a time-recursive
assignment. The assignment is then evaluated for each date in
\texttt{Dates}, from the first to the last.

See more on time-recursive expressions in
\href{trec/Contents}{Contents}, including the description of instances
in which IRIS fails to evaluate the time-recursive expressions
correctly.

\paragraph{Example}\label{example}

Construct a first-order autoregressive process with normally distributed
residuals:

\begin{verbatim}
T = trec(qq(2010,1):qq(2020,4));
x = tseries(qq(2009,4),10);
e = tseries(qq(2010,1):qq(2020,4),@randn);
x(T) = 10 + 0.8*x(T-1) + e(T);
\end{verbatim}


