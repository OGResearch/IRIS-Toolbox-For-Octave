

    \filetitle{eval}{Evaluate posterior density at specified points}{poster/eval}

	\paragraph{Syntax}
 
 \begin{verbatim}
 [X,L,PP,SrfP,FrfP] = eval(Pos)
 [X,L,PP,SrfP,FrfP] = eval(Pos,P)
 \end{verbatim}
 
 \paragraph{Input arguments}
 
 \begin{itemize}
 \item
   \texttt{Pos} {[} poster {]} - Posterior object returned by the
   \url{model/estimate} function.
 \item
   \texttt{P} {[} struct {]} - Struct with parameter values at which the
   posterior density will be evaluated; if \texttt{P} is not specified,
   the posterior density at the point of the estimated mode is returned.
 \end{itemize}
 
 \paragraph{Output arguments}
 
 \begin{itemize}
 \item
   \texttt{X} {[} numeric {]} - The value of log posterior density
   evaluated at \texttt{P}; N.B. the returned value is log posterior, and
   not minus log posterior.
 \item
   \texttt{L} {[} numeric {]} - Contribution of data likelihood to log
   posterior.
 \item
   \texttt{PP} {[} numeric {]} - Contribution of parameter priors to log
   posterior.
 \item
   \texttt{SrfP} {[} numeric {]} - Contribution of shock response
   function priors to log posterior.
 \item
   \texttt{FrfP} {[} numeric {]} - Contribution of frequency response
   function priors to log posterior.
 \end{itemize}
 
 \paragraph{Description}
 
 The total log posterior consists, in general, of the four contributions
 listed above:
 
 \begin{verbatim}
 X = L + PP + SrfP + FrfP.
 \end{verbatim}
 
 \paragraph{Example}


