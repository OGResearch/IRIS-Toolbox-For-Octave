

    \filetitle{poster}{Posterior objects and functions}{poster/poster}

	Posterior objects, \texttt{poster}, are used to evaluate the behaviour
of the posterior dsitribution, and to draw model parameters from the
posterior distibution.

Posterior objects are set up within the
\href{model/estimate}{\texttt{model/estimate}} function and returned as
the second output argument - the set up and initialisation of the
posterior object is fully automated in this case. Alternatively, you can
set up a posterior object manually, by setting all its properties
appropriately.

Poster methods:

\paragraph{Constructor}\label{constructor}

\begin{itemize}
\itemsep1pt\parskip0pt\parsep0pt
\item
  \href{poster/poster}{\texttt{poster}} - Posterior objects and
  functions.
\end{itemize}

\paragraph{Evaluating posterior
density}\label{evaluating-posterior-density}

\begin{itemize}
\itemsep1pt\parskip0pt\parsep0pt
\item
  \href{poster/arwm}{\texttt{arwm}} - Adaptive random-walk Metropolis
  posterior simulator.
\item
  \href{poster/eval}{\texttt{eval}} - Evaluate posterior density at
  specified points.
\end{itemize}

\paragraph{Chain statistics}\label{chain-statistics}

\begin{itemize}
\itemsep1pt\parskip0pt\parsep0pt
\item
  \href{poster/stats}{\texttt{stats}} - Evaluate selected statistics of
  ARWM chain.
\end{itemize}

\paragraph{Getting on-line help on model
functions}\label{getting-on-line-help-on-model-functions}

\begin{verbatim}
help poster
help poster/function_name
\end{verbatim}


