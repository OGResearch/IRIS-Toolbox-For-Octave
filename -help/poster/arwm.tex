

    \filetitle{arwm}{Adaptive random-walk Metropolis posterior simulator}{poster/arwm}

	\paragraph{Syntax}
 
 \begin{verbatim}
 [Theta,LogPost,AR,Scale,FinalCov] = arwm(Pos,NDraw,...)
 \end{verbatim}
 
 \paragraph{Input arguments}
 
 \begin{itemize}
 \item
   \texttt{Pos} {[} poster {]} - Initialised posterior simulator object.
 \item
   \texttt{NDraw} {[} numeric {]} - Length of the chain not including
   burn-in.
 \end{itemize}
 
 \paragraph{Output arguments}
 
 \begin{itemize}
 \item
   \texttt{Theta} {[} numeric {]} - MCMC chain with individual parameters
   in rows.
 \item
   \texttt{LogPost} {[} numeric {]} - Vector of log posterior density (up
   to a constant) in each draw.
 \item
   \texttt{AR} {[} numeric {]} - Vector of cumulative acceptance ratios
   in each draw.
 \item
   \texttt{Scale} {[} numeric {]} - Vector of proposal scale factors in
   each draw.
 \item
   \texttt{FinalCov} {[} numeric {]} - Final proposal covariance matrix;
   the final covariance matrix of the random walk step is
   Scale(end)\^{}2*FinalCov.
 \end{itemize}
 
 \paragraph{Options}
 
 \begin{itemize}
 \item
   \texttt{'adaptProposalCov='} {[} numeric \textbar{}
   \emph{\texttt{0.5}} {]} - Speed of adaptation of the Cholesky factor
   of the proposal covariance matrix towards the target acceptanace
   ratio, \texttt{targetAR}; zero means no adaptation.
 \item
   \texttt{'adaptScale='} {[} numeric \textbar{} \emph{\texttt{1}} {]} -
   Speed of adaptation of the scale factor to deviations of acceptance
   ratios from the target ratio, \texttt{targetAR}.
 \item
   \texttt{'burnin='} {[} numeric \textbar{} \emph{\texttt{0.10}} {]} -
   Number of burn-in draws entered either as a percentage of total draws
   (between 0 and 1) or directly as a number (integer greater that one).
   Burn-in draws will be added to the requested number of draws
   \texttt{ndraw} and discarded after the posterior simulation.
 \item
   \texttt{'estTime='} {[} \texttt{true} \textbar{} \emph{\texttt{false}}
   {]} - Display and update the estimated time to go in the command
   window.
 \item
   \texttt{'firstPrefetch='} {[} numeric \textbar{} \emph{\texttt{Inf}}
   {]} - First draw where parallelised pre-fetching will be used;
   \texttt{Inf} means no pre-fetching.
 \item
   \texttt{'gamma='} {[} numeric \textbar{} \emph{\texttt{0.8}} {]} - The
   rate of decay at which the scale and/or the proposal covariance will
   be adapted with each new draw.
 \item
   \texttt{'initScale='} {[} numeric \textbar{} \texttt{1/3} {]} -
   Initial scale factor by which the initial proposal covariance will be
   multiplied; the initial value will be adapted to achieve the target
   acceptance ratio.
 \item
   \texttt{'lastAdapt='} {[} numeric \textbar{} \emph{\texttt{Inf}} {]} -
   Last draw in which proposal covariance will be adapted; \texttt{Inf}
   means adaptation will continue until the last draw.
 \item
   \texttt{'nStep='} {[} numeric \textbar{} *\texttt{1} {]} - Number of
   pre-fetched steps computed in parallel; only works with
   \texttt{firstPrefetch=} smaller than \texttt{NDraw}.
 \item
   \texttt{'progress='} {[} \texttt{true} \textbar{}
   \emph{\texttt{false}} {]} - Display progress bar in the command
   window.
 \item
   \texttt{'saveAs='} {[} char \textbar{} \emph{empty} {]} - File name
   where results will be saved when the option \texttt{'saveEvery='} is
   used.
 \item
   \texttt{'saveEvery='} {[} numeric \textbar{} \emph{\texttt{Inf}} {]} -
   Every N draws will be saved to an HDF5 file, and removed from
   workspace immediately; no values will be returned in the output
   arguments \texttt{Theta}, \texttt{LogPost}, \texttt{AR},
   \texttt{Scale}; the option \texttt{'saveAs='} must be used to specify
   the file name; \texttt{Inf} means a normal run with no saving.
 \item
   \texttt{'targetAR='} {[} numeric \textbar{} \emph{\texttt{0.234}} {]}
   - Target acceptance ratio.
 \end{itemize}
 
 \paragraph{Description}
 
 Use the \url{poster/stats} function to process the simulated chain of
 parameters, and calculate selected statistics.
 
 \subparagraph{Parallelised ARWM}
 
 Set \texttt{'nStep='} greater than \texttt{1}, and
 \texttt{'firstPrefetch='} smaller than \texttt{NDraw} to start a
 pre-fetching parallelised algorithm (pre-fetched will be all draws
 starting from \texttt{'firstPrefetch='}); to that end, a pool of
 parallel workers (using e.g. \texttt{matlabpool} from the Parallel
 Computing Toolbox) must be opened before calling \texttt{arwm}.
 
 With pre-fetching, all possible paths \texttt{'nStep='} steps ahead
 (i.e.~all possible combinations of reject/accept) are pre-evaluated in
 parallel, and then the resulting path is selected. Adapation then occurs
 only every \texttt{'nStep='} steps, and hence the results will always
 somewhat differ from a serial run. Identical results can be obtained by
 turning down adaptation before pre-fetching starts, i.e.~by setting
 \texttt{'lastAdapt='} smaller than \texttt{'firstPrefetch='} (and,
 obviously, by re-setting the random number generator).
 
 \paragraph{Example}


