

    \filetitle{regen}{Regeneration time MCMC Metropolis posterior simulator}{poster/regen}

	\paragraph{Syntax}

\begin{verbatim}
[Theta,LogPost,AR,Scale,FinalCov] = regen(Pos,NDraw,...)
\end{verbatim}

\paragraph{Input arguments}

\begin{itemize}
\item
  \texttt{Pos} {[} poster {]} - Initialised posterior simulator object.
\item
  \texttt{NDraw} {[} numeric {]} - Length of the chain not including
  burn-in.
\end{itemize}

\paragraph{Output arguments}

\begin{itemize}
\item
  \texttt{Theta} {[} numeric {]} - MCMC chain with individual parameters
  in rows.
\item
  \texttt{LogPost} {[} numeric {]} - Vector of log posterior density (up
  to a constant) in each draw.
\item
  \texttt{AR} {[} numeric {]} - Vector of cumulative acceptance ratios
  in each draw.
\item
  \texttt{Scale} {[} numeric {]} - Vector of proposal scale factors in
  each draw.
\item
  \texttt{FinalCov} {[} numeric {]} - Final proposal covariance matrix;
  the final covariance matrix of the random walk step is
  \texttt{Scale(end)\^{}2*FinalCov}.
\end{itemize}

\paragraph{Options}

\paragraph{References}

\begin{enumerate}
\def\labelenumi{\arabic{enumi}.}
\itemsep1pt\parskip0pt\parsep0pt
\item
  Brockwell, A.E., and Kadane, J.B., 2004. ``Identification of
  Regeneration Times in MCMC Simulation, with Application to Adaptive
  Schemes,'' mimeo, Carnegie Mellon University.
\end{enumerate}

\paragraph{Example}


