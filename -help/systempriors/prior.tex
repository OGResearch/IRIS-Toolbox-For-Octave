

    \filetitle{prior}{Add new prior to system priors object}{systempriors/prior}

	\paragraph{Syntax}

\begin{verbatim}
S = prior(S,Expr,PriorFunc,...)
S = prior(S,Expr,[],...)
\end{verbatim}

\paragraph{Input arguments}

\begin{itemize}
\item
  \texttt{S} {[} systempriors {]} - System priors object.
\item
  \texttt{Expr} {[} char {]} - Expression that defines a value for which
  a prior density will be defined; see Description for system properties
  that can be referred to in the expression.
\item
  \texttt{PriorFunc} {[} function\_handle \textbar{} empty {]} -
  Function handle returning the log of prior density; empty prior
  function, \texttt{{[}{]}}, means a uniform prior.
\end{itemize}

\paragraph{Output arguments}

\begin{itemize}
\itemsep1pt\parskip0pt\parsep0pt
\item
  \texttt{S} {[} systempriors {]} - The system priors object with the
  new prior added.
\end{itemize}

\paragraph{Options}

\begin{itemize}
\item
  \texttt{'lowerBound='} {[} numeric \textbar{} \emph{\texttt{-Inf}} {]}
  - Lower bound for the prior.
\item
  \texttt{'upperBound='} {[} numeric \textbar{} \emph{\texttt{Inf}} {]}
  - Upper bound for the prior.
\end{itemize}

\paragraph{Description}

\subparagraph{System properties that can be used in
\texttt{Expr}}

\begin{itemize}
\item
  \texttt{srf{[}VarName,ShockName,T{]}} - Plain shock response function
  of variables \texttt{VarName} to shock \texttt{ShockName} in period
  \texttt{T}. Mind the square brackets.
\item
  \texttt{ffrf{[}VarName,MVarName,Freq{]}} - Filter frequency response
  function of transition variables \texttt{TVarName} to measurement
  variable \texttt{MVarName} at frequency \texttt{Freq}. Mind the square
  brackets.
\item
  \texttt{corr{[}VarName1,VarName2,Lag{]}} - Correlation between
  variable \texttt{VarName1} and variables \texttt{VarName2} lagged by
  \texttt{Lag} periods.
\item
  \texttt{spd{[}VarName1,VarName2,Freq{]}} - Spectral density between
  variables \texttt{VarName1} and \texttt{VarName2} at frequency
  \texttt{Freq}.
\end{itemize}

If a variable is declared as a
\href{modellang/logvariables}{\texttt{log-variable}}, it must be
referred to as \texttt{log(VarName)} in the above expressions, and the
log of that variables is returned, e.g.
\texttt{srf{[}log(VarName),ShockName,T{]}}. or
\texttt{ffrf{[}log(TVarName),MVarName,T{]}}.

\subparagraph{Expressions involving combinations or functions of
parameters}

Model parameter names can be referred to in \texttt{Expr} preceded by a
dot (period), e.g. \texttt{.alpha\^{}2 + .beta\^{}2} defines a prior on
the sum of squares of the two parameters (\texttt{alpha} and
\texttt{beta}).

\paragraph{Example}

Create a new, empty systemprios object based on an existing model.

\begin{verbatim}
s = systempriors(m);
\end{verbatim}

Add a prior on minus the shock response function of variable
\texttt{ygap} to shock \texttt{eps\_pie} in period 4. The prior density
is lognormal with mean 0.3 and std deviation 0.05;

\begin{verbatim}
s = prior(s,'-srf[ygap,eps_pie,4]',logdist.lognormal(0.3,0.05));
\end{verbatim}

Add a prior on the gain of the frequency response function of transition
variable \texttt{ygap} to measurement variable `y' at frequency
\texttt{2*pi/40}. The prior density is normal with mean 0.5 and std
deviation 0.01. This prior says that we wish to keep the cut-off
periodicity for trend-cycle decomposition close to 40 periods.

\begin{verbatim}
s = prior(s,'abs(ffrf[ygap,y,2*pi/40])',logdist.normal(0.5,0.01));
\end{verbatim}

Add a prior on the sum of parameters \texttt{alpha1} and
\texttt{alpha2}. The prior is normal with mean 0.9 and std deviation
0.1, but the sum is forced to be between 0 and 1 by imposing lower and
upper bounds.

\begin{verbatim}
s = prior(s,'alpha1+alpha2',logdist.normal(0.9,0.1), ...
    'lowerBound=',0,'upperBound=',1);
\end{verbatim}


