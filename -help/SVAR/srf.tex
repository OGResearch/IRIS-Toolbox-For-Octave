

    \filetitle{srf}{Shock (impulse) response function}{SVAR/srf}

	\paragraph{Syntax}
 
 \begin{verbatim}
 [R,Cum] = srf(V,NPer)
 [R,Cum] = srf(V,Range)
 \end{verbatim}
 
 \paragraph{Input arguments}
 
 \begin{itemize}
 \item
   \texttt{V} {[} SVAR {]} - SVAR object for which the impulse response
   function will be computed.
 \item
   \texttt{NPer} {[} numeric {]} - Number of periods.
 \item
   \texttt{Range} {[} numeric {]} - Date range.
 \end{itemize}
 
 \paragraph{Output arguments}
 
 \begin{itemize}
 \item
   \texttt{R} {[} tseries \textbar{} struct {]} - Shock response
   functions.
 \item
   \texttt{Cum} {[} tseries \textbar{} struct {]} - Cumulative shock
   response functions.
 \end{itemize}
 
 \paragraph{Options}
 
 \begin{itemize}
 \item
   \texttt{'presample='} {[} \texttt{true} \textbar{}
   \emph{\texttt{false}} {]} - Include zeros for pre-sample initial
   conditions in the output data.
 \item
   \texttt{'select='} {[} cellstr \textbar{} char \textbar{} logical
   \textbar{} numeric \textbar{} \emph{\texttt{Inf}} {]} - Selection of
   shocks to which the responses will be simulated.
 \end{itemize}
 
 \paragraph{Description}
 
 For backward compatibility, the following calls into the \texttt{srf}
 function is also possible:
 
 \begin{verbatim}
 [~,~,s,c] = srf(this,nper)
 [~,~,s,c] = srf(this,range)
 \end{verbatim}
 
 \paragraph{Example}


