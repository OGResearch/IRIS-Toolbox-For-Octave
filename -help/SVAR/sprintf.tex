

    \filetitle{sprintf}{Format SVAR as a model code and write to text string}{SVAR/sprintf}

	\paragraph{Syntax}
 
 \begin{verbatim}
 [C,D] = sprintf(S,...)
 \end{verbatim}
 
 \paragraph{Input arguments}
 
 \begin{itemize}
 \item
   \texttt{S} {[} SVAR {]} - SVAR object that will be written as a model
   code.
 \item
   Output arguments
 \item
   \texttt{C} {[} cellstr {]} - Text string with the model code for each
   parameterisation.
 \item
   \texttt{D} {[} cell {]} - Parameter database for each
   parameterisation; if \texttt{'hardParameters='} is true, the databases
   will be empty.
 \end{itemize}
 
 \paragraph{Options}
 
 \begin{itemize}
 \item
   \texttt{'decimal='} {[} numeric \textbar{} \emph{empty} {]} -
   Precision (number of decimals) at which the coefficients will be
   written if \texttt{'hardParameters='} is true; if empty, the
   \texttt{'format='} options is used.
 \item
   \texttt{'declare='} {[} \texttt{true} \textbar{} \emph{\texttt{false}}
   {]} - Add declaration blocks and keywords for VAR variables, shocks,
   and equations.
 \item
   \texttt{'eNames='} {[} cellstr \textbar{} char \textbar{} \emph{empty}
   {]} - Names that will be given to the VAR residuals; if empty, the
   names from the SVAR object will be used.
 \item
   \texttt{'format='} {[} char \textbar{} \emph{`\%+.16e'} {]} - Numeric
   format for parameter values; it will be used only if
   \texttt{'decimal='} is empty.
 \item
   \texttt{'hardParameters='} {[} \emph{\texttt{true}} \textbar{}
   \texttt{false} {]} - Print coefficients as hard numbers; otherwise,
   create parameter names and return a parameter database.
 \item
   \texttt{'yNames='} {[} cellstr \textbar{} char \textbar{} \emph{empty}
   {]} - Names that will be given to the variables; if empty, the names
   from the SVAR object will be used.
 \item
   \texttt{'tolerance='} {[} numeric \textbar{} \emph{getrealsmall()} {]}
   - Treat VAR coefficients smaller than \texttt{'tolerance='} in
   absolute value as zeros; zero coefficients will be dropped from the
   model code.
 \end{itemize}
 
 \paragraph{Description}
 
 \paragraph{Example}


