

    \filetitle{fprintf}{Format SVAR as a model code and write to text file}{SVAR/fprintf}

	\paragraph{Syntax}
 
 \begin{verbatim}
 [C,D] = fprintf(S,FName,...)
 \end{verbatim}
 
 \paragraph{Input arguments}
 
 \begin{itemize}
 \item
   \texttt{S} {[} SVAR {]} - SVAR object that will be printed to a model
   file.
 \item
   \texttt{FName} {[} char \textbar{} cellstr {]} - Filename, or filename
   format string, under which the model code will be saved.
 \item
   Output arguments
 \item
   \texttt{C} {[} cellstr {]} - Text string with the model code for each
   parameterisation.
 \item
   \texttt{D} {[} cell {]} - Parameter databases for each
   parameterisation; if \texttt{'hardParameters='} true, the database
   will be empty.
 \end{itemize}
 
 \paragraph{Options}
 
 See help on \href{SVAR/sprintf}{\texttt{sprintf}} for options available.
 
 \paragraph{Description}
 
 \paragraph{Example}


