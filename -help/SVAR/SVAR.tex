

    \filetitle{SVAR}{Convert reduced-form VAR to structural VAR}{SVAR/SVAR}

	\paragraph{Syntax}
 
 \begin{verbatim}
 [S,DATA,B,COUNT] = SVAR(V,DATA,...)
 \end{verbatim}
 
 \paragraph{Input arguments}
 
 \begin{itemize}
 \item
   \texttt{V} {[} VAR {]} - Reduced-form VAR object.
 \item
   \texttt{DATA} {[} struct \textbar{} tseries {]} - Data associated with
   the input VAR object.
 \end{itemize}
 
 \paragraph{Output arguments}
 
 \begin{itemize}
 \item
   \texttt{S} {[} VAR {]} - Structural VAR object.
 \item
   \texttt{DATA} {[} struct \textbar{} tseries {]} - Data with
   transformed structural residuals.
 \item
   \texttt{B} {[} numeric {]} - Impact matrix of structural residuals.
 \item
   \texttt{COUNT} {[} numeric {]} - Number of draws actually performed
   (both successful and unsuccessful) when \texttt{'method'='draw'};
   otherwise \texttt{COUNT=1}.
 \end{itemize}
 
 \paragraph{Options}
 
 \begin{itemize}
 \item
   \texttt{'maxIter='} {[} numeric \textbar{} \emph{\texttt{0}} {]} -
   Maximum number of attempts when \texttt{'method'='draw'}.
 \item
   \texttt{'method='} {[} \emph{\texttt{'chol'}} \textbar{}
   \texttt{'draw'} \textbar{} \texttt{'qr'} \textbar{} \texttt{'svd'} {]}
   - Method that will be used to identify structural residuals.
 \item
   \texttt{'nDraw='} {[} numeric \textbar{} \emph{\texttt{0}} {]} -
   Target number of successful draws when \texttt{'method'='draw'}.
 \item
   \texttt{'reorder='} {[} numeric \textbar{} \emph{empty} {]} - Re-order
   VAR variables before identifying structural residuals, and bring the
   variables back in original order afterwards. Use the option
   '\texttt{backorderResiduals='} to control if also the structural
   residuals are brought back in original order.
 \item
   \texttt{'output='} {[} \emph{\texttt{'auto'}} \textbar{}
   \texttt{'dbase'} \textbar{} \texttt{'tseries'} {]} - Format of output
   data.
 \item
   \texttt{'progress='} {[} \texttt{true} \textbar{}
   \emph{\texttt{false}} {]} - Display progress bar in the command
   window.
 \item
   \texttt{'rank='} {[} numeric \textbar{} \emph{\texttt{Inf}} {]} -
   Reduced rank of the covariance matrix of structural residuals when
   \texttt{'method=' 'svd'}; \texttt{Inf} means full rank is preserved.
 \item
   \texttt{'backorderResiduals='} {[} \emph{\texttt{true}} \textbar{}
   \texttt{false} {]} - Bring the identified structural residuals back in
   original order.
 \item
   \texttt{'std='} {[} numeric \textbar{} \emph{\texttt{1}} {]} - Std
   deviation of structural residuals; the resulting structural covariance
   matrix will be re-scaled (divided) by this factor.
 \item
   \texttt{'test='} {[} char {]} - Works with \texttt{'method=draw'}
   only; a string that will be evaluated for each random draw of the
   impact matrix \texttt{B}. The evaluation must result in \texttt{true}
   or \texttt{false}; only the matrices \texttt{B} that evaluate to
   \texttt{true} will be kept. See Description for more on how to write
   the option \texttt{'test='}.
 \end{itemize}
 
 \paragraph{Description}
 
 \subparagraph{Identification random Householder transformations}
 
 The structural impact matrices \texttt{B} are randomly generated using a
 Householder transformation algorithm. Each matrix is tested by
 evaluating the \texttt{test} string supplied by the user. If it
 evaluates to true the matrix is kept and one more SVAR parameterisation
 is created, if it is false the matrix is discarded.
 
 The \texttt{test} string can refer to the following characteristics:
 
 \begin{itemize}
 \item
   \texttt{S} -- the impulse (or shock) response function; the
   \texttt{S(i,j,k)} element is the response of the \texttt{i}-th
   variable to the \texttt{j}-th shock in period \texttt{k}.
 \item
   \texttt{Y} -- the asymptotic cumulative response function; the
   \texttt{Y(i,j)} element is the asumptotic (long-run) cumulative
   response of the \texttt{i}-th variable to the \texttt{j}-th shock.
 \end{itemize}
 
 \paragraph{Example}


