
    \foldertitle{setup}{Installing IRIS}{setup/Contents}

	\paragraph{Requirements}

\begin{itemize}
\itemsep1pt\parskip0pt\parsep0pt
\item
  Matlab R2010a or later.
\end{itemize}

\paragraph{Optional components}

\subparagraph{Optimization Toolbox}

The Optimization Toolbox is needed to compute the steady state of
non-linear models, and to run estimation.

\subparagraph{LaTeX}

LaTeX is a free typesetting system used to produce PDF reports in IRIS.

\paragraph{Installing IRIS}

\subparagraph{Step 1}

Download the latest IRIS zip archive, \texttt{IRIS\_Tbx\_YYYYMMDD.zip},
from the download area on the website, and save it in a temporary
location on your disk.

\subparagraph{Step 2}

If you are going to install IRIS in a folder where an older version
already resides, you MUST first delete the old version completely.

\subparagraph{Step 3}

Unzip the archive into a folder on your hard drive, e.g.
\texttt{C:\textbackslash{}IRIS\_Tbx}. This folder is called the IRIS
root.

\subparagraph{Step 4}

After installing a new version of IRIS, we recommend that you remove all
older versions of IRIS from the Matlab search path, and restart Matlab.

\subparagraph{Getting started}

Each time you want to start working with IRIS, run the following line

\begin{verbatim}
>> addpath C:\IRIS_Tbx; irisstartup
\end{verbatim}

where \texttt{C:\textbackslash{}IRIS\_Tbx} needs to be, obviously,
replaced with the proper IRIS root folder chosen in Step 3 above.

Alternatively, you can put the IRIS root folder permanently on the
Matlab seatch path (using the menu File - Set Path), and only run the
\texttt{irisstartup} command at the beginning of each IRIS session.

See also the section on \href{config/Contents}{Starting and quitting
IRIS}.

\paragraph{Syntax highlighting}

You can get the IRIS model files syntax-highlighted. Syntax highlighting
improves enormously the readability of the files: it helps you
understand the model better, and discover typos and mistakes more
quickly.

Add any number of extensions you want to use for model files (such as
\texttt{'model'} or \texttt{'iris'}, there is really no limitation) to
the Matlab editor. Open the menu Home - Preferences, unfold
Editor/Debugger and choose Language. Make sure Matlab is selected at the
top as the language. Use the Add button in the File extensions panel to
associate any number of new extensions with the editor. Re-start the
editor. The IRIS model files will be syntax highlighted from that moment
on.

\paragraph{Components distributed with
IRIS}

\subparagraph{X13-ARIMA-SEATS (formerly X12-ARIMA,
X11-ARIMA)}

Courtesy of the U.S. Census Bureau, the X13-ARIMA-SEATS (formerly
X12-ARIMA) program is now incoporated in, and distributed with IRIS. No
extra installation or setup is needed.

\subparagraph{Symbolic/automatic
differentiator}

Symbolic/automatic differentiator. IRIS is equipped with its own
symbolic/automatic differentiator, Sydney. There is no need to have the
Symbolic Math Toolbox as was the case in earlier versions of IRIS.



