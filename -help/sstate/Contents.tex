
    \foldertitle{sstate}{Steady-state objects and functions}{sstate/Contents}

	You can create a steady-state (sstate) object by loading a steady-state
 (sstate) file. The sstate object can be then saved as a stand-alone
 m-file function and repeatedly solved for different parameterisations.
 
 Sstate methods:
 
 \paragraph{Constructor}
 
 \begin{itemize}
 \item
   \href{sstate/sstate}{\texttt{sstate}} - Create new model object based
   on sstate file.
 \end{itemize}
 
 \paragraph{Compiling stand-alone m-file functions}
 
 \begin{itemize}
 \item
   \href{sstate/compile}{\texttt{compile}} - Compile an m-file function
   based on a steady-state file.
 \end{itemize}
 
 \paragraph{Running stand-alone sstate m-file functions}
 
 \begin{itemize}
 \item
   \href{sstate/standalonemfile}{\texttt{standalonemfile}} - Run a
   compiled stand-alone sstate m-file function.
 \end{itemize}
 
 \paragraph{Getting on-line help on sstate functions}
 
 \begin{verbatim}
 help sstate
 help sstate/function_name
 \end{verbatim}



