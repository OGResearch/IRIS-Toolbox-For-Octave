

    \filetitle{sstate}{Create new steady-state object based on sstate file}{sstate/sstate}

	\paragraph{Syntax}

\begin{verbatim}
S = sstate(File,...)
\end{verbatim}

\paragraph{Input arguments}

\begin{itemize}
\itemsep1pt\parskip0pt\parsep0pt
\item
  \texttt{File} {[} char {]} - Name of the steady-state file that will
  loaded and converted to a new sstate object.
\end{itemize}

\paragraph{Output arguments}

\begin{itemize}
\itemsep1pt\parskip0pt\parsep0pt
\item
  \texttt{S} {[} sstate {]} - New sstate object based on the input
  steady-state file.
\end{itemize}

\paragraph{Options}

\begin{itemize}
\itemsep1pt\parskip0pt\parsep0pt
\item
  \texttt{'assign='} {[} struct \textbar{} \emph{empty} {]} - Database
  that will used by the preparser to evaluate conditions and expressions
  in the \texttt{!if} and \texttt{!switch} structures.
\end{itemize}

\paragraph{Description}

\paragraph{Example}


