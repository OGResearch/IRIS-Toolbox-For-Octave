

    \filetitle{nonlinearise}{Select equations for simulation in an exact non-linear mode}{plan/nonlinearise}

	\paragraph{Syntax}
 
 \begin{verbatim}
 P = nonlinearise(P)
 P = nonlinearise(P,Range)
 P = nonlinearise(P,List,Range)
 \end{verbatim}
 
 \paragraph{Input arguments}
 
 \begin{itemize}
 \item
   \texttt{P} {[} plan {]} - Simulation plan object.
 \item
   \texttt{List} {[} cellstr {]} - List of labels for equations that will
   be simulated in an exact non-linear mode; all selected equations must
   be marked by an \href{modellang/exactnonlin}{\texttt{=\#}} mark in the
   model file. If \texttt{List} is not specified, all equations marked in
   the model file will be simulated in a non-linear mode.
 \item
   \texttt{Range} {[} numeric {]} - Date range on which the equations
   will be simulated in an exact non-linear mode; currently, the range
   must start at the start of the plan range. If \texttt{Range} is not
   specified, the equations are non-linearised over the whole simulation
   range.
 \end{itemize}
 
 \paragraph{Output arguments}
 
 \begin{itemize}
 \item
   \texttt{P} {[} plan {]} - Simulation plan with information on exact
   non-linear simulation included.
 \end{itemize}
 
 \paragraph{Description}
 
 \paragraph{Example}


