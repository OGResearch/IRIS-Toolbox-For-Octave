

    \filetitle{plan}{Create new, empty simulation plan object}{plan/plan}

	\paragraph{Syntax}

\begin{verbatim}
P = plan(M,Range)
\end{verbatim}

\paragraph{Input arguments}

\begin{itemize}
\item
  \texttt{M} {[} model {]} - Model object that will be simulated subject
  to this simulation plan.
\item
  \texttt{Range} {[} numeric {]} - Simulation range; this range must
  exactly correspond to the range on which the model will be simulated.
\end{itemize}

\paragraph{Output arguments}

\begin{itemize}
\itemsep1pt\parskip0pt\parsep0pt
\item
  \texttt{P} {[} plan {]} - New, empty simulation plan.
\end{itemize}

\paragraph{Description}

You need to use a simulation plan object to set up the following types
of more complex simulations or forecats:

\paragraph{simulations or forecasts with some of the model variables
temporarily
exogenised;}

\paragraph{simulations with some of the non-linear equations solved
exactly.}

\paragraph{forecasts conditioned upon some
variables;}

The plan object is passed to the \href{model/simulate}{simulate} or
\href{model/jforecast}{\texttt{jforecast}} functions through the option
\texttt{'plan='}.

\paragraph{Example}


