

    \filetitle{get}{Query to plan object}{plan/get}

	\paragraph{Syntax}

\begin{verbatim}
Ans = get(P,Query)
[Ans,Ans,...] = get(P,Query,Query,...)
\end{verbatim}

\paragraph{Input arguments}

\begin{itemize}
\item
  \texttt{P} {[} plan {]} - Simulation plan object.
\item
  \texttt{Query} {[} char {]} - Name of the queried property.
\end{itemize}

\paragraph{Output arguments}

\begin{itemize}
\itemsep1pt\parskip0pt\parsep0pt
\item
  \texttt{Ans} {[} \ldots{} {]} - Answer.
\end{itemize}

\paragraph{Valid queries to plan
objects}

\begin{itemize}
\item
  \texttt{'endogenised='} -- Returns {[} struct {]} a database with time
  series for each shock with 1 in each period where the variable is
  endogenised, and 0 in each period where the variable is not
  endogenised.
\item
  \texttt{'exogenised='} -- Returns {[} struct {]} a database with time
  series for each measurement and transition variable with 1 in each
  period where the variable is exogenised, and 0 in each period where
  the variable is not exogenised.
\item
  \texttt{'onlyEndogenised='} -- Returns {[} struct {]} the same
  database as \texttt{'endogenised='} but including only those shocks
  that are endogenised at least in one period.
\item
  \texttt{'onlyExogenised='} -- Returns {[} struct {]} the same database
  as \texttt{'exogenised='} but including only those measurement and
  transition variables that are endogenised at least in one period.
\item
  \texttt{'range='} -- Returns {[} numeric {]} the simulation plan
  range.
\end{itemize}

\paragraph{Description}

\paragraph{Example}


