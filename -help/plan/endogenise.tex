

    \filetitle{endogenise}{Endogenise shocks or re-endogenise variables at the specified dates}{plan/endogenise}

	\paragraph{Syntax}\label{syntax}

\begin{verbatim}
P = endogenise(P,List,Dates)
P = endogenise(P,Dates,List)
P = endogenise(P,List,Dates,Sigma)
P = endogenise(P,Dates,List,Sigma)
\end{verbatim}

\paragraph{Input arguments}\label{input-arguments}

\begin{itemize}
\item
  \texttt{P} {[} plan {]} - Simulation plan.
\item
  \texttt{List} {[} cellstr \textbar{} char {]} - List of shocks that
  will be endogenised, or list of variables that will be re-endogenise.
\item
  \texttt{Dates} {[} numeric {]} - Dates at which the shocks or
  variables will be endogenised.
\item
  \texttt{Sigma} {[} numeric {]} - Select the anticipation mode, and
  assign a weight to the shock in the case of underdetermined simulation
  plans; if omitted, \texttt{Sigma = 1}.
\end{itemize}

\paragraph{Output arguments}\label{output-arguments}

\begin{itemize}
\itemsep1pt\parskip0pt\parsep0pt
\item
  \texttt{P} {[} plan {]} - Simulation plan with new information on
  endogenised shocks included.
\end{itemize}

\paragraph{Description}\label{description}

\paragraph{Example}\label{example}


