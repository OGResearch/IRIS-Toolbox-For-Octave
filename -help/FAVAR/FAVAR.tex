

    \filetitle{FAVAR}{Create new, empty FAVAR object}{FAVAR/FAVAR}

	\paragraph{Syntax}\label{syntax}

\begin{verbatim}
F = FAVAR(YNames)
\end{verbatim}

\paragraph{Input arguments}\label{input-arguments}

\begin{itemize}
\itemsep1pt\parskip0pt\parsep0pt
\item
  \texttt{YNames} {[} cellstr \textbar{} char {]} - Names of observed
  variables in the FAVAR model.
\end{itemize}

\paragraph{Output arguments}\label{output-arguments}

\begin{itemize}
\itemsep1pt\parskip0pt\parsep0pt
\item
  \texttt{F} {[} FAVAR {]} - New FAVAR object.
\end{itemize}

\paragraph{Description}\label{description}

This function creates a new empty FAVAR object. It is usually followed
by the \href{FAVAR/estimate}{estimate} function to estimate the FAVAR
parameters on data.

\paragraph{Example}\label{example}

To estimate a FAVAR, you first need to create an empty VAR object, and
then run the \href{FAVAR/estimate}{FAVAR} function on it, e.g.

\begin{verbatim}
list = {'DLCPI','DLGDP','R'};
f = FAVAR(list);
f = estimate(f,d,range);
\end{verbatim}


