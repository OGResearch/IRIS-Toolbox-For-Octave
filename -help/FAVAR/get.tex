

    \filetitle{get}{Query model object properties}{FAVAR/get}

	\paragraph{Syntax}
 
 \begin{verbatim}
 Ans = get(A,Query)
 [Ans,Ans,...] = get(A,Query,Query,...)
 \end{verbatim}
 
 \paragraph{Input arguments}
 
 \begin{itemize}
 \item
   \texttt{A} {[} FAVAR {]} - FAVAR object.
 \item
   \texttt{Query} {[} char {]} - Query to the FAVAR object.
 \end{itemize}
 
 \paragraph{Output arguments}
 
 \begin{itemize}
 \item
   \texttt{Ans} {[} \ldots{} {]} - Answer to the query.
 \end{itemize}
 
 \paragraph{Valid queries to FAVAR objects}
 
 \subparagraph{System matrices}
 
 \begin{itemize}
 \item
   \texttt{'A*'} Returns {[} numeric {]} the transition matrix of the
   underlying VAR system on factors.
 \item
   \texttt{'B'} Returns {[} numeric {]} tne matrix mapping the impact of
   structural residuals on the factors in the underlying VAR.
 \item
   \texttt{'C'} Returns {[} numeric {]} the matrix mapping the factors
   into the observables.
 \item
   \texttt{'Omega'} Returns {[} numeric {]} the reduced-form covariance
   matrix of the residuals in the underlying VAR.
 \item
   \texttt{'Sigma'} Returns {[} numeric {]} the covariance matrix of
   idiosyncratic shocks.
 \end{itemize}
 
 \subparagraph{Underlying VAR}
 
 \begin{itemize}
 \item
   \texttt{'VAR'} Returns {[} VAR {]} a VAR object describing the factor
   dynamics.
 \end{itemize}
 
 \subparagraph{Eigenvalues and singular values}
 
 \begin{itemize}
 \item
   \texttt{'eig'} Returns {[} numeric {]} the vector of eigenvalues of
   the underlying VAR.
 \item
   \texttt{'sing'} Returns {[} numeric {]} the vector of singular values
   from the principal component estimation step.
 \end{itemize}
 
 \subparagraph{Observables and factors}
 
 \begin{itemize}
 \item
   \texttt{'mean'} Returns {[} numeric {]} the estimated mean of the
   observables used to standardise the input data.
 \item
   \texttt{'std'} Returns {[} numeric {]} the estimated std deviations of
   the observables used to standardise the input data.
 \item
   \texttt{'ny'} Returns {[} numeric {]} the number of observables.
 \item
   \texttt{'nx'} Returns {[} numeric {]} the number of factors.
 \item
   \texttt{'yList'} Returns {[} cellstr {]} the list of the names of
   observables.
 \end{itemize}
 
 \paragraph{Description}
 
 \paragraph{Example}


