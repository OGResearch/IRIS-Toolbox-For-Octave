

    \filetitle{dbextend}{Combine tseries observations from two or more databases}{dbase/dbextend}

	\paragraph{Syntax}
 
 \begin{verbatim}
 D = dbextend(D,D1,D2,...)
 \end{verbatim}
 
 \paragraph{Input arguments}
 
 \begin{itemize}
 \item
   \texttt{D} {[} struct {]} - Primary input database.
 \item
   \texttt{D1}, \texttt{D2}, \ldots{} {[} struct {]} - Databases whose
   tseries observations will be used to extend or overwrite observations
   in the tseries objects of the same name in the primary database.
 \end{itemize}
 
 \paragraph{Output arguments}
 
 \begin{itemize}
 \item
   \texttt{D} {[} struct {]} - Output database.
 \end{itemize}
 
 \paragraph{Description}
 
 If more than two databases are combined then they are processed
 one-by-one: the first is combined with the second, then the result is
 combined with the third, and so on, using the following rules:
 
 \begin{itemize}
 \item
   If two non-empty tseries objects with the same frequency are combined,
   the observations are spliced together. If some of the observations
   overlap the observations from the second tseries are used.
 \item
   If two empty tseries objects are combined the first is used.
 \item
   If a non-empty tseries is combined with an empty tseries, the
   non-empty one is used.
 \item
   If two objects are combined of which at least one is a non-tseries
   object, the second input object is used.
 \end{itemize}
 
 \paragraph{Example}


