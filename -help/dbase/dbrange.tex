

    \filetitle{dbrange}{Find a range that encompasses the ranges of the listed tseries objects}{dbase/dbrange}

	\paragraph{Syntax}\label{syntax}

\begin{verbatim}
[Range,FreqList] = dbrange(D)
[Range,FreqList] = dbrange(D,List,...)
[Range,FreqList] = dbrange(D,Inf,...)
\end{verbatim}

\paragraph{Input arguments}\label{input-arguments}

\begin{itemize}
\item
  \texttt{D} {[} struct {]} - Input database.
\item
  \texttt{List} {[} char \textbar{} cellstr \textbar{} \texttt{Inf} {]}
  - List of tseries objects that will be included in the range search;
  \texttt{Inf} means all tseries objects existing in the input databases
  will be included.
\end{itemize}

\paragraph{Output arguments}\label{output-arguments}

\begin{itemize}
\item
  \texttt{Range} {[} numeric \textbar{} cell {]} - Range that
  encompasses the observations of the tseries objects in the input
  database; if tseries objects with different frequencies exist, the
  ranges are returned in a cell array.
\item
  \texttt{FreqList} {[} numeric {]} - Vector of date frequencies
  coresponding to the returned ranges.
\end{itemize}

\paragraph{Options}\label{options}

\begin{itemize}
\item
  \texttt{'startDate='} {[} \emph{\texttt{'maxRange'}} \textbar{}
  \texttt{'minRange'} {]} - \texttt{'maxRange'} means the \texttt{range}
  will start at the earliest start date of all tseries included in the
  search; \texttt{'minRange'} means the \texttt{range} will start at the
  latest start date found.
\item
  \texttt{'endDate='} {[} \emph{\texttt{'maxRange'}} \textbar{}
  \texttt{'minRange'} {]} - \texttt{'maxRange'} means the \texttt{range}
  will end at the latest end date of all tseries included in the search;
  \texttt{'minRange'} means the \texttt{range} will end at the earliest
  end date.
\end{itemize}

\paragraph{Description}\label{description}

\paragraph{Example}\label{example}


