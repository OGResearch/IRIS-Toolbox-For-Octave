

    \filetitle{xls2csv}{Convert XLS file to CSV file}{dbase/xls2csv}

	\paragraph{Syntax}
 
 \begin{verbatim}
 xls2csv(InpFile)
 xls2csv(InpFile,OutpFile,...)
 \end{verbatim}
 
 \paragraph{Input arguments}
 
 \begin{itemize}
 \item
   \texttt{InpFile} {[} char {]} - Name of an XLS input file that will be
   converted to CSV.
 \item
   \texttt{OutpFile} {[} empty \textbar{} char {]} - Name of the CSV
   output file; if not supplied or empty, the CSV file name will be
   derived from the XLS input file name.
 \end{itemize}
 
 \paragraph{Options}
 
 \begin{itemize}
 \item
   \texttt{'sheet='} {[} numeric \textbar{} char \textbar{}
   \emph{\texttt{1}} {]} - Worksheet in the XLS file that will be saved;
   can be either the sheet number or the sheet name.
 \end{itemize}
 
 \paragraph{Description}
 
 This function calls a third-party JavaScript (courtesy of Christopher
 West). The script uses an MS Excel application on the background, and
 hence MS Excel must be installed on the computer.
 
 Only one worksheet at a time can be saved to CSV. By default, it is the
 first worksheet found in the input XLS file; use the option
 \texttt{'sheet='} to control which worksheet will be saved.
 
 See also \$irisroot/+thirdparty/xls2csv.js for copyright information.
 
 \paragraph{Example 1}
 
 Save the first worksheets of the following XLS files to CSV files.
 
 \begin{verbatim}
 xls2csv('myDataFile.xls');
 xls2csv('C:\Data\myDataFile.xls');
 \end{verbatim}
 
 \paragraph{Example 2}
 
 Save the worksheet named `Sheet3' to a CSV file; the name of the CSV
 file will be \texttt{'myDataFile.csv'}.
 
 \begin{verbatim}
 xls2csv('myDataFile.xls',[],'sheet=','Sheet3');
 \end{verbatim}
 
 \paragraph{Example 3}
 
 Save the second worksheet to a CSV file under the name
 \texttt{'myDataFile\_2.csv'}.
 
 \begin{verbatim}
 xls2csv('myDataFile.xls','myDataFile_2.csv,'sheet=',2);
 \end{verbatim}


