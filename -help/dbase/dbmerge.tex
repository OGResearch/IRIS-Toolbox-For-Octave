

    \filetitle{dbmerge}{Merge two or more databases}{dbase/dbmerge}

	\paragraph{Syntax}\label{syntax}

\begin{verbatim}
D = dbmerge(D1,D2,...)
\end{verbatim}

\paragraph{Input arguments}\label{input-arguments}

\begin{itemize}
\itemsep1pt\parskip0pt\parsep0pt
\item
  \texttt{D1}, \texttt{D2}, \ldots{} {[} struct {]} - Input databases
  whose entries will be combined in the output datase.
\end{itemize}

\paragraph{Output arguments}\label{output-arguments}

\begin{itemize}
\itemsep1pt\parskip0pt\parsep0pt
\item
  \texttt{D} {[} struct {]} - Output database that combines entries from
  all input database; if some entries are found in more than one input
  databases, the last occurence is used.
\end{itemize}

\paragraph{Description}\label{description}

\paragraph{Example}\label{example}

\begin{verbatim}
d1 = struct('a',1,'b',2);
d2 = struct('a',10,'c',20);
d = dbmerge(d1,d2)
d =
   a: 10
   b: 2
   c: 20
\end{verbatim}


