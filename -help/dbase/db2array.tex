

    \filetitle{db2array}{Convert tseries database entries to numeric array}{dbase/db2array}

	\paragraph{Syntax}\label{syntax}

\begin{verbatim}
[X,Incl,Range] = db2array(D)
[X,Incl,Range] = db2array(D,List)
[X,Incl,Range] = db2array(D,List,Range)
\end{verbatim}

\paragraph{Input arguments}\label{input-arguments}

\begin{itemize}
\item
  \texttt{D} {[} struct {]} - Input database with tseries objects that
  will be converted to a numeric array.
\item
  \texttt{List} {[} char \textbar{} cellstr {]} - List of tseries names
  that will be converted to a numeric array; if not specified, all
  tseries entries found in the input database, \texttt{D}, will be
  included in the output arrays, \texttt{X}.
\item
  \texttt{Range} {[} numeric \textbar{} \texttt{Inf} {]} - Date range;
  \texttt{Inf} means a range from the very first non-NaN observation to
  the very last non-NaN observation.
\end{itemize}

\paragraph{Output arguments}\label{output-arguments}

\begin{itemize}
\item
  \texttt{X} {[} numeric {]} - Numeric array with observations from
  individual tseries objects in columns.
\item
  \texttt{Incl} {[} cellstr {]} - List of tseries names that have been
  actually found in the database.
\item
  \texttt{Range} {[} numeric {]} - Date range actually used; this output
  argument is useful when the input argument \texttt{Range} is missing
  or \texttt{Inf}.
\end{itemize}

\paragraph{Description}\label{description}

The output array, \texttt{X}, is always NPer-by-NList-by-NAlt, where
NPer is the length of the \texttt{Range} (the number of periods), NList
is the number of tseries included in the \texttt{List}, and NAlt is the
maximum number of columns that any of the tseries included in the
\texttt{List} have.

All tseries with more than one dimension (i.e.~with more than one
column) are always expanded along 3rd dimension only. For instance, a
10-by-2-by-3 tseries will occupy a 10-by-1-by-6 space in \texttt{X} at
its respective location.

\paragraph{Example}\label{example}


