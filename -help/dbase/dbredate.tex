

    \filetitle{dbredate}{Redate all tseries objects in a database}{dbase/dbredate}

	\paragraph{Syntax}
 
 \begin{verbatim}
 D = redate(D,OldDate,NewDate)
 \end{verbatim}
 
 \paragraph{Input arguments}
 
 \begin{itemize}
 \item
   \texttt{D} {[} struct {]} - Input database with tseries objects.
 \item
   \texttt{OldDate} {[} numeric {]} - Base date that will be converted to
   a new date in all tseries objects.
 \item
   \texttt{NewDate} {[} numeric {]} - A new date to which the base date
   \texttt{OldDate} will be changed in all tseries objects;
   \texttt{newDate} need not be the same frequency as \texttt{OldDate}.
 \end{itemize}
 
 \paragraph{Output arguments}
 
 \begin{itemize}
 \item
   \texttt{d} {[} struct {]} - Output database where all tseries objects
   have identical data as in the input database, but with their time
   dimension changed.
 \end{itemize}
 
 \paragraph{Description}
 
 \paragraph{Example}


