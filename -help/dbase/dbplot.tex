

    \filetitle{dbplot}{Plot from database}{dbase/dbplot}

	\paragraph{Syntax}\label{syntax}

\begin{verbatim}
[FF,AA,PDb] = dbplot(D,List,Range,...)
[FF,AA,PDb] = dbplot(D,Range,List,...)
\end{verbatim}

\paragraph{Input arguments}\label{input-arguments}

\begin{itemize}
\item
  \texttt{D} {[} struct {]} - Database with input data.
\item
  \texttt{List} {[} cellstr {]} - List of expressions (or labelled
  expressions) that will be evaluated and plotted in separate graphs.
\item
  \texttt{Range} {[} numeric {]} - Date range.
\end{itemize}

\paragraph{Output arguments}\label{output-arguments}

\begin{itemize}
\item
  \texttt{FF} {[} numeric {]} - Handles to figures created by
  \texttt{qplot}.
\item
  \texttt{AA} {[} cell {]} - Handles to axes created by \texttt{qplot}.
\item
  \texttt{PDB} {[} struct {]} - Database with actually plotted series.
\end{itemize}

\paragraph{Options}\label{options}

\begin{itemize}
\itemsep1pt\parskip0pt\parsep0pt
\item
  \texttt{'plotFunc='} {[} @bar \textbar{} @hist \textbar{} *@plot*
  \textbar{} @plotpred \textbar{} @stem {]} - Plot function used to
  create the graphs.
\end{itemize}

See help on \href{qreport/qplot}{\texttt{qreport/qplot}} for other
options available.

\paragraph{Description}\label{description}

The function \texttt{dbplot} opens a new figure window (as many as
needed to accommodate all graphs given the option \texttt{'subplot='}),
and creates a graph for each entry in the cell array \texttt{List}.

\texttt{List} can contain the names of the database time series,
expression referring to the database fields evaluating to time series.
You can also add labels (that will be displayed as graph titles)
enclosed in double quotes and preceding the expressions. If you start
the expression with a \texttt{\^{}} (hat) symbol, the function specified
in the \texttt{'transform='} option will not be applied to that
expression.

\paragraph{Example}\label{example}

\begin{verbatim}
dbplot(d,qq(2010,1):qq(2015,4), ...
   {'x','"Series Y" y','^"Series z"'}, ...
   'transform=',@(x) 100*(x-1));
\end{verbatim}


