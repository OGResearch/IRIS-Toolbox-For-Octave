

    \filetitle{dbsave}{Save database as CSV file}{dbase/dbsave}

	\paragraph{Syntax}
 
 \begin{verbatim}
 List = dbsave(D,FName)
 List = dbsave(D,FName,Dates,...)
 \end{verbatim}
 
 \paragraph{Output arguments}
 
 \begin{itemize}
 \item
   \texttt{List} {[} cellstr {]} - - List of actually saved database
   entries.
 \end{itemize}
 
 \paragraph{Input arguments}
 
 \begin{itemize}
 \item
   \texttt{D} {[} struct {]} - Database whose tseries and numeric entries
   will be saved.
 \item
   \texttt{FName} {[} char {]} - Filename under which the CSV will be
   saved, including its extension.
 \item
   \texttt{Dates} {[} numeric \textbar{} \emph{\texttt{Inf}} {]} Dates or
   date range on which the tseries objects will be saved.
 \end{itemize}
 
 \paragraph{Options}
 
 \begin{itemize}
 \item
   \texttt{'class='} {[} \emph{\texttt{true}} \textbar{} false {]} -
   Include a row with class and size specifications.
 \item
   \texttt{'comment='} {[} \emph{\texttt{true}} \textbar{} \texttt{false}
   {]} - Include a row with comments for tseries objects.
 \item
   \texttt{'decimal='} {[} numeric \textbar{} \emph{empty} {]} - Number
   of decimals up to which the data will be saved; if empty the
   \texttt{'format'} option is used.
 \item
   \texttt{'format='} {[} char \textbar{} \emph{\texttt{'\%.8e'}} {]} -
   Numeric format that will be used to represent the data, see
   \texttt{sprintf} for details on formatting, The format must start with
   a \texttt{'\%'}, and must not include identifiers specifying order of
   processing, i.e.~the \texttt{'\$'} signs, or left-justify flags, the
   \texttt{'-'} signs.
 \item
   \texttt{'freqLetters='} {[} char \textbar{} \emph{\texttt{'YHQBM'}}
   {]} - Five letters to represent the five possible date frequencies
   (annual, semi-annual, quarterly, bimonthly, monthly).
 \item
   \texttt{'nan='} {[} char \textbar{} \emph{\texttt{'NaN'}} {]} - String
   that will be used to represent NaNs.
 \item
   \texttt{'saveSubdb='} {[} \texttt{true} \textbar{}
   \emph{\texttt{false}} {]} - Save sub-databases (structs found within
   the struct \texttt{D}); the sub-databases will be saved to separate
   CSF files.
 \item
   \texttt{'userData='} {[} char \textbar{} \emph{`userdata'} {]} - Field
   name from which any kind of userdata will be read and saved in the CSV
   file.
 \end{itemize}
 
 \paragraph{Description}
 
 The data saved include also imaginary parts of complex numbers.
 
 \subparagraph{Saving user data with the database}
 
 If your database contains field named \texttt{'userdata='}, this will be
 saved in the CSV file on a separate row. The \texttt{'userdata='} field
 can be any combination of numeric, char, and cell arrays and 1-by-1
 structs.
 
 You can use the \texttt{'userdata='} field to describe the database or
 preserve any sort of metadata. To change the name of the field that is
 treated as user data, use the \texttt{'userData='} option.
 
 \paragraph{Example}
 
 Create a simple database with two time series.
 
 \begin{verbatim}
 d = struct();
 d.x = tseries(qq(2010,1):qq(2010,4),@rand);
 d.y = tseries(qq(2010,1):qq(2010,4),@rand);
 \end{verbatim}
 
 Add your own description of the database, e.g.
 
 \begin{verbatim}
 d.userdata = {'My database',datestr(now())};
 \end{verbatim}
 
 Save the database as CSV using \texttt{dbsave},
 
 \begin{verbatim}
 dbsave(d,'mydatabase.csv');
 \end{verbatim}
 
 When you later load the database,
 
 \begin{verbatim}
 d = dbload('mydatabase.csv')
 
 d = 
 
    userdata: {'My database'  '23-Sep-2011 14:10:17'}
           x: [4x1 tseries]
           y: [4x1 tseries]
 \end{verbatim}
 
 the database will preserve the \texttt{'userdata='} field.
 
 \paragraph{Example}
 
 To change the field name under which you store your own user data, use
 the \texttt{'userdata='} option when running \texttt{dbsave},
 
 \begin{verbatim}
 d = struct();
 d.x = tseries(qq(2010,1):qq(2010,4),@rand);
 d.y = tseries(qq(2010,1):qq(2010,4),@rand);
 d.MYUSERDATA = {'My database',datestr(now())};
 dbsave(d,'mydatabase.csv',Inf,'userData=','MYUSERDATA');
 \end{verbatim}
 
 The name of the user data field is also kept in the CSV file so that
 \texttt{dbload} works fine in this case, too, and returns a database
 identical to the saved one,
 
 \begin{verbatim}
 d = dbload('mydatabase.csv')
 
 d = 
 
    MYUSERDATA: {'My database'  '23-Sep-2011 14:10:17'}
             x: [4x1 tseries]
             y: [4x1 tseries]
 \end{verbatim}


