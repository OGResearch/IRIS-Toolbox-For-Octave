

    \filetitle{dbminuscontrol}{Create simulation-minus-control database}{dbase/dbminuscontrol}

	\paragraph{Syntax}\label{syntax}

\begin{verbatim}
[D,C] = dbminuscontrol(M,D)
[D,C] = dbminuscontrol(M,D,C)
\end{verbatim}

\paragraph{Input arguments}\label{input-arguments}

\begin{itemize}
\item
  \texttt{M} {[} model {]} - Model object on which the databases
  \texttt{D} and \texttt{C} are based.
\item
  \texttt{D} {[} struct {]} - Simulation database.
\item
  \texttt{C} {[} struct {]} - Control database; if the input argument
  \texttt{C} is not specified, the steady-state database of the model
  \texttt{M} is used for the control database.
\end{itemize}

\paragraph{Output arguments}\label{output-arguments}

\begin{itemize}
\item
  \texttt{D} {[} struct {]} - Simulation-minus-control database, in
  which all log-variables are \texttt{d.x/c.x}, and all other variables
  are \texttt{d.x-c.x}.
\item
  \texttt{C} {[} struct {]} - Control database.
\end{itemize}

\paragraph{Description}\label{description}

\paragraph{Example}\label{example}

We run a shock simulation in full levels using a steady-state (or
balanced-growth-path) database as input, and then compute the deviations
from the steady state.

\begin{verbatim}
d = sstatedb(m,1:40);
... % Set up a shock or shocks here.
s = simulate(m,d,1:40);
s = dbextend(d,s);
s = dbminuscontrol(m,s,d);
\end{verbatim}

Note that this is equivalent to running

\begin{verbatim}
d = zerodb(m,1:40);
... % Set up a shock or shocks here.
s = simulate(m,d,1:40,'deviation',true);
s = dbextend(d,s);
\end{verbatim}


