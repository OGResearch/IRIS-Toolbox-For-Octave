

    \filetitle{dbclip}{Clip all tseries entries in a database down to the specified date range}{dbase/dbclip}

	\paragraph{Syntax}
 
 \begin{verbatim}
 D = dbclip(D,Range)
 \end{verbatim}
 
 \paragraph{Input arguments}
 
 \begin{itemize}
 \item
   \texttt{D} {[} struct {]} - Database or nested databases with tseries
   objects.
 \item
   \texttt{Range} {[} numeric \textbar{} cell {]} - Range or a cell array
   of ranges to which all tseries objects will be clipped; multiple
   ranges can be specified, each for a different date
   frequency/periodicity.
 \end{itemize}
 
 \paragraph{Output arguments}
 
 \begin{itemize}
 \item
   \texttt{D} {[} struct {]} - Database with tseries objects cut down to
   \texttt{range}.
 \end{itemize}
 
 \paragraph{Description}
 
 This functions looks up all tseries objects within the database
 \texttt{d}, including tseries objects nested in sub-databases, and cuts
 off any values preceding the start date of \texttt{Range} or following
 the end date of \texttt{range}. The tseries object comments, if any, are
 preserved in the new database.
 
 If a tseries entry does not match the date frequency of the input range,
 a warning is thrown.
 
 Multiple ranges can be specified in \texttt{Range} (as a cell array),
 each for a different date frequency/periodicity (i.e.~one or more of the
 following: monthly, bi-monthly, quarterly, half-yearly, yearly,
 indeterminate). Each tseries entry will be clipped to the range that
 matches its date frequency.
 
 \paragraph{Example}
 
 \begin{verbatim}
 d = struct();
 d.x = tseries(qq(2005,1):qq(2010,4),@rand);
 d.y = tseries(qq(2005,1):qq(2010,4),@rand)
 
 d =
    x: [24x1 tseries]
    y: [24x1 tseries]
 
 dbclip(d,qq(2007,1):qq(2007,4))
 
 ans =
     x: [4x1 tseries]
     y: [4x1 tseries]
 \end{verbatim}


