

    \filetitle{dbload}{Create database by loading CSV file}{dbase/dbload}

	\paragraph{Syntax}\label{syntax}

\begin{verbatim}
D = dbload(FName, ...)
D = dbload(D,FName, ...)
\end{verbatim}

\paragraph{Input arguments}\label{input-arguments}

\begin{itemize}
\item
  \texttt{FName} {[} char \textbar{} cellstr {]} - Name of the Input CSV
  data file or a cell array of CSV file names that will be combined.
\item
  \texttt{D} {[} struct {]} - An existing database (struct) to which the
  new entries from the input CSV data file entries will be added.
\end{itemize}

\paragraph{Output arguments}\label{output-arguments}

\begin{itemize}
\itemsep1pt\parskip0pt\parsep0pt
\item
  \texttt{D} {[} struct {]} - Database created from the input CSV
  file(s).
\end{itemize}

\paragraph{Options}\label{options}

\begin{itemize}
\item
  \texttt{'case='} {[} \texttt{'lower'} \textbar{} \texttt{'upper'}
  \textbar{} \emph{empty} {]} - Change case of variable names.
\item
  \texttt{'commentRow='} {[} char \textbar{} cellstr \textbar{}
  \emph{\texttt{\{'comment','comments'\}}} {]} - Label at the start of
  row that will be used to create tseries object comments.
\item
  \texttt{'convert='} {[} numeric \textbar{} cellstr \textbar{}
  \emph{empty} {]} - If non-empty, frequency conversion will be run on
  all time series loaded; specify the target frequency (numeric) or a
  cell array of input arguments and options in a call to the function
  \texttt{convert}.
\item
  \texttt{'dateFormat='} {[} char \textbar{} \emph{\texttt{'YYYYFP'}}
  {]} - Format of dates in first column.
\item
  \texttt{'delimiter='} {[} char \textbar{} \emph{\texttt{','}} {]} -
  Delimiter separating the individual values (cells) in the CSV file; if
  different from a comma, all occurences of the delimiter will replaced
  with commas -- note that this will also affect text in comments.
\item
  \texttt{'firstDateOnly='} {[} \texttt{true} \textbar{}
  \emph{\texttt{false}} {]} - Read and parse only the first date string,
  and fill in the remaining dates assuming a range of consecutive dates.
\item
  \texttt{'freq='} {[} \texttt{0} \textbar{} \texttt{1} \textbar{}
  \texttt{2} \textbar{} \texttt{4} \textbar{} \texttt{6} \textbar{}
  \texttt{12} \textbar{} \texttt{365} \textbar{} \texttt{'daily'}
  \textbar{} \emph{empty} {]} - Advise frequency of dates; if empty,
  frequency will be automatically recognised.
\item
  \texttt{'freqLetters='} {[} char \textbar{} \emph{\texttt{'YHQBM'}}
  {]} - Letters representing frequency of dates in date column.
\item
  \texttt{'inputFormat='} {[} \emph{\texttt{'auto'}} \textbar{}
  \texttt{'csv'} \textbar{} \texttt{'xls'} {]} - Format of input data
  file; \texttt{'auto'} means the format will be determined by the file
  extension.
\item
  \texttt{'nameRow='} {[} char \textbar{} numeric \textbar{}
  \emph{\texttt{\{'','Variables'\}}} {]} - String, or cell array of
  possible strings, that is found at the beginning (in the first cell)
  of the row with variable names, or the line number at which the row
  with variable names appears (first row is numbered 1).
\item
  \texttt{'nameFunc='} {[} cell \textbar{} function\_handle \textbar{}
  \emph{empty} {]} - Function used to change or transform the variable
  names. If a cell array of function handles, each function will be
  applied in the given order.
\item
  \texttt{'nan='} {[} char \textbar{} \emph{\texttt{NaN}} {]} - String
  representing missing observations (case insensitive).
\item
  \texttt{'preProcess='} {[} function\_handle \textbar{} cell \textbar{}
  empty {]} - Apply this function, or cell array of functions, to the
  raw text file before parsing the data.
\item
  \texttt{'skipRows='} {[} char \textbar{} cellstr \textbar{} numeric
  \textbar{} \emph{empty} {]} - Skip rows whose first cell matches the
  string or strings (regular expressions); or, skip a vector of row
  numbers.
\item
  \texttt{'userData='} {[} char \textbar{} \emph{\texttt{Inf}} {]} -
  Field name under which the database userdata loaded from the CSV file
  (if they exist) will be stored in the output database; \texttt{Inf}
  means the field name will be read from the CSV file (and will be thus
  identical to the originally saved database).
\item
  \texttt{'userDataField='} {[} char \textbar{} \emph{\texttt{'.'}} {]}
  - A leading character denoting userdata fields for individual time
  series; if empty, no userdata fields will be read in and created.
\item
  \texttt{'userDataFieldList='} {[} cellstr \textbar{} numeric
  \textbar{} empty {]} - List of row headers, or vector of row numbers,
  that will be included as user data in each time series.
\end{itemize}

\paragraph{Description}\label{description}

Use the \texttt{'freq='} option whenever there is ambiguity in
intepreting the date strings, and IRIS is not able to determine the
frequency correctly (see Example).

\subparagraph{Structure of CSV database
files}\label{structure-of-csv-database-files}

The minimalist structure of a CSV database file has a leading row with
variables names, a leading column with dates in the basic IRIS format,
and individual columns with numeric data:

\begin{verbatim}
+---------+---------+---------+--
|         |       Y |       P |
+---------+---------+---------+--
|  2010Q1 |       1 |      10 |
+---------+---------+---------+--
|  2010Q2 |       2 |      20 |
+---------+---------+---------+--
|         |         |         |
\end{verbatim}

You can add a comment row (must be placed before the data part, and
start with a label `Comment' in the first cell) that will also be read
in and assigned as comments to the individual tseries objects created in
the output database.

\begin{verbatim}
+---------+---------+---------+--
|         |       Y |       P |
+---------+---------+---------+--
| Comment |  Output |  Prices |
+---------+---------+---------+--
|  2010Q1 |       1 |      10 |
+---------+---------+---------+--
|  2010Q2 |       2 |      20 |
+---------+---------+---------+--
|         |         |         |
\end{verbatim}

You can use a different label in the first cell to denote a comment row;
in that case you need to set the option \texttt{'commentRow='}
accordingly.

All CSV rows whose names start with a character specified in the option
\texttt{'userdataField='} (a dot by default) will be added to output
tseries objects as fields of their userdata.

\begin{verbatim}
+---------+---------+---------+--
|         |       Y |       P |
+---------+---------+---------+--
| Comment |  Output |  Prices |
+---------+---------+---------+--
| .Source |   Stat  |  IMFIFS |
+---------+---------+---------+--
| .Update | 17Feb11 | 01Feb11 |
+---------+---------+---------+--
| .Units  | Bil USD |  2010=1 |
+---------+---------+---------+--
|  2010Q1 |       1 |      10 |
+---------+---------+---------+--
|  2010Q2 |       2 |      20 |
+---------+---------+---------+--
|         |         |         |
\end{verbatim}

\paragraph{Example}\label{example}

Typical example of using the \texttt{'freq='} option is a quarterly
database with dates represented by the corresponding months, such as a
sequence 2000-01-01, 2000-04-01, 2000-07-01, 2000-10-01, etc. In this
case, you can use the following options:

\begin{verbatim}
d = dbload('filename.csv','dateFormat','YYYY-MM-01','freq',4);
\end{verbatim}


