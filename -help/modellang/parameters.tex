

    \filetitle{!parameters}{List of parameters}{modellang/parameters}

	\paragraph{Syntax}
 
 \begin{verbatim}
 !parameters
     parameter_name, parameter_name, ...
     ...
 \end{verbatim}
 
 \paragraph{Syntax with descriptors}
 
 \begin{verbatim}
 !parameters
     parameter_name, parameter_name, ...
     'Description of the parameter...' parameter_name
 \end{verbatim}
 
 \paragraph{Syntax with steady-state values}
 
 \begin{verbatim}
 !parameters
     parameter_name, parameter_name, ...
     parameter_name = value
 \end{verbatim}
 
 \paragraph{Description}
 
 The \texttt{!parameters} keyword starts a new declaration block for
 parameters; the names of the parameters must be separated by commas,
 semi-colons, or line breaks. You can have as many declaration blocks as
 you wish in any order in your model file: They all get combined together
 when you read the model file in. Each parameters must be declared
 (exactly once).
 
 You can add descriptors to the parameters (enclosed in single or double
 quotes, preceding the name of the parameter); these will be stored in,
 and accessible from, the model object. You can also assign parameter
 values straight in the model file (following an equal sign after the
 name of the parameter); this is, though, rather rare and unnecessary
 practice because you can assign and change parameter values more
 conveniently in the model object.
 
 \paragraph{Example}
 
 \begin{verbatim}
 !parameters
     alpha, 'Discount factor' beta
     'Labour share' gamma = 0.60
 \end{verbatim}


