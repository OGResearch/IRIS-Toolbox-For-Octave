

    \filetitle{=\#}{Mark an equation for exact non-linear simulation}{modellang/exactnonlin}

	\paragraph{Syntax}\label{syntax}

\begin{verbatim}
LHS =# RHS;
\end{verbatim}

\paragraph{Description}\label{description}

Equations that have the equal sign marked with an \texttt{\#} can be
simulated in an exact non-linear mode.

Why is it the channels sign, \texttt{\#}, that is used to mark the
equations for exact non-linear simulations? Because if you associate
your model file extension with the Matlab editor, the channel signs are
displayed red making it easier to spot them.


