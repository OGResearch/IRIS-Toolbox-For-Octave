

    \filetitle{!!}{Steady-state version of an equation}{modellang/sstateversion}

	\paragraph{Syntax}

\begin{verbatim}
Equation !! Steady_state_equation;
\end{verbatim}

\paragraph{Description}

For each transition or measurement equation, you can provide a separate
steady-state version of it. The steady-state version is used when you
run the \href{model/sstate}{\texttt{sstate}} function. This is useful
when you can substantially simplify some parts of the dynamic equations,
and help therefore the numerical solver to achieve faster and more
accurate results.

Why is a double exclamation point, \texttt{!!}, used to start the
steady-state versions of equations? Because if you associate your model
file extension with the Matlab editor, anything after an exclamation
point is displayed red making it easier to spot the steady-state
equations.

\paragraph{Example}

The following steady state version will be, of course, valid only in
stationary models where we can safely remove lags and leads.

\begin{verbatim}
Lambda = Lambda{1}*(1+r)*beta !! r = 1/beta - 1;
\end{verbatim}

\paragraph{Example}

\begin{verbatim}
log(A) = log(A{-1}) + epsilon_a !! A = 1;
\end{verbatim}


