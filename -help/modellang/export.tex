

    \filetitle{!export}{Create a carry-around file to be saved on the disk}{modellang/export}

	\paragraph{Syntax}\label{syntax}

\begin{verbatim}
!export(FILENAME)
    FILE_CONTENTS
!end
\end{verbatim}

\paragraph{Description}\label{description}

You can include in the model file the contents of files you need or want
to carry around together with the model; a typical example is your own
m-file functions used in the model equations.

The file or files are created and save under the name specified in the
\texttt{!export} keyword at the time you load the model using the
function \href{model/model}{\texttt{model}}. The contents of the export
files is are also stored in the model objects. You can manually
re-create and re-save the files by running the function
\href{model/export}{\texttt{export}}.

Note that if no filename is provided or \texttt{FILENAME} is empty, the
corresponding \texttt{!export} block is discarded without an error or
warning.


