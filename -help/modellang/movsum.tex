

    \filetitle{movsum}{Moving sum pseudofunction}{modellang/movsum}

	\paragraph{Syntax}

\begin{verbatim}
movsum(EXPR)
movsum(EXPR,K)
\end{verbatim}

\paragraph{Description}

If the second input argument, \texttt{K}, is negative, this function
expands to the moving sum of the last K periods (including the current
period), i.e.

\begin{verbatim}
((EXPR)+(EXPR{-1})+ ... +(EXPR{-(K-1)})
\end{verbatim}

where \texttt{EXPR\{-N\}} derives from \texttt{EXPR} and has all its
time subscripts shifted by \texttt{-N} (if specified).

If the second input argument, \texttt{K}, is positive, this function
expands to the moving sum of the next K periods ahead (including the
current period), i.e.

\begin{verbatim}
((EXPR)+(EXPR{1})+ ... +(EXPR{K-1})
\end{verbatim}

If the second input argument, \texttt{K}, is not specified, the default
value -4 is used (based on the fact that most of the macroeconomic
models are quarterly).

\paragraph{Example}

The following three lines

\begin{verbatim}
movsum(Z)
movsum(Z,-3)
movsum(X+Y{-1},2)
\end{verbatim}

will expand to

\begin{verbatim}
((Z)+(Z{-1})+(Z{-2})+(Z{-3}))
((Z)+(Z{-1})+(Z{-2}))
((X+Y{-1})+(X{1}+Y))
\end{verbatim}


