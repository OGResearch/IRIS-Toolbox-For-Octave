

    \filetitle{\dollar[...]\dollar}{Pseudosubstitutions}{modellang/pseudosubs}

	\paragraph{Syntax}

\begin{verbatim}
$[Expr]$
\end{verbatim}

\paragraph{Description}

The expression \texttt{Expr} enclosed within \texttt{\${[}...{]}\$} is
evaluated as a Matlab expression, and converted to a character string.
The expression may refer to parameters passed into the function
\href{model}{\texttt{model/model}}, or to
\href{modellang/for}{\texttt{!for}} loop control variable names. The
expression must evaluate to a scalar number, a logical scalar, or
character string.

\paragraph{Example}

The following line of code

\begin{verbatim}
pie{$[K]$}
\end{verbatim}

which is assumed to be part of a model file named \texttt{my.model},
will expand to

\begin{verbatim}
pie{3}
\end{verbatim}

in either of the following two calls to the function \texttt{model}:

\begin{verbatim}
model('my.model','K=',3);

P = struct();
P.K = 3;
model('my.model','assign=',P);
\end{verbatim}

\paragraph{Example}

The following \href{modellang/for}{\texttt{!for}} loop

\begin{verbatim}
!for
    [ 2 : 4 ]
!do
    x? = x$[?-1]${-1};
!end
\end{verbatim}

will expand to

\begin{verbatim}
x2 = x1{-1};
x3 = x2{-1};
x4 = x3{-1};
\end{verbatim}


