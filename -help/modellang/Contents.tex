
    \foldertitle{modellang}{Model file language}{modellang/Contents}

	Model file language is used to write model files. The model files are
 plain text files (saved under any filename with any extension) that
 describes the model: its equations, variables, parameters, etc. The
 model file, on the other hand, does not describe what to do with the
 model. To run the tasks you want to perform with the model, you need
 first to load the model file into Matlab using the
 \href{model/model}{\texttt{model}} function. This function creates a
 model object. Then you write your own m-files using Matlab and IRIS
 functions to perform the desired tasks with the model object.
 
 Why do all the keywords (except pseudofunctions) start with an
 exclamation point? Why do the comments have the same style as in Matlab?
 Why do substitutions and steady-state references use the dollar sign?
 Because this way, you can get the model files syntax-highlighted in the
 Matlab editor. Syntax highlighting improves enormously the readability
 of the files, and helps understand the model more quickly. See
 \href{setup/Contents}{the setup instructions} for more details.
 
 \paragraph{Variables, parameters, substitutions and functions}
 
 \begin{itemize}
 \item
   \href{modellang/transitionvariables}{\texttt{!transition\_variables}}
   - List of transition variables.
 \item
   \href{modellang/transitionshocks}{\texttt{!transition\_shocks}} - List
   of transition shocks.
 \item
   \href{modellang/measurementvariables}{\texttt{!measurement\_variables}}
   - List of measurement variables.
 \item
   \href{modellang/measurementshocks}{\texttt{!measurement\_shocks}} -
   List of measurement shocks.
 \item
   \href{modellang/exogenousvariables}{\texttt{!exogenous\_variables}} -
   List of exogenous variables.
 \item
   \href{modellang/parameters}{\texttt{!parameters}} - List of
   parameters.
 \item
   \href{modellang/autoexogenise}{\texttt{!autoexogenise}} - Definition
   of variable/shock pairs for use in autoexogenised simulation plans.
 \end{itemize}
 
 \paragraph{Equations}
 
 \begin{itemize}
 \item
   \href{modellang/transitionequations}{\texttt{!transition\_equations}}
   - Block of transition equations.
 \item
   \href{modellang/measurementequations}{\texttt{!measurement\_equations}}
   - Block of measurement equations.
 \item
   \href{modellang/dtrends}{\texttt{!dtrends}} - Block of deterministic
   trend equations.
 \item
   \href{modellang/links}{\texttt{!links}} - Define dynamic links.
 \end{itemize}
 
 \paragraph{Linearised and log-linearised variables}
 
 \begin{itemize}
 \item
   \href{modellang/logvariables}{\texttt{!log\_variables}} - List of
   log-linearised variables.
 \item
   \href{modellang/allbut}{\texttt{!allbut}} - Inverse list of
   log-linearised variables.
 \item
   \href{modellang/regexpression}{\texttt{\textless{}...\textgreater{}}}
   - Regular expression in log-varible list.
 \end{itemize}
 
 \paragraph{Model pseudofunctions}
 
 Pseudofunctions do not start with an exclamation point.
 
 \begin{itemize}
 \item
   \href{modellang/min}{\texttt{min}} - Define the loss function in a
   time-consistent optimal policy model.
 \end{itemize}
 
 \paragraph{Special operators}
 
 \begin{itemize}
 \item
   \href{modellang/sstateversion}{\texttt{!!}} - Steady-state version of
   an equation.
 \item
   \href{modellang/alias}{\texttt{!!}} - Beginning of aliasing in
   descriptions and labels.
 \item
   \href{modellang/ttrend}{\texttt{!ttrend}} - Linear time trend in
   deterministic trend equations.
 \item
   \href{modellang/laglead}{\texttt{\{...\}}} - Lag or lead.
 \item
   \href{modellang/sstateref}{\texttt{\&}} - Reference to the
   steady-state level of a variable.
 \item
   \href{modellang/exactnonlin}{\texttt{=\#}} - Mark an equation for
   exact non-linear simulation.
 \end{itemize}
 
 \paragraph{Preparser pseudofunctions}
 
 Pseudofunctions do not start with an exclamation point.
 
 \begin{itemize}
 \item
   \href{modellang/diff}{\texttt{diff}} - First difference
   pseudofunction.
 \item
   \href{modellang/dot}{\texttt{dot}} - Gross rate of growth
   pseudofunction.
 \item
   \href{modellang/difflog}{\texttt{difflog}} - First log-difference
   pseudofunction.
 \item
   \href{modellang/movavg}{\texttt{movavg}} - Moving average
   pseudofunction.
 \item
   \href{modellang/movprod}{\texttt{movprod}} - Moving product
   pseudofunction.
 \item
   \href{modellang/movsum}{\texttt{movsum}} - Moving sum pseudofunction.
 \end{itemize}
 
 \paragraph{Preparser control commands}
 
 \begin{itemize}
 \item
   \href{modellang/substitutions}{\texttt{!substitutions}} - Define text
   substitutions.
 \item
   \href{modellang/import}{\texttt{!import}} - Include the content of
   another model file.
 \item
   \href{modellang/export}{\texttt{!export}} - Create a carry-around file
   to be saved on the disk.
 \item
   \href{modellang/if}{\texttt{!if...!elseif...!else...!end}} - Choose
   block of code depending on a condition.
 \item
   \href{modellang/switch}{\texttt{!switch...!case...!end}} - Switch
   among several branches of the model code depending on the value of an
   expression.
 \item
   \href{modellang/for}{\texttt{!for...!do...!end}} - For loop for
   automated creation of model code.
 \item
   \href{modellang/linecomments}{\texttt{\%}} - Line comments.
 \item
   \href{modellang/blockcomments}{\texttt{\%\{...\%\}}} - Block comments.
 \end{itemize}
 
 \paragraph{Getting on-line help on model file language}
 
 When getting help on model file language, type the names of the keywords
 and commands without the exclamation point:
 
 \begin{verbatim}
 help modellang
 help modellang/!keyword
 help modellang/!command 
 help modellang/pseudofunction
 \end{verbatim}
 
 \paragraph{Matlab functions and user functions in model files}
 
 You can use any of the built-in functions (Matlab functions, functions
 within the Toolboxes you have on your computer, and so on). In addition,
 you can also use your own functions (written as an m-file) as long as
 the m-file is on the Matlab search path or in the current directory.
 
 In your own m-file functions, you can also (optionally) supply the first
 derivatives that will be used to compute Taylor expansions when the
 model is being solved, and also the second derivatives that will be used
 when the function occurs in a loss function. Functions in which you wish
 to use your own derivative must be listed in a
 \href{modellang/userdiff}{\texttt{!userdiff}} section in the model file.
 
 Furthermore, the function itself must comply with the following
 requirements. When asked for the derivatives, the function is called
 with two extra input arguments on top of that function's regular input
 arguments. The first extra input argument is a text string
 \texttt{'diff'} (indicating the call to the function is supposed to
 return a derivative). The second extra input argument is a number or a
 vector of two numbers; it determines with respect to which input
 argument or arguments the first derivative or the second derivative is
 requested.
 
 For instance, your function takes three input arguments,
 \texttt{myfunc(x,y,z)}. To be able to supply derivates avoiding thus
 numerical differentiation, the function must be written so that the
 following three calls
 
 \begin{verbatim}
 myfunc(x,y,z,'diff',1)
 myfunc(x,y,z,'diff',2)
 myfunc(x,y,z,'diff',3)
 \end{verbatim}
 
 return the first derivative wrt to the first, second, and third input
 argument, respectively, while
 
 \begin{verbatim}
 myfunc(x,y,z,'diff',[1,2])
 \end{verbatim}
 
 returns the second derivative wrt to the first and second input
 arguments. Note that second derivatives are only needed for functions
 that occur in an equation defining optimal policy objective,
 \href{modellang/min}{min}.
 
 If any of these calls fail, the respective derivative will be simply
 evaluated numerically.
 
 \paragraph{Basic rules IRIS model files}
 
 \begin{itemize}
 \item
   There can be four types of equations in IRIS models: transition
   equations which are simply the endogenous dynamic equations,
   measurement equations which link the model to observables,
   deterministic trend equations which can be added at the top of
   measurement equations, and dynamic links which can be used to link
   some parameters or steady-state values to each other.
 \item
   There can be two types of variables and two types of shocks in IRIS
   models: transition variables and shocks, and measurement variables and
   shocks.
 \item
   Each model must have at least one transition (aka endogenous) variable
   and one transition equation.
 \item
   Each variable, shock, or parameter must be declared in the appropriate
   declaration section.
 \item
   The declaration sections and equations sections can be written in any
   order.
 \item
   You can have as many declaration sections or equations sections of the
   same kind as you wish in one model file; they all get combined
   together at the time the model is being loaded.
 \item
   Transition variables can occur with lags and leads in transition
   equations. Transition variables cannot, though, have leads in
   measurement equations.
 \item
   Measurement variables and the shocks cannot have any lags or leads.
 \item
   Transition shocks cannot occur in measurement equations, and the
   measurement shocks cannot occur in transition equations.
 \item
   Exogenous variables can only occur in dtrends (deterministic trend
   equations), and must be always supplied in the input database to
   commands like \texttt{model/simulate}, \texttt{model/jforecast},
   \texttt{model/filter}, \texttt{model/estimate}, etc. They are not
   returned in the output databases.
 \item
   You can choose between linearisation and log-linearisation for each
   individual transition and measurement variable. Shocks are always
   linearised. Exogenous variables must be always introduced so that
   their effect on the respective measurement variable is linear.
 \end{itemize}



