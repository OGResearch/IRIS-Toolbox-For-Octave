

    \filetitle{!transition\_equations}{Block of transition equations}{modellang/transitionequations}

	\paragraph{Syntax}

\begin{verbatim}
!transition_equations
    Equation1;
    Equation2;
    Equation2;
    ...
\end{verbatim}

\paragraph{Short-cut syntax}

\begin{verbatim}
!equations
    Equation1;
    Equation2;
    Equation3;
    ...
\end{verbatim}

\paragraph{Syntax with equation
labels}

\begin{verbatim}
!transition_equations
    Equation1;
    'Equation label' Equation2;
    Equation3;
    ...
\end{verbatim}

\paragraph{Description}

The \texttt{!transition\_equations} keyword starts a new block of
transition equations (i.e.~endogenous equations); the eqautions can
stretch over multiple lines and must be separated by semi-colons. You
can have as many equation blocks as you wish in any order in your model
file: They all get combined together when you read the model file in.

You can add descriptive labels to the equations (in single or double
quotes, preceding the equation); these will be stored in, and accessible
from, the model object.

\paragraph{Example}

\begin{verbatim}
!transition_equations
    'Euler equation' C{1}/C = R*beta;
\end{verbatim}


