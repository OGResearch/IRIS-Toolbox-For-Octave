

    \filetitle{!for...!do...!end}{For loop for automated creation of model code}{modellang/for}

	\paragraph{Short-cut syntax}
 
 \begin{verbatim}
 !for
     List_of_Tokens
 !do
     Template
 !end
 \end{verbatim}
 
 \paragraph{Full syntax}
 
 \begin{verbatim}
 !for
     ?Control_Name = List_of_Tokens
 !do
     Template
 !end
 \end{verbatim}
 
 \paragraph{Description}
 
 Use the '!for\ldots{}!do\ldots{}!end' command to specify a template and
 let the IRIS preparser automatically create multiple instances of the
 template by iterating over a list of tokens. The preparser cycles over
 the individual strings from the list; in each iteration, the current
 string is used to replace all occurences of the control variable in the
 template. The name of the control name is either a question mark, '?',
 in the abbreviated syntax, or any string (not to blank spaces) specified
 by the user starting with a question mark in the full syntax, such as
 '?x', '?\#', '?NAME', etc.
 
 The tokens (text strings) in the list must be separated by commas, blank
 spaces, or line breaks and they themselves must not contain any of
 those. In each iteration,
 
 \begin{itemize}
 \item
   all occurrences of the control variable in the template are replaced
   with the currently processed string;
 \item
   all occurrences in the template of \texttt{?.Control\_Name} are
   replaced with the currently processed string converted to lower case;
   this option is NOT available with the short-cut syntax;
 \item
   all occurrences in the template of \texttt{?:Control\_Name} are
   replaced with the currently processed string converted to upper case;
   this option is NOT available with the short-cut syntax;
 \end{itemize}
 
 The list of tokens can be based on Matlab expressions. The expressions
 must be enclosed in square brackets, and must evaluate into either a
 numeric vector, a char vector, or a cell array of numerics and/or
 strings.
 
 \paragraph{Example 1}
 
 In a model code file, instead of writing a number of definitions of
 growth rates like the following ones
 
 \begin{verbatim}
 dP = P/P{-1} - 1;
 dW = W/W{-1} - 1;
 dX = X/X{-1} - 1;
 dY = Y/Y{-1} - 1;
 \end{verbatim}
 
 you can use '!for\ldots{}!do\ldots{}!end' as follows:
 
 \begin{verbatim}
 !for
     P, W, X, Y
 !do
     d? = ?/?{-1} - 1;
 !end
 \end{verbatim}
 
 \paragraph{Example 2}
 
 We redo the example 1, but using now the fact that you can have as many
 variable declaration sections or equation sections as you wish. The
 '!for\ldots{}!do\ldots{}!end' structure can therefore not only produce
 the equations for you, but also make sure all the growth rate variables
 are properly declared.
 
 \begin{verbatim}
 !for
     P, W, X, Y
 !do
     !transition_variables
         d?
     !transition_equations
         d? = ?/?{-1} - 1;
 !end
 \end{verbatim}
 
 The preparser expands this structure to the following :
 
 \begin{verbatim}
 !transition_variables
     dP
 !transition_equations
     dP = P/P{-1} - 1;
 !transition_variables
     dW
 !transition_equations
     dW = W/W{-1} - 1;
 !transition_variables
     dX
 !transition_equations
     dX = X/X{-1} - 1;
 !transition_variables
     dY
 !transition_equations
     dY = Y/Y{-1} - 1;
 \end{verbatim}
 
 Obviously, you now do not include the growth rate variables in the
 section where you declare the rest of the variables.
 
 \paragraph{Example 3}
 
 In a model code file, instead of writing a number of autoregression
 processes like the following ones
 
 \begin{verbatim}
 X = rhox*X{-1} + ex;
 Y = rhoy*Y{-1} + ey;
 Z = rhoz*Z{-1} + ez;
 \end{verbatim}
 
 you can use '!for\ldots{}!do\ldots{}!end' as follows:
 
 \begin{verbatim}
 !for
     ?# = X, Y, Z
 !do
     ?# = rho?.#*?{-1} + e?.#;
 !end
 \end{verbatim}
 
 \paragraph{Example 4}
 
 We redo Example 3, but now for six variables named `A1', `A2', `B1',
 `B2', `C1', `C2', nesting two '!for\ldots{}!do\ldots{}!end' structures
 one within the other:
 
 \begin{verbatim}
 !for
     ?letter = A, B, C
 !do
     !for
         ?number = 1, 2
     !do
         ?letter?number = rho?.letter?number*?letter?number{-1}
             + e?.letter?number;
     !end
 !end
 \end{verbatim}
 
 The preparser produces the following six equations:
 
 \begin{verbatim}
 A1 = rhoa1*A1{-1} + ea1;
 A2 = rhoa2*A2{-1} + ea2;
 B1 = rhob1*B1{-1} + eb1;
 B2 = rhob2*B2{-1} + eb2;
 C1 = rhoc1*C1{-1} + ec1;
 C2 = rhoc2*C2{-1} + ec2;
 \end{verbatim}
 
 \paragraph{Example 5}
 
 We use a Matlab expression (the colon operator) to simplify the list of
 tokens. The following block of code
 
 \begin{verbatim}
 !for
     1, 2, 3, 4, 5, 6, 7
 !do
     a? = a?{-1} + res_a?;
 !end
 \end{verbatim}
 
 can be simplified as follow:
 
 \begin{verbatim}
 !for
     [ 1 : 7 ]
 !do
     a? = a?{-1} + res_a?;
 !end
 \end{verbatim}


