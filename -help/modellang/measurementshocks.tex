

    \filetitle{!measurement\_shocks}{List of measurement shocks}{modellang/measurementshocks}

	\paragraph{Syntax}

\begin{verbatim}
!measurement_shocks
    SHOCK_NAME, SHOCK_NAME, ...
    ...
\end{verbatim}

\paragraph{Syntax with descriptors}

\begin{verbatim}
!measurement_shocks
    SHOCK_NAME, SHOCK_NAME, ...
    'Description of the shock...' SHOCK_NAME
\end{verbatim}

\paragraph{Description}

The \texttt{!measurement\_shocks} keyword starts a new declaration block
for measurement shocks (i.e.~shocks or errors to measurement equation);
the names of the shocks must be separated by commas, semi-colons, or
line breaks. You can have as many declaration blocks as you wish in any
order in your model file: They all get combined together when you read
the model file in. Each shock must be declared (exactly once).

You can add descriptors to the shocks (enclosed in single or double
quotes, preceding the name of the shock); these will be stored in, and
accessible from, the model object.

\paragraph{Example}

\begin{verbatim}
!measurement_shocks
    u1, 'Output measurement error' u2
    u3
\end{verbatim}


