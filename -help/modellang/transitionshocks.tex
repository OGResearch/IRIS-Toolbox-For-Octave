

    \filetitle{!transition\_shocks}{List of transition shocks}{modellang/transitionshocks}

	\paragraph{Syntax}

\begin{verbatim}
!transition_shocks
    shock_name, shock_name, ...
    ...
\end{verbatim}

\paragraph{Short-cut syntax}

\begin{verbatim}
!shocks
    shock_name, shock_name, ...
    ...
\end{verbatim}

\paragraph{Syntax with descriptors}

\begin{verbatim}
!transition_shocks
    shock_name, shock_name, ...
    'Description of the shock...' shock_name
\end{verbatim}

\paragraph{Description}

The \texttt{!transition\_shocks} keyword starts a new declaration block
for transition shocks (i.e.~shocks to transition equation); the names of
the shocks must be separated by commas, semi-colons, or line breaks. You
can have as many declaration blocks as you wish in any order in your
model file: They all get combined together when you read the model file
in. Each shock must be declared (exactly once).

You can add descriptors to the shocks (enclosed in single or double
quotes, preceding the name of the shock); these will be stored in, and
accessible from, the model object.

\paragraph{Example}

\begin{verbatim}
!transition_shocks
    e1, 'Aggregate supply shock' e2
    e3
\end{verbatim}


