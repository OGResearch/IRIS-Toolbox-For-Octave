

    \filetitle{!exogenous\_variables}{List of exogenous variables}{modellang/exogenousvariables}

	\paragraph{Syntax}
 
 \begin{verbatim}
 !exogenous_variables
     Variable_Name, Variable_Name, ...
     ...
 \end{verbatim}
 
 \paragraph{Syntax with descriptors}
 
 \begin{verbatim}
 !exogenous_variables
     Variable_Name, Variable_Name, ...
     'Description of the variable...' Variable_Name
 \end{verbatim}
 
 \paragraph{Syntax with steady-state values}
 
 \begin{verbatim}
 !exogenous_variables
     Variable_Name, Variable_Name, ...
     Variable_Name = Value
 \end{verbatim}
 
 \paragraph{Description}
 
 The \texttt{!exogenous\_variables} keyword starts a new declaration
 block for exogenous variables, i.e.~variables that can appear only in
 \href{modellang/dtrends}{\texttt{!dtrends}} equations. The names of the
 variables must be separated by commas, semi-colons, or line breaks. You
 can have as many declaration blocks as you wish in any order in your
 model file: They all get combined together when you read the model file
 in. Each variable must be declared (exactly once).
 
 You can add descriptors to the variables (enclosed in single or double
 quotes, preceding the name of the variable); these will be stored in,
 and accessible from, the model object. You can also assign steady-state
 values to the variables straight in the model file (following an equal
 sign after the name of the variable); this is, though, rather rare and
 unnecessary practice because you can assign and change steady-state
 values more conveniently in the model object.
 
 \paragraph{Example}
 
 \begin{verbatim}
 !exogenous_variables
     X, 'Tax effects' Y
     'Population growth effects' Z = 0 + 0.5i;
 \end{verbatim}


