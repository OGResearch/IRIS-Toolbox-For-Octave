

    \filetitle{min}{Define the loss function in a time-consistent optimal policy model}{modellang/min}

	\paragraph{Syntax}
 
 \begin{verbatim}
 min(DISC) EXPRESSION;
 \end{verbatim}
 
 \paragraph{Syntax for exact non-linear simulations}
 
 \begin{verbatim}
 min#(DISC) EXPRESSION;
 \end{verbatim}
 
 \paragraph{Description}
 
 The loss function must be types as one of the transition equations. The
 \texttt{DISC} is a parameter or an expression defining the discount
 factor (applied to future dates), the \texttt{EXPRESSION} defines the
 loss fuction proper.
 
 If you use the \texttt{min\#(DISC)} syntax, all equations created by
 differentiating the lagrangian w.r.t. individual variables will be
 earmarked for exact non-linear simulations provided the respective
 derivative is nonzero.
 
 \paragraph{Example}
 
 This is a simple model file with a Phillips curve and a quadratic loss
 function.
 
 \begin{verbatim}
 !transition_variables
     x, pi
 
 !transition_shocks
     u
 
 !parameters
     alpha, beta, gamma
 
 !transition_equations
     min(beta) pi^2 + lambda*x^2;
     pi = alpha*pi{-1} + (1-alpha)*pi{1} + gamma*y + u;
 \end{verbatim}


