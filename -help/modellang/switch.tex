

    \filetitle{!switch...!case...!otherwise...!end}{Switch among several branches of the model code depending on the value of an expression}{modellang/switch}

	\paragraph{Syntax with an otherwise
clause}\label{syntax-with-an-otherwise-clause}

\begin{verbatim}
!switch Expression
    !case Balue1
        Block1
    !case Balue2
        Block2
    ...
    !otherwise
        Otherwise_Block
!end
\end{verbatim}

\paragraph{Syntax without an otherwise
clause}\label{syntax-without-an-otherwise-clause}

\begin{verbatim}
!switch Expression
    !case Value1
        Block1
    !case Value2
        Block2
    ...
!end
\end{verbatim}

\paragraph{Description}\label{description}

The \texttt{!switch...!case...!otherwise...!end} command works the same
way as its counterpart in the Matlab programming language.

Use the \texttt{!switch...!case...!end} command to create a larger
number of branches of the model code. Which block of code is actually
read in and which blocks are discarded depends on which value in the
\texttt{!case} clauses matches the value of the \texttt{!switch}
expression. This works exactly as the \texttt{switch...case...end}
command in Matlab. The expression after the \texttt{!switch} part of the
command must must be a valid Matlab expression, and can refer to the
model parameters, or to other fields included in the parameter database
passed in when you run the \href{model/model}{\texttt{model}} function;
see \href{model/model}{the option \texttt{'assign='}}.

If the expression fails to be matched by any value in the \texttt{!case}
clauses, the branch in the \texttt{!otherwise} clause is used. If it is
a \texttt{!switch} command without the \texttt{!otherwise} clause, the
whole command is discarded. The Matlab function \texttt{isequal} is used
to match the \texttt{!switch} expression with the \texttt{!case} values.

\paragraph{Example}\label{example}

\begin{verbatim}
!switch policy_regime

    !case 'IT'
        r = rho*r{-1} + (1-rho)*kappa*pie{4} + epsilon;

    !case 'Managed_exchange_rate'
        s = s{-1} + epsilon;

    !case 'Constant_money_growth'
        m-m{-1} = m{-1}-m{-2} + epsilon;
   
!end
\end{verbatim}

When reading the model file in, create a parameter database, include at
least a field named \texttt{policy\_regime} in it, and use the option
\texttt{'assign='} to pass the database in. Note that you do not need to
declare \texttt{policy\_regime} as a parameter in the model file.

\begin{verbatim}
P = struct();
P.policy_regime = 'Managed_exchange_rate';
...
m = model('my.model','assign',P);
\end{verbatim}

In this case, the managed exchange rate policy rule,
\texttt{s = s\{-1\} + epsilon;} is read in and the rest of the
\texttt{!switch} command is discarded. To use another branch of the
\texttt{!switch} command you need to re-read the model file again with a
different value assigned to the \texttt{policy\_regime} field of the
input database.


