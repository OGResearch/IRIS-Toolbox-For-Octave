

    \filetitle{!measurement\_equations}{Block of measurement equations}{modellang/measurementequations}

	\paragraph{Syntax}
 
 \begin{verbatim}
 !measurement_equations
     Equation1;
     Equation2;
     Equation3;
     ...
 \end{verbatim}
 
 \paragraph{Syntax with equation labels}
 
 \begin{verbatim}
 !measurement_equations
     Equation1;
     'Equation label' Equation2;
     Equation3;
     ...
 \end{verbatim}
 
 \paragraph{Description}
 
 The \texttt{!measurement\_equations} keyword starts a new block of
 measurement equations; the eqautions can stretch over multiple lines and
 must be separated by semi-colons. You can have as many equation blocks
 as you wish in any order in your model file: They all get combined
 together when you read the model file in.
 
 You can add descriptive labels to the equations (in single or double
 quotes, preceding the equation); these will be stored in, and accessible
 from, the model object.
 
 \paragraph{Example}
 
 \begin{verbatim}
 !measurement_equations
     'Inflation observations' Infl = 40*(P/P{-1} - 1);
 \end{verbatim}


