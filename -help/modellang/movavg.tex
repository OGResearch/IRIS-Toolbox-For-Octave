

    \filetitle{movavg}{Moving average pseudofunction}{modellang/movavg}

	\paragraph{Syntax}\label{syntax}

\begin{verbatim}
movavg(Expr)
movavg(Expr,K)
\end{verbatim}

\paragraph{Description}\label{description}

If the second input argument, \texttt{K}, is negative, this function
expands to the moving average of the last K periods (including the
current period), i.e.

\begin{verbatim}
(((Expr)+(Expr{-1})+ ... +(Expr{-(K-1)})/-K)
\end{verbatim}

where \texttt{Expr\{-N\}} derives from \texttt{Expr} and has all its
time subscripts shifted by \texttt{-N} (if specified).

If the second input argument, \texttt{K}, is positive, this function
expands to the moving average of the next K periods ahead (including the
current period), i.e.

\begin{verbatim}
(((Expr)+(Expr{1})+ ... +(Expr{K-1})/K)
\end{verbatim}

If the second input argument, \texttt{K}, is not specified, the default
value -4 is used (based on the fact that most of the macroeconomic
models are quarterly).

\paragraph{Example}\label{example}

The following three lines

\begin{verbatim}
movavg(Z)
movavg(Z,-3)
movavg(X+Y{-1},2)
\end{verbatim}

will expand to

\begin{verbatim}
(((Z)+(Z{-1})+(Z{-2})+(Z{-3}))/4)
(((Z)+(Z{-1})+(Z{-2}))/3)
(((X+Y{-1})+(X{1}+Y))/2)
\end{verbatim}


