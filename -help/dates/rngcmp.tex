

    \filetitle{rngcmp}{Compare two IRIS date ranges}{dates/rngcmp}

	\paragraph{Syntax}
 
 \begin{verbatim}
 Flag = rngcmp(R1,R2)
 \end{verbatim}
 
 \paragraph{Input arguments}
 
 \begin{itemize}
 \item
   \texttt{R1}, \texttt{R2} {[} numeric {]} - Two IRIS date ranges that
   will be compared.
 \end{itemize}
 
 \paragraph{Output arguments}
 
 \begin{itemize}
 \item
   \texttt{Flag} {[} \texttt{true} \textbar{} \texttt{false} {]} - True
   if the two date ranges are the same.
 \end{itemize}
 
 \paragraph{Description}
 
 An IRIS date range is distinct from a vector of dates in that only the
 first and the last dates matter. Often, date ranges are context
 sensitive. In that case, you can use \texttt{-Inf} for the start date
 (meaning the earliest possible date in the given context) and
 \texttt{Inf} for the end date (meaning the latest possible date in the
 given context), or simply \texttt{Inf} for the whole range (meaning from
 the earliest possible date to the latest possible date in the given
 context).
 
 \paragraph{Example}
 
 \begin{verbatim}
 r1 = qq(2010,1):qq(2020,4);
 r2 = [qq(2010,1),qq(2020,4)];
 
 rngcmp(r1,r2)
 ans =
     1
 \end{verbatim}


