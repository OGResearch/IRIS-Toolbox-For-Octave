

    \filetitle{datcmp}{Compare two IRIS serial date numbers}{dates/datcmp}

	\paragraph{Syntax}\label{syntax}

\begin{verbatim}
Flag = datcmp(Dat1,Dat2)
\end{verbatim}

\paragraph{Input arguments}\label{input-arguments}

\begin{itemize}
\itemsep1pt\parskip0pt\parsep0pt
\item
  \texttt{Dat1}, \texttt{Dat2} {[} numeric {]} - IRIS serial date
  numbers or vectors.
\end{itemize}

\paragraph{Output arguments}\label{output-arguments}

\begin{itemize}
\itemsep1pt\parskip0pt\parsep0pt
\item
  \texttt{Flag} {[} \texttt{true} \textbar{} \texttt{false} {]} - True
  for numbers that represent the same date.
\end{itemize}

\paragraph{Description}\label{description}

The two date vectors must either be the same lengths, or one of them
must be scalar.

Use this function instead of the plain comparison operator, \texttt{==},
to compare dates. The plain comparision can sometimes give false results
because of round-off errors.

\paragraph{Example}\label{example}

\begin{verbatim}
d1 = qq(2010,1);
d2 = qq(2009,1):qq(2010,4);
datcmp(d1,d2)
ans =
    0     0     0     0     1     0     0     0
\end{verbatim}


