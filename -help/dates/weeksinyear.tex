

    \filetitle{weeksinyear}{Number of weeks in year}{dates/weeksinyear}

	\paragraph{Syntax}\label{syntax}

\begin{verbatim}
N = weeksinyear(Year)
\end{verbatim}

\paragraph{Input arguments}\label{input-arguments}

\begin{itemize}
\itemsep1pt\parskip0pt\parsep0pt
\item
  \texttt{Year} {[} numeric {]} - Year.
\end{itemize}

\paragraph{Output arguments}\label{output-arguments}

\begin{itemize}
\itemsep1pt\parskip0pt\parsep0pt
\item
  \texttt{N} {[} numeric {]} - Number of weeks in \texttt{Year}.
\end{itemize}

\paragraph{Description}\label{description}

The number of weeks in a year is either \texttt{52} or \texttt{53}, and
complies with the definition of the first week in a year in ISO 8601.
The first week of a year is the one that contains the 4th day of January
(in other words, has most of its days in that year).

\paragraph{Example}\label{example}

\begin{verbatim}
weeksinyear(2000:2010)
ans =
    52    52    52    52    53    52    52    52    52    53    52
\end{verbatim}


