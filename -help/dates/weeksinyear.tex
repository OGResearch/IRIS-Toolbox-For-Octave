

    \filetitle{weeksinyear}{Number of weeks in year}{dates/weeksinyear}

	\paragraph{Syntax}
 
 \begin{verbatim}
 N = weeksinyear(Year)
 \end{verbatim}
 
 \paragraph{Input arguments}
 
 \begin{itemize}
 \item
   \texttt{Year} {[} numeric {]} - Year.
 \end{itemize}
 
 \paragraph{Output arguments}
 
 \begin{itemize}
 \item
   \texttt{N} {[} numeric {]} - Number of weeks in \texttt{Year}.
 \end{itemize}
 
 \paragraph{Description}
 
 The number of weeks in a year is either \texttt{52} or \texttt{53}, and
 complies with the definition of the first week in a year in ISO 8601.
 The first week of a year is the one that contains the 4th day of January
 (in other words, has most of its days in that year).
 
 \paragraph{Example}
 
 \begin{verbatim}
 weeksinyear(2000:2010)
 ans =
     52    52    52    52    53    52    52    52    52    53    52
 \end{verbatim}


