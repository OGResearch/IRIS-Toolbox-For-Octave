

    \filetitle{datrange}{Numerically safe way to create a date range}{dates/datrange}

	\paragraph{Syntax}

\begin{verbatim}
Rng = datrange(Start,End)
Rng = datrange(Start,End,Step)
\end{verbatim}

\paragraph{Input arguments}

\begin{itemize}
\item
  \texttt{Start} {[} numeric {]} - Start date of the range.
\item
  \texttt{End} {[} numeric {]} - End date of the range.
\item
  \texttt{Step} {[} numeric {]} - Step size in the number of base
  periods; if omitted, \texttt{Step = 1}.
\end{itemize}

\paragraph{Output arguments}

\begin{itemize}
\itemsep1pt\parskip0pt\parsep0pt
\item
  \texttt{Rng} {[} numeric {]} - Date vector
  \texttt{Start : Step : End}.
\end{itemize}

\paragraph{Description}

Most of the time, using a colon operator to create a date range works
fine,

\begin{verbatim}
Start : Step : End
\end{verbatim}

Under some rare circumstances, the colon operator may bump into round
error difficulties as IRIS serial date numbers are non-integer values.
In that case, the function \texttt{datrange} provides a safe workaround:

\begin{verbatim}
datrange(Start,End,Step)
\end{verbatim}

is equivalent (but numerically safer) to

\begin{verbatim}
Start : Step : End
\end{verbatim}

\paragraph{Example}

The date ranges created in this example are identical, and no numerical
inaccuracies exist:

\begin{verbatim}
r1 = qq(2000,1) : qq(2010,4);
r2 = datrange(qq(2000,1),qq(2010,4));
format long
r1 - r2
\end{verbatim}


