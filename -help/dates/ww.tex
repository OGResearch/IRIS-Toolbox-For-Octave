

    \filetitle{ww}{IRIS serial date number for weekly date}{dates/ww}

	\paragraph{Syntax}

\begin{verbatim}
Dat = ww(Year,Week)
Dat = ww(Year,Month,Day)
\end{verbatim}

\paragraph{Input arguments}

\begin{itemize}
\item
  \texttt{Year} {[} numeric {]} - Years.
\item
  \texttt{Week} {[} numeric {]} - Week of the year.
\item
  \texttt{Month} {[} numeric {]} - Calendar month.
\item
  \texttt{Day} {[} numeric {]} - Calendar day of the month
  \texttt{Month}.
\end{itemize}

\paragraph{Output arguments}

\begin{itemize}
\itemsep1pt\parskip0pt\parsep0pt
\item
  \texttt{Dat} {[} numeric {]} - IRIS serial date number representing
  the weekly date.
\end{itemize}

\paragraph{Description}

The IRIS weekly dates comply with the ISO 8601 definition:

\begin{itemize}
\item
  every week starts on Monday and ends on Sunday;
\item
  the month or year to which the week belongs is determined by its
  Thurdsay.
\end{itemize}

\paragraph{Example}


