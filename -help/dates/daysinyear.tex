

    \filetitle{daysinyear}{Number of days in year}{dates/daysinyear}

	\paragraph{Syntax}

\begin{verbatim}
N = daysinyear(Year)
\end{verbatim}

\paragraph{Input arguments}

\begin{itemize}
\itemsep1pt\parskip0pt\parsep0pt
\item
  \texttt{Year} {[} numeric {]} - Year.
\end{itemize}

\paragraph{Output arguments}

\begin{itemize}
\itemsep1pt\parskip0pt\parsep0pt
\item
  \texttt{N} {[} numeric {]} - Number of days in \texttt{Year}.
\end{itemize}

\paragraph{Description}

\texttt{N} is \texttt{365} for non-leap years, and \texttt{366} for leap
years. Leap years either those divisible by \texttt{4} but not
\texttt{100}, or those divisible by \texttt{400}.

\paragraph{Example}

\begin{verbatim}
daysinyear([2000,2200])
ans =
   366   365
\end{verbatim}


