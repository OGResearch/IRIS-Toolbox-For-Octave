

    \filetitle{dat2ttrend}{Construct linear time trend from date range}{dates/dat2ttrend}

	\paragraph{Syntax}\label{syntax}

\begin{verbatim}
[TTrend,BaseDate] = dat2ttrend(Range)
[TTrend,BaseDate] = dat2ttrend(Range,BaseYear)
[TTrend,BaseDate] = dat2ttrend(Range,Obj)
\end{verbatim}

\paragraph{Input arguments}\label{input-arguments}

\begin{itemize}
\item
  \texttt{Range} {[} numeric {]} - Date range from which an integer
  linear time trend will be constructed.
\item
  \texttt{BaseYear} {[} model \textbar{} VAR {]} - Base year that will
  be used to construct the time trend.
\item
  \texttt{Obj} {[} model \textbar{} VAR {]} - Model or VAR object whose
  base year will be used to construct the time trend; if both
  \texttt{BaseYear} and \texttt{Obj} are omitted, the base year from
  \texttt{irisget('baseYear')} will be used.
\end{itemize}

\paragraph{Output arguments}\label{output-arguments}

\begin{itemize}
\item
  \texttt{TTrend} {[} numeric {]} - Integer linear time trend, unique to
  the input date range \texttt{Range} and the base year.
\item
  \texttt{BaseDate} {[} numeric {]} - Base date used to normalize the
  input date range; see Description.
\end{itemize}

\paragraph{Description}\label{description}

For regular date frequencies, the time trend is constructed the
following way. First, a base date is created first period in the base
year of a given frequency. For instance, for a quarterly input range,
\texttt{BaseDate = qq(baseYear,1)}, for a monthly input range,
\texttt{BaseDate == mm(baseYear,1)}, etc. Then, the output trend is an
integer vector normalized to the base date,

\begin{verbatim}
TTrend = floor(Range - BaseDate);
\end{verbatim}

For indeterminate date frequencies, \texttt{BaseDate = 0}, and the
output time trend is simply the input date range.

\paragraph{Example}\label{example}


