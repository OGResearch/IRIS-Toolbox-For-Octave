

    \filetitle{daterange}{Use the colon operator to create date ranges}{dates/daterange}

	\paragraph{Syntax}

\begin{verbatim}
startdate : enddate
startdate : increment : enddate
\end{verbatim}

\paragraph{Input arguments}

\begin{itemize}
\item
  \texttt{startdate} {[} numeric {]} - IRIS serial date number
  representing the startdate.
\item
  \texttt{enddate} {[} numeric {]} - IRIS serial date number
  representing the enddate; \texttt{startdate} and \texttt{enddate} must
  be the same frequency.
\item
  \texttt{increment} {[} numeric {]} - Number of periods (specific to
  each frequency) between the dates in the date vector.
\end{itemize}

\paragraph{Description}

You can use the colon operator to create continuous date ranges because
the IRIS serial date numbers are designed so that whatever the frequency
two consecutive dates are represented by numbers that differ exactly by
one.

\paragraph{Example}


