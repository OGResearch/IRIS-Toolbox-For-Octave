

    \filetitle{islinear}{True for models declared as linear}{model/islinear}

	\paragraph{Syntax}\label{syntax}

\begin{verbatim}
Flag = islinear(M)
\end{verbatim}

\paragraph{Input arguments}\label{input-arguments}

\begin{itemize}
\itemsep1pt\parskip0pt\parsep0pt
\item
  \texttt{m} {[} model {]} - Queried model object.
\end{itemize}

\paragraph{Output arguments}\label{output-arguments}

\begin{itemize}
\itemsep1pt\parskip0pt\parsep0pt
\item
  \texttt{Flag} {[} \texttt{true} \textbar{} \texttt{false} {]} - True
  if the model has been declared linear.
\end{itemize}

\paragraph{Description}\label{description}

The value returned dependes on whether the model has been declared as
linear by the user when constructing the model object by calling the
\href{model/model}{\texttt{model/model}} function. In other words, no
check is performed whether or not the model is actually linear.

\paragraph{Example}\label{example}

\begin{verbatim}
m = model('mymodel.file','linear=',true);
islinear(m)
ans =
     1
\end{verbatim}


