

    \filetitle{islinear}{True for models declared as linear}{model/islinear}

	\paragraph{Syntax}
 
 \begin{verbatim}
 Flag = islinear(M)
 \end{verbatim}
 
 \paragraph{Input arguments}
 
 \begin{itemize}
 \item
   \texttt{m} {[} model {]} - Queried model object.
 \end{itemize}
 
 \paragraph{Output arguments}
 
 \begin{itemize}
 \item
   \texttt{Flag} {[} \texttt{true} \textbar{} \texttt{false} {]} - True
   if the model has been declared linear.
 \end{itemize}
 
 \paragraph{Description}
 
 The value returned dependes on whether the model has been declared as
 linear by the user when constructing the model object by calling the
 \url{model/model} function. In other words, no check is performed
 whether or not the model is actually linear.
 
 \paragraph{Example}
 
 \begin{verbatim}
 m = model('mymodel.file','linear=',true);
 islinear(m)
 ans =
      1
 \end{verbatim}


