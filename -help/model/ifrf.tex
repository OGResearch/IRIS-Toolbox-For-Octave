

    \filetitle{ifrf}{Frequency response function to shocks}{model/ifrf}

	\paragraph{Syntax}

\begin{verbatim}
[W,List] = ifrf(M,Freq,...)
\end{verbatim}

\paragraph{Input arguments}

\begin{itemize}
\item
  \texttt{M} {[} model {]} - Model object for which the frequency
  response function will be computed.
\item
  \texttt{Freq} {[} numeric {]} - Vector of frequencies for which the
  response function will be computed.
\end{itemize}

\paragraph{Output arguments}

\begin{itemize}
\item
  \texttt{W} {[} numeric {]} - Array with frequency responses of
  transition variables (in rows) to shocks (in columns).
\item
  \texttt{List} {[} cell {]} - List of transition variables in rows of
  the \texttt{W} matrix, and list of shocks in columns of the \texttt{W}
  matrix.
\end{itemize}

\paragraph{Options}

\begin{itemize}
\item
  \texttt{'output='} {[} \emph{\texttt{'namedmat'}} \textbar{}
  \texttt{'numeric'} {]} - Output matrix \texttt{W} will be either a
  namedmat object or a plain numeric array; if the option
  \texttt{'select='} is used, \texttt{'output='} is always
  \texttt{'namedmat'}.
\item
  \texttt{'select='} {[} char \textbar{} cellstr \textbar{}
  \emph{\texttt{Inf}} {]} - Return the frequency response function only
  for selected variables and/or selected shocks.
\end{itemize}

\paragraph{Description}

\paragraph{Example}


