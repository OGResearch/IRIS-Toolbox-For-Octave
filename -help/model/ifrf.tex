

    \filetitle{ifrf}{Frequency response function to shocks}{model/ifrf}

	\paragraph{Syntax}\label{syntax}

\begin{verbatim}
[W,List] = ifrf(M,Freq,...)
\end{verbatim}

\paragraph{Input arguments}\label{input-arguments}

\begin{itemize}
\item
  \texttt{M} {[} model {]} - Model object for which the frequency
  response function will be computed.
\item
  \texttt{Freq} {[} numeric {]} - Vector of frequencies for which the
  response function will be computed.
\end{itemize}

\paragraph{Output arguments}\label{output-arguments}

\begin{itemize}
\item
  \texttt{W} {[} namedmat \textbar{} numeric {]} - Array with frequency
  responses of transition variables (in rows) to shocks (in columns).
\item
  \texttt{List} {[} cell {]} - List of transition variables in rows of
  the \texttt{W} matrix, and list of shocks in columns of the \texttt{W}
  matrix.
\end{itemize}

\paragraph{Options}\label{options}

\begin{itemize}
\item
  \texttt{'matrixFmt='} {[} \emph{\texttt{'namedmat'}} \textbar{}
  \texttt{'plain'} {]} - Return matrix \texttt{W} as either a
  \href{namedmat/Contents}{\texttt{namedmat}} object (i.e.~matrix with
  named rows and columns) or a plain numeric array.
\item
  \texttt{'select='} {[} \emph{\texttt{@all}} \textbar{} char \textbar{}
  cellstr {]} - Return IFRF for selected variables only; \texttt{@all}
  means all variables.
\end{itemize}

\paragraph{Description}\label{description}

\paragraph{Example}\label{example}


