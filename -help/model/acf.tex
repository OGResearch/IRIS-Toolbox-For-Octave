

    \filetitle{acf}{Autocovariance and autocorrelation functions for model variables}{model/acf}

	\paragraph{Syntax}
 
 \begin{verbatim}
 [C,R,List] = acf(M,...)
 \end{verbatim}
 
 \paragraph{Input arguments}
 
 \begin{itemize}
 \item
   \texttt{M} {[} model {]} - Solved model object for which the ACF will
   be computed.
 \end{itemize}
 
 \paragraph{Output arguments}
 
 \begin{itemize}
 \item
   \texttt{C} {[} namedmat \textbar{} numeric {]} - Auto/cross-covariance
   matrices.
 \item
   \texttt{R} {[} namedmat \textbar{} numeric {]} -
   Auto/cross-correlation matrices.
 \item
   \texttt{List} {[} cellstr {]} - List of variables in rows and columns
   of \texttt{C} and \texttt{R}.
 \end{itemize}
 
 \paragraph{Options}
 
 \begin{itemize}
 \item
   \texttt{'applyTo='} {[} cellstr \textbar{} char \textbar{}
   \emph{\texttt{Inf}} {]} - List of variables to which the
   \texttt{'filter='} will be applied; \texttt{Inf} means all variables.
 \item
   \texttt{'contributions='} {[} \texttt{true} \textbar{}
   \emph{\texttt{false}} {]} - If \texttt{true} the contributions of
   individual shocks to ACFs will be computed and stored in the 5th
   dimension of the \texttt{C} and \texttt{R} matrices.
 \item
   \texttt{'filter='} {[} char \textbar{} \emph{empty} {]} - Linear
   filter that is applied to variables specified by `applyto'.
 \item
   \texttt{'nFreq='} {[} numeric \textbar{} \emph{\texttt{256}} {]} -
   Number of equally spaced frequencies over which the filter in the
   option \texttt{'filter='} is numerically integrated.
 \item
   \texttt{'order='} {[} numeric \textbar{} \emph{\texttt{0}} {]} - Order
   up to which ACF will be computed.
 \item
   \texttt{'output='} {[} \emph{\texttt{'namedmat'}} \textbar{}
   \texttt{'numeric'} {]} - Output matrices \texttt{C} and \texttt{R}
   will be either namedmat objects or plain numeric arrays; if the option
   \texttt{'select='} is used, \texttt{'output='} is always a namedmat
   object.
 \item
   \texttt{'select='} {[} cellstr \textbar{} \emph{\texttt{Inf}} {]} -
   Return ACF for selected variables only; \texttt{Inf} means all
   variables.
 \end{itemize}
 
 \paragraph{Description}
 
 \texttt{C} and \texttt{R} are both N-by-N-by-(P+1)-by-Alt matrices,
 where N is the number of measurement and transition variables (including
 auxiliary lags and leads in the state space vector), P is the order up
 to which the ACF is computed (controlled by the option
 \texttt{'order='}), and Alt is the number of alternative
 parameterisations in the input model object, \texttt{M}. If
 \texttt{'contributions=' true}, the size of the two matrices is
 N-by-N-by-(P+1)-by-E-Alt, where E is the number of measurement and
 transition shocks in the model.
 
 \subparagraph{ACF with linear filters}
 
 You can use the option \texttt{'filter='} to get the ACF for variables
 as though they were filtered through a linear filter. You can specify
 the filter in both the time domain (such as first-difference filter, or
 Hodrick-Prescott) and the frequncy domain (such as a band of certain
 frequncies or periodicities). The filter is a text string in which you
 can use the following references:
 
 \begin{itemize}
 \item
   \texttt{'L'}, the lag operator, which will be replaced with
   \texttt{exp(-1i*freq)};
 \item
   \texttt{'per'}, the periodicity;
 \item
   \texttt{'freq'}, the frequency.
 \end{itemize}
 
 \paragraph{Example 1}
 
 A first-difference filter (i.e.~computes the ACF for the first
 differences of the respective variables):
 
 \begin{verbatim}
 [C,R] = acf(m,'filter=','1-L')
 \end{verbatim}
 
 \paragraph{Example 2}
 
 The cyclical component of the Hodrick-Prescott filter with the smoothing
 parameter, $lambda$, 1,600. The formula for the filter follows from the
 classical Wiener-Kolmogorov signal extraction theory,
 
 \[w(L) = \frac{\lambda}{\lambda + \frac{1}{ | (1-L)(1-L) | ^2}}\]
 
 \begin{verbatim}
 [C,R] = acf(m,'filter','1600/(1600 + 1/abs((1-L)^2)^2)')
 \end{verbatim}
 
 \paragraph{Example 3}
 
 A band-pass filter with user-specified lower and upper bands. The
 band-pass filters can be defined either in frequencies or periodicities;
 the latter is usually more convenient. The following is a filter which
 retains periodicities between 4 and 40 periods (this would be between 1
 and 10 years in a quarterly model),
 
 \begin{verbatim}
 [C,R] = acf(m,'filter','per >= 4 & per <= 40')
 \end{verbatim}


