

    \filetitle{fmse}{Forecast mean square error matrices}{model/fmse}

	\paragraph{Syntax}\label{syntax}

\begin{verbatim}
[F,List,D] = fmse(M,NPer,...)
[F,List,D] = fmse(M,Range,...)
\end{verbatim}

\paragraph{Input arguments}\label{input-arguments}

\begin{itemize}
\item
  \texttt{M} {[} model {]} - Model object for which the forecast MSE
  matrices will be computed.
\item
  \texttt{NPer} {[} numeric {]} - Number of periods.
\item
  \texttt{Range} {[} numeric {]} - Date range.
\end{itemize}

\paragraph{Output arguments}\label{output-arguments}

\begin{itemize}
\item
  \texttt{F} {[} numeric {]} - Forecast MSE matrices.
\item
  \texttt{List} {[} cellstr {]} - List of variables in rows and columns
  of \texttt{M}.
\item
  \texttt{D} {[} dbase {]} - Database with the std deviations of
  individual variables, i.e.~the square roots of the diagonal elements
  of \texttt{F}.
\end{itemize}

\paragraph{Options}\label{options}

\begin{itemize}
\item
  \texttt{'output='} {[} \emph{\texttt{'namedmat'}} \textbar{}
  \texttt{'numeric'} {]} - Output matrix \texttt{M} will be either a
  namedmat object or a plain numeric array; if the option
  \texttt{'select='} is used, \texttt{'output='} is always
  \texttt{'namedmat'}.
\item
  \texttt{'select='} {[} cellstr \textbar{} \emph{\texttt{Inf}} {]} -
  Return FMSE for selected variables only; \texttt{Inf} means all
  variables. The option does not apply to the output database
  \texttt{D}.
\end{itemize}

\paragraph{Description}\label{description}

\paragraph{Example}\label{example}


