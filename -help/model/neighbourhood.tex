

    \filetitle{neighbourhood}{Evaluate the local behaviour of the objective function around the estimated parameter values}{model/neighbourhood}

	\paragraph{Syntax}
 
 \begin{verbatim}
 [D,FigH,AxH,ObjH,LikH,EstH,BH] = neighbourhood(M,PS,Neigh,...)
 \end{verbatim}
 
 \paragraph{Input arguments}
 
 \begin{itemize}
 \item
   \texttt{M} {[} model \textbar{} bkwmodel {]} - Model or bkwmodel
   object.
 \item
   \texttt{PS} {[} poster {]} - Posterior simulator (poster) object
   returned by the \url{model/estimate} function.
 \item
   \texttt{Neigh} {[} numeric {]} - The neighbourhood in which the
   objective function will be evaluated defined as multiples of the
   parameter estimates.
 \end{itemize}
 
 \paragraph{Output arguments}
 
 \begin{itemize}
 \item
   \texttt{D} {[} struct {]} - Struct describing the local behaviour of
   the objective function and the data likelihood (minus log likelihood)
   within the specified range around the optimum for each parameter.
 \end{itemize}
 
 The following output arguments are non-empty only if you choose
 \texttt{'plot='} true:
 
 \begin{itemize}
 \item
   \texttt{FigH} {[} numeric {]} - Handles to the figures created.
 \item
   \texttt{AxH} {[} numeric {]} - Handles to the axes objects created.
 \item
   \texttt{ObjH} {[} numeric {]} - Handles to the objective function
   curves plotted.
 \item
   \texttt{LikH} {[} numeric {]} - Handles to the data likelihood curves
   plotted.
 \item
   \texttt{EstH} {[} numeric {]} - Handles to the actual estimate marks
   plotted.
 \item
   \texttt{BH} {[} numeric {]} - Handles to the bounds plotted.
 \end{itemize}
 
 \paragraph{Options}
 
 \begin{itemize}
 \item
   \texttt{'plot='} {[} \emph{\texttt{true}} \textbar{} \texttt{false}
   {]} - Call the \href{grfun/plotneigh}{\texttt{grfun.plotneigh}}
   function from within \texttt{neighbourhood} to visualise the results.
 \item
   \texttt{'neighbourhood='} {[} struct \textbar{} \emph{empty} {]} -
   Struct specifying the neighbourhood points for some of the parameters;
   these points will be used instead of those based on \texttt{Neigh}.
 \end{itemize}
 
 \paragraph{Plotting options}
 
 See help on \href{grfun/plotneigh}{\texttt{grfun.plotneigh}} for options
 available when you choose \texttt{'plot='} true.
 
 \paragraph{Description}
 
 In the output database, \texttt{D}, each parameter is a 1-by-3 cell
 array. The first cell is a vector of parameter values at which the local
 behaviour was investigated. The second cell is a matrix with two column
 vectors: the values of the overall minimised objective function (as set
 up in the \href{model/estimate}{\texttt{estimate}} function), and the
 values of the data likelihood component. The third cell is a vector of
 four numbers: the parameter estimate, the value of the objective
 function at optimum, the lower bound and the upper bound.
 
 \paragraph{Example}


