/bin/bash: lhsmrhs: command not found


    \filetitle{data4lhsmrhs}{Prepare data array for running}{model/data4lhsmrhs}

	\paragraph{Syntax}\label{syntax}

\begin{verbatim}
[YXE,List,XRange] = data4lhsmrhs(M,D,Range)
\end{verbatim}

\paragraph{Input arguments}\label{input-arguments}

\begin{itemize}
\item
  \texttt{M} {[} model {]} - Model object whose equations will be later
  evaluated by calling \href{model/lhsmrhs}{\texttt{lhsmrhs}}.
\item
  \texttt{D} {[} struct {]} - Input database with observations on
  measurement variables, transition variables, and shocks on which
  \href{model/lhsmrhs}{\texttt{lhsmrhs}} will be evaluated.
\item
  \texttt{Range} {[} numeric {]} - Date range on which
  \href{model/lhsmrhs}{\texttt{lhsmrhs}} will be evaluated.
\end{itemize}

\paragraph{Output arguments}\label{output-arguments}

\begin{itemize}
\item
  \texttt{YXE} {[} numeric {]} - Numeric array with the observations on
  measurement variables, transition variables, and shocks organised
  row-wise.
\item
  \texttt{List} {[} cellstr {]} - List of measurement variables,
  transition variables and shocks in order of their appearance in the
  rows of \texttt{YXE}.
\item
  \texttt{XRange} {[} numeric {]} - Extended range including pre-sample
  and post-sample observations needed to evaluate lags and leads of
  transition variables.
\end{itemize}

\paragraph{Description}\label{description}

The resulting array, \texttt{YXE}, is \texttt{nVar} by \texttt{nXPer} by
\texttt{nData}, where \texttt{nVar} is the total number of measurement
variables, transition variables, and shocks, \texttt{nXPer} is the
number of periods including the pre-sample and post-sample periods
needed to evaluate lags and leads, and \texttt{nData} is the number of
alternative data sets (i.e.~the number of columns in each input time
series) in the input database, \texttt{D}.

\paragraph{Example}\label{example}

\begin{verbatim}
YXE = data4lhsmrhs(M,d,range);
D = lhsmrhs(M,YXE);
\end{verbatim}


