

    \filetitle{zerodb}{Create model-specific zero-deviation database}{model/zerodb}

	\paragraph{Syntax}\label{syntax}

\begin{verbatim}
[D,IsDev] = zerodb(M,Range)
[D,IsDev] = zerodb(M,Range,NCol)
\end{verbatim}

\paragraph{Input arguments}\label{input-arguments}

\begin{itemize}
\item
  \texttt{M} {[} model {]} - Model object for which the zero database
  will be created.
\item
  \texttt{Range} {[} numeric {]} - Intended simulation range; the zero
  database will be created on a range that also automatically includes
  all the necessary lags.
\item
  \texttt{NCol} {[} numeric {]} - Number of columns for each variable;
  the input argument \texttt{NCol} can be only used on models with one
  parameterisation.
\end{itemize}

\paragraph{Output arguments}\label{output-arguments}

\begin{itemize}
\item
  \texttt{D} {[} struct {]} - Database with a tseries object filled with
  zeros for each linearised variable, a tseries object filled with ones
  for each log-linearised variables, and a scalar or vector of the
  currently assigned values for each model parameter.
\item
  \texttt{IsDev} {[} \texttt{true} {]} - The second output argument is
  always \texttt{true}, and can be used to set the option
  \texttt{'deviation='} in
  \href{model/simulate}{\texttt{model/simulate}}.
\end{itemize}

\paragraph{Description}\label{description}

\paragraph{Example}\label{example}


