

    \filetitle{simulate}{Simulate model}{model/simulate}

	\paragraph{Syntax}
 
 \begin{verbatim}
 S = simulate(M,D,Range,...)
 [S,Flag,AddF,Discrep] = simulate(M,D,Range,...)
 \end{verbatim}
 
 \paragraph{Input arguments}
 
 \begin{itemize}
 \item
   \texttt{M} {[} model {]} - Solved model object.
 \item
   \texttt{D} {[} struct \textbar{} cell {]} - Input database or datapack
   from which the initial conditions and shocks from within the
   simulation range will be read.
 \item
   \texttt{Range} {[} numeric {]} - Simulation range.
 \end{itemize}
 
 \paragraph{Output arguments}
 
 \begin{itemize}
 \item
   \texttt{S} {[} struct \textbar{} cell {]} - Database with simulation
   results.
 \end{itemize}
 
 \paragraph{Output arguments in non-linear simulations}
 
 \begin{itemize}
 \item
   \texttt{Flag} {[} cell \textbar{} empty {]} - Cell array with exit
   flags for non-linearised simulations.
 \item
   \texttt{AddF} {[} cell \textbar{} empty {]} - Cell array of tseries
   with final add-factors added to first-order approximate equations to
   make non-linear equations hold.
 \item
   \texttt{Discrep} {[} cell \textbar{} empty {]} - Cell array of tseries
   with final discrepancies between LHS and RHS in equations earmarked
   for non-linear simulations by a double-equal sign.
 \end{itemize}
 
 \paragraph{Options}
 
 \begin{itemize}
 \item
   \texttt{'anticipate='} {[} \emph{\texttt{true}} \textbar{}
   \texttt{false} {]} - If \texttt{true}, real future shocks are
   anticipated, imaginary are unanticipated; vice versa if
   \texttt{false}.
 \item
   \texttt{'contributions='} {[} \texttt{true} \textbar{}
   \emph{\texttt{false}} {]} - Decompose the simulated paths into
   contributions of individual shocks.
 \item
   \texttt{'deviation='} {[} \texttt{true} \textbar{}
   \emph{\texttt{false}} {]} - Treat input and output data as deviations
   from balanced-growth path.
 \item
   \texttt{'dbOverlay='} {[} \texttt{true} \textbar{}
   \emph{\texttt{false}} \textbar{} struct {]} - Use the function
   \texttt{dboverlay} to combine the simulated output data with the input
   database, or with another database, at the end.
 \item
   \texttt{'dTrends='} {[} \emph{`auto'} \textbar{} \texttt{true}
   \textbar{} \texttt{false} {]} - Add deterministic trends to
   measurement variables.
 \item
   \texttt{'ignoreShocks='} {[} \texttt{true} \textbar{}
   \emph{\texttt{false}} {]} - Read only initial conditions from input
   data, and ignore any shocks within the simulation range.
 \item
   \texttt{'plan='} {[} plan {]} - Specify a simulation plan to swap
   endogeneity and exogeneity of some variables and shocks temporarily,
   and/or to simulate some of the non-linear equations accurately.
 \item
   \texttt{'progress='} {[} \texttt{true} \textbar{}
   \emph{\texttt{false}} {]} - Display progress bar in the command
   window.
 \end{itemize}
 
 \paragraph{Options for models with non-linearised equations}
 
 \begin{itemize}
 \item
   \texttt{'addSstate='} {[} \emph{\texttt{true}} \textbar{}
   \texttt{false} {]} - Add steady state levels to simulated paths before
   evaluating non-linear equations; this option is used only if
   \texttt{'deviation=' true}.
 \item
   \texttt{'display='} {[} \emph{\texttt{true}} \textbar{} \texttt{false}
   \textbar{} numeric \textbar{} Inf {]} - Report iterations on the
   screen; if \texttt{'display=' N}, report every \texttt{N} iterations;
   if \texttt{'display=' Inf}, report only final iteration.
 \item
   \texttt{'error='} {[} \texttt{true} \textbar{} \emph{\texttt{false}}
   {]} - Throw an error whenever a non-linear simulation fails converge;
   if \texttt{false}, only an warning will display.
 \item
   \texttt{'lambda='} {[} numeric \textbar{} \emph{\texttt{1}} {]} - Step
   size (between \texttt{0} and \texttt{1}) for add factors added to
   non-linearised equations in every iteration.
 \item
   \texttt{'reduceLambda='} {[} numeric \textbar{} \emph{\texttt{0.5}}
   {]} - Factor (between \texttt{0} and \texttt{1}) by which
   \texttt{lambda} will be multiplied if the non-linear simulation gets
   on an divergence path.
 \item
   \texttt{'maxIter='} {[} numeric \textbar{} \emph{\texttt{100}} {]} -
   Maximum number of iterations.
 \item
   \texttt{'tolerance='} {[} numeric \textbar{} \emph{\texttt{1e-5}} {]}
   - Convergence tolerance.
 \end{itemize}
 
 \paragraph{Description}
 
 \subparagraph{Output range}
 
 Time series in the output database, \texttt{S}, are are defined on the
 simulation range, \texttt{RANGE}, plus include all necessary initial
 conditions, i.e.~lags of variables that occur in the model code. You can
 use the option \texttt{'dboverlay='} to combine the output database with
 the input database (i.e.~to include a longer history of data in the
 simulated series).
 
 \subparagraph{Simulations with multilple parameterisations and/or
 multiple data sets}
 
 If you simulate a model with \texttt{N} parameterisations and the input
 database contains \texttt{K} data sets (i.e.~each variable is a time
 series with \texttt{K} columns), then the following happens:
 
 \begin{itemize}
 \item
   The model will be simulated a total of \texttt{P = max(N,K)} number of
   times. This means that each variables in the output database will have
   \texttt{P} columns.
 \item
   The 1st parameterisation will be simulated using the 1st data set, the
   2nd parameterisation will be simulated using the 2nd data set, etc.
   until you reach either the last parameterisation or the last data set,
   i.e. \texttt{min(N,K)}. From that point on, the last parameterisation
   or the last data set will be simply repeated (re-used) in the
   remaining simulations.
 \item
   Put formally, the \texttt{I}-th column in the output database, where
   \texttt{I = 1, ..., P}, is a simulation of the \texttt{min(I,N)}-th
   model parameterisation using the \texttt{min(I,K)}-th input data set
   number.
 \end{itemize}
 
 \paragraph{Example}


