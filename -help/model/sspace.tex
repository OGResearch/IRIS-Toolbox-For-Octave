

    \filetitle{sspace}{State-space matrices describing the model solution}{model/sspace}

	\paragraph{Syntax}
 
 \begin{verbatim}
 [T,R,K,Z,H,D,U,Omg] = sspace(m,...)
 \end{verbatim}
 
 \paragraph{Input arguments}
 
 \begin{itemize}
 \item
   \texttt{m} {[} model {]} - Solved model object.
 \end{itemize}
 
 \paragraph{Output arguments}
 
 \begin{itemize}
 \item
   \texttt{T} {[} numeric {]} - Transition matrix.
 \item
   \texttt{R} {[} numeric {]} - Matrix at the shock vector in transition
   equations.
 \item
   \texttt{K} {[} numeric {]} - Constant vector in transition equations.
 \item
   \texttt{Z} {[} numeric {]} - Matrix mapping transition variables to
   measurement variables.
 \item
   \texttt{H} {[} numeric {]} - Matrix at the shock vector in measurement
   equations.
 \item
   \texttt{D} {[} numeric {]} - Constant vector in measurement equations.
 \item
   \texttt{U} {[} numeric {]} - Transformation matrix for predetermined
   variables.
 \item
   \texttt{Omg} {[} numeric {]} - Covariance matrix of shocks.
 \end{itemize}
 
 \paragraph{Options}
 
 \begin{itemize}
 \item
   \texttt{'triangular='} {[} \emph{\texttt{true}} \textbar{}
   \texttt{false} {]} - If true, the state-space form returned has the
   transition matrix \texttt{T} quasi triangular and the vector of
   predetermined variables transformed accordingly. This is the form used
   in IRIS calculations.
 \end{itemize}
 
 \paragraph{Description}
 
 The state-space representation has the following form:
 
 \begin{verbatim}
 [xf;alpha] = T*alpha(-1) + K + R*e
 
 y = Z*alpha + D + H*e
 
 xb = U*alpha
 
 Cov[e] = Omg
 \end{verbatim}
 
 where \texttt{xb} is an nb-by-1 vector of predetermined
 (backward-looking) transition variables and their auxiliary lags,
 \texttt{xf} is an nf-by-1 vector of non-predetermined (forward-looking)
 variables and their auxiliary leads, \texttt{alpha} is a transformation
 of \texttt{xb}, \texttt{e} is an ne-by-1 vector of shocks, and
 \texttt{y} is an ny-by-1 vector of measurement variables. Furthermore,
 we denote the total number of transition variables, and their auxiliary
 lags and leads, nx = nb + nf.
 
 The transition matrix, \texttt{T}, is, in general, rectangular nx-by-nb.
 Furthremore, the transformed state vector alpha is chosen so that the
 lower nb-by-nb part of \texttt{T} is quasi upper triangular.
 
 You can use the \texttt{get(m,'xVector')} function to learn about the
 order of appearance of transition variables and their auxiliary lags and
 leads in the vectors \texttt{xb} and \texttt{xf}. The first nf names are
 the vector \texttt{xf}, the remaining nb names are the vector
 \texttt{xb}.


