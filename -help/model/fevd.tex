

    \filetitle{fevd}{Forecast error variance decomposition for model variables}{model/fevd}

	\paragraph{Syntax}\label{syntax}

\begin{verbatim}
[X,Y,List,A,B] = fevd(M,Range,...)
[X,Y,List,A,B] = fevd(M,NPer,...)
\end{verbatim}

\paragraph{Input arguments}\label{input-arguments}

\begin{itemize}
\item
  \texttt{M} {[} model {]} - Model object for which the decomposition
  will be computed.
\item
  \texttt{Range} {[} numeric {]} - Decomposition date range with the
  first date beign the first forecast period.
\item
  \texttt{NPer} {[} numeric {]} - Number of periods for which the
  decomposition will be computed.
\end{itemize}

\paragraph{Output arguments}\label{output-arguments}

\begin{itemize}
\item
  \texttt{X} {[} namedmat \textbar{} numeric {]} - Array with the
  absolute contributions of individual shocks to total variance of each
  variables.
\item
  \texttt{Y} {[} namedmat \textbar{} numeric {]} - Array with the
  relative contributions of individual shocks to total variance of each
  variables.
\item
  \texttt{List} {[} cellstr {]} - List of variables in rows of the
  \texttt{X} an \texttt{Y} arrays, and shocks in columns of the
  \texttt{X} and \texttt{Y} arrays.
\item
  \texttt{A} {[} struct {]} - Database with the absolute contributions
  converted to time series.
\item
  \texttt{B} {[} struct {]} - Database with the relative contributions
  converted to time series.
\end{itemize}

\paragraph{Options}\label{options}

\begin{itemize}
\item
  \texttt{'output='} {[} \emph{\texttt{'namedmat'}} \textbar{}
  \texttt{'numeric'} {]} - Output matrices \texttt{X} and \texttt{Y}
  will be either namedmat objects or plain numeric arrays; if the option
  \texttt{'select='} is used, \texttt{'output='} is always
  \texttt{'namedmat'}.
\item
  \texttt{'select='} {[} char \textbar{} cellstr {]} - Return FEVD for
  selected variables and/or shocks only; \texttt{Inf} means all
  variables. This option does not apply to the output databases,
  \texttt{A} and \texttt{B}.
\end{itemize}

\paragraph{Description}\label{description}

\paragraph{Example}\label{example}


