
    \foldertitle{model}{Model objects and functions}{model/Contents}

	Model objects are created by loading a \href{modellang/Contents}{model
file}. Once a model object exists, you can use model functions and
standard Matlab functions to write your own m-files to perform the
desired tasks, such calibrate or estimate the model, find its steady
state, solve and simulate it, produce forecasts, analyse its properties,
and so on.

Model methods:

\paragraph{Constructor}\label{constructor}

\begin{itemize}
\itemsep1pt\parskip0pt\parsep0pt
\item
  \href{model/model}{\texttt{model}} - Create new model object based on
  model file.
\end{itemize}

\paragraph{Getting information about
models}\label{getting-information-about-models}

\begin{itemize}
\itemsep1pt\parskip0pt\parsep0pt
\item
  \href{model/addparam}{\texttt{addparam}} - Add model parameters to a
  database (struct).
\item
  {[}\texttt{autocaption}{]} (model/autocaption) -
\item
  \href{model/autoexogenise}{\texttt{autoexogenise}} - Get or set
  variable/shock pairs for use in autoexogenised simulation plans.
\item
  \href{model/comment}{\texttt{comment}} - Get or set user comments in
  an IRIS object.
\item
  \href{model/eig}{\texttt{eig}} - Eigenvalues of the model transition
  matrix.
\item
  \href{model/findeqtn}{\texttt{findeqtn}} - Find equations by the
  labels.
\item
  \href{model/findname}{\texttt{findname}} - Find names of variables,
  shocks, or parameters by their descriptors.
\item
  \href{model/get}{\texttt{get}} - Query model object properties.
\item
  \href{model/iscompatible}{\texttt{iscompatible}} - True if two models
  can occur together on the LHS and RHS in an assignment.
\item
  \href{model/islinear}{\texttt{islinear}} - True for models declared as
  linear.
\item
  \href{model/islog}{\texttt{islog}} - True for log-linearised
  variables.
\item
  \href{model/isnan}{\texttt{isnan}} - Check for NaNs in model object.
\item
  \href{model/isname}{\texttt{isname}} - True for valid names of
  variables, parameters, or shocks in model object.
\item
  \href{model/issolved}{\texttt{issolved}} - True if a model solution
  exists.
\item
  \href{model/isstationary}{\texttt{isstationary}} - True if model or
  specified combination of variables is stationary.
\item
  \href{model/length}{\texttt{length}} - Number of alternative
  parameterisations.
\item
  \href{model/omega}{\texttt{omega}} - Get or set the covariance matrix
  of shocks.
\item
  \href{model/sspace}{\texttt{sspace}} - State-space matrices describing
  the model solution.
\item
  \href{model/system}{\texttt{system}} - System matrices before model is
  solved.
\item
  \href{model/userdata}{\texttt{userdata}} - Get or set user data in an
  IRIS object.
\end{itemize}

\paragraph{Referencing model objects}\label{referencing-model-objects}

\begin{itemize}
\itemsep1pt\parskip0pt\parsep0pt
\item
  \href{model/subsasgn}{\texttt{subsasgn}} - Subscripted assignment for
  model and systemfit objects.
\item
  \href{model/subsref}{\texttt{subsref}} - Subscripted reference for
  model and systemfit objects.
\end{itemize}

\paragraph{Changing model objects}\label{changing-model-objects}

\begin{itemize}
\itemsep1pt\parskip0pt\parsep0pt
\item
  \href{model/alter}{\texttt{alter}} - Expand or reduce number of
  alternative parameterisations.
\item
  \href{model/assign}{\texttt{assign}} - Assign parameters, steady
  states, std deviations or cross-correlations.
\item
  \href{model/export}{\texttt{export}} - Save carry-around files on the
  disk.
\item
  \href{model/horzcat}{\texttt{horzcat}} - Combine two compatible model
  objects in one object with multiple parameterisations.
\item
  \href{model/refresh}{\texttt{refresh}} - Refresh dynamic links.
\item
  {[}\texttt{reset}{]}{[}model/reset) -
\item
  \href{model/stdscale}{\texttt{stdscale}} - Re-scale all std deviations
  by the same factor.
\item
  \href{model/set}{\texttt{set}} - Change modifiable model object
  property.
\item
  \href{model/single}{\texttt{single}} - Convert solution matrices to
  single precision.
\end{itemize}

\paragraph{Steady state}\label{steady-state}

\begin{itemize}
\itemsep1pt\parskip0pt\parsep0pt
\item
  \href{model/chksstate}{\texttt{chksstate}} - Check if equations hold
  for currently assigned steady0state values.
\item
  \href{model/sstate}{\texttt{sstate}} - Compute steady state or
  balance-growth path of the model.
\item
  \href{model/sstatefile}{\texttt{sstatefile}} - Create a steady-state
  file based on the model object's steady-state equations.
\end{itemize}

\paragraph{Solution, simulation and
forecasting}\label{solution-simulation-and-forecasting}

\begin{itemize}
\itemsep1pt\parskip0pt\parsep0pt
\item
  \href{model/diffsrf}{\texttt{diffsrf}} - Differentiate shock response
  functions w.r.t. specified parameters.
\item
  \href{model/expand}{\texttt{expand}} - Compute forward expansion of
  model solution for anticipated shocks.
\item
  \href{model/jforecast}{\texttt{jforecast}} - Forecast with judgmental
  adjustments (conditional forecasts).
\item
  \href{model/icrf}{\texttt{icrf}} - Initial-condition response
  functions.
\item
  \href{model/lhsmrhs}{\texttt{lhsmrhs}} - Evaluate the discrepancy
  between the LHS and RHS for each model equation and given data.
\item
  \href{model/resample}{\texttt{resample}} - Resample from the model
  implied distribution.
\item
  \href{model/reporting}{\texttt{reporting}} - Run reporting equations.
\item
  \href{model/shockplot}{\texttt{shockplot}} - Short-cut for running and
  plotting plain shock simulation.
\item
  \href{model/simulate}{\texttt{simulate}} - Simulate model.
\item
  \href{model/solve}{\texttt{solve}} - Calculate first-order accurate
  solution of the model.
\item
  \href{model/srf}{\texttt{srf}} - Shock response functions.
\end{itemize}

\paragraph{Model data}\label{model-data}

\begin{itemize}
\itemsep1pt\parskip0pt\parsep0pt
\item
  \href{model/data4lhsmrhs}{\texttt{data4lhsmrhs}} - Prepare data array
  for running \texttt{lhsmrhs}.
\item
  \href{model/emptydb}{\texttt{emptydb}} - Create model-specific
  database with variables, shocks, and parameters.
\item
  \href{model/sstatedb}{\texttt{sstatedb}} - Create model-specific
  steady-state or balanced-growth-path database.
\item
  \href{model/zerodb}{\texttt{zerodb}} - Create model-specific
  zero-deviation database.
\end{itemize}

\paragraph{Stochastic properties}\label{stochastic-properties}

\begin{itemize}
\itemsep1pt\parskip0pt\parsep0pt
\item
  \href{model/acf}{\texttt{acf}} - Autocovariance and autocorrelation
  functions for model variables.
\item
  \href{model/ifrf}{\texttt{ifrf}} - Frequency response function to
  shocks.
\item
  \href{model/fevd}{\texttt{fevd}} - Forecast error variance
  decomposition for model variables.
\item
  \href{model/ffrf}{\texttt{ffrf}} - Filter frequency response function
  of transition variables to measurement variables.
\item
  \href{model/fmse}{\texttt{fmse}} - Forecast mean square error
  matrices.
\item
  \href{model/vma}{\texttt{vma}} - Vector moving average representation
  of the model.
\item
  \href{model/xsf}{\texttt{xsf}} - Power spectrum and spectral density
  of model variables.
\end{itemize}

\paragraph{Identification, estimation and
filtering}\label{identification-estimation-and-filtering}

\begin{itemize}
\itemsep1pt\parskip0pt\parsep0pt
\item
  \href{model/bn}{\texttt{bn}} - Beveridge-Nelson trends.
\item
  \href{model/diffloglik}{\texttt{diffloglik}} - Approximate gradient
  and hessian of log-likelihood function.
\item
  \href{model/estimate}{\texttt{estimate}} - Estimate model parameters
  by optimising selected objective function.
\item
  \href{model/evalsystempriors}{\texttt{evalsystempriors}} - Evaluate
  minus log of system prior density.
\item
  \href{model/filter}{\texttt{filter}} - Kalman smoother and estimator
  of out-of-likelihood parameters.
\item
  \href{model/fisher}{\texttt{fisher}} - Approximate Fisher information
  matrix in frequency domain.
\item
  \href{model/lognormal}{\texttt{lognormal}} - Characteristics of
  log-normal distributions returned by filter of forecast.
\item
  \href{model/loglik}{\texttt{loglik}} - Evaluate minus the
  log-likelihood function in time or frequency domain.
\item
  \href{model/neighbourhood}{\texttt{neighbourhood}} - Evaluate the
  local behaviour of the objective function around the estimated
  parameter values.
\item
  \href{model/regress}{\texttt{regress}} - Centred population regression
  for selected model variables.
\item
  \href{model/VAR}{\texttt{VAR}} - Population VAR for selected model
  variables.
\end{itemize}

\paragraph{Getting on-line help on model
functions}\label{getting-on-line-help-on-model-functions}

\begin{verbatim}
help model
help model/function_name
\end{verbatim}



