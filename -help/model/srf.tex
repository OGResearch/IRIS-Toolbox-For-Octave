

    \filetitle{srf}{Shock response functions}{model/srf}

	\paragraph{Syntax}

\begin{verbatim}
S = srf(M,NPer,...)
S = srf(M,Range,...)
\end{verbatim}

\paragraph{Input arguments}

\begin{itemize}
\item
  \texttt{M} {[} model {]} - Model object whose shock responses will be
  simulated.
\item
  \texttt{Range} {[} numeric {]} - Simulation date range with the first
  date being the shock date.
\item
  \texttt{NPer} {[} numeric {]} - Number of simulation periods.
\end{itemize}

\paragraph{Output arguments}

\begin{itemize}
\itemsep1pt\parskip0pt\parsep0pt
\item
  \texttt{S} {[} struct {]} - Database with shock response time series.
\end{itemize}

\paragraph{Options}

\begin{itemize}
\item
  \texttt{'delog='} {[} \emph{\texttt{true}} \textbar{} \texttt{false}
  {]} - Delogarithmise the responses for variables declared as
  \texttt{!variables:log}.
\item
  \texttt{'select='} {[} cellstr \textbar{} \emph{\texttt{Inf}} {]} -
  Run the shock response function for a selection of shocks only;
  \texttt{Inf} means all shocks are simulated.
\item
  \texttt{'size='} {[} \emph{`std'} \textbar{} numeric {]} - Size of the
  shocks that will be simulated; `std' means that each shock will be set
  to its std dev currently assigned in the model object \texttt{m}.
\end{itemize}

\paragraph{Description}

\paragraph{Example}


