

    \filetitle{xsf}{Power spectrum and spectral density of model variables}{model/xsf}

	\paragraph{Syntax}\label{syntax}

\begin{verbatim}
[S,D,List] = xsf(M,Freq,...)
[S,D,List,Freq] = xsf(M,NFreq,...)
\end{verbatim}

\paragraph{Input arguments}\label{input-arguments}

\begin{itemize}
\item
  \texttt{M} {[} model {]} - Model object.
\item
  \texttt{Freq} {[} numeric {]} - Vector of frequencies at which the
  XSFs will be evaluated.
\item
  \texttt{NFreq} {[} numeric {]} - Total number of requested
  frequencies; the frequencies will be evenly spread between 0 and
  \texttt{pi}.
\end{itemize}

\paragraph{Output arguments}\label{output-arguments}

\begin{itemize}
\item
  \texttt{S} {[} namedmat \textbar{} numeric {]} - Power spectrum
  matrices.
\item
  \texttt{D} {[} namedmat \textbar{} numeric {]} - Spectral density
  matrices.
\item
  \texttt{List} {[} cellstr {]} - List of variable in order of
  appearance in rows and columns of \texttt{S} and \texttt{D}.
\item
  \texttt{Freq} {[} numeric {]} - Vector of frequencies at which the
  XSFs has been evaluated.
\end{itemize}

\paragraph{Options}\label{options}

\begin{itemize}
\item
  \texttt{'applyTo='} {[} cellstr \textbar{} char \textbar{}
  \emph{\texttt{Inf}} {]} - List of variables to which the option
  \texttt{'filter='} will be applied; \texttt{Inf} means all variables.
\item
  \texttt{'filter='} {[} char \textbar{} \emph{empty} {]} - Linear
  filter that is applied to variables specified by `applyto'.
\item
  \texttt{'nFreq='} {[} numeric \textbar{} \emph{\texttt{256}} {]} -
  Number of equally spaced frequencies over which the `filter' is
  numerically integrated.
\item
  \texttt{'output='} {[} \emph{\texttt{'namedmat'}} \textbar{}
  \texttt{'numeric'} {]} - Output matrices \texttt{S} and \texttt{F}
  will be either namedmat objects or plain numeric arrays; if the option
  \texttt{'select='} is used, \texttt{'output='} is always a namedmat
  object.
\item
  \texttt{'progress='} {[} \texttt{true} \textbar{}
  \emph{\texttt{false}} {]} - Display progress bar on in the command
  window.
\item
  \texttt{'select='} {[} cellstr \textbar{} \emph{\texttt{Inf}} {]} -
  Return XSF for selected variables only; \texttt{Inf} means all
  variables.
\end{itemize}

\paragraph{Description}\label{description}

\paragraph{Example}\label{example}


