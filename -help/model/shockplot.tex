

    \filetitle{shockplot}{Short-cut for running and plotting plain shock simulation}{model/shockplot}

	\paragraph{Syntax}\label{syntax}

\begin{verbatim}
[S,FF,AA] = shockplot(M,ShockName,SimRange,PlotList,...)
\end{verbatim}

\paragraph{Input arguments}\label{input-arguments}

\begin{itemize}
\item
  \texttt{M} {[} model {]} - Model object that will be simulated.
\item
  \texttt{ShockName} {[} char {]} - Name of the shock that will be
  simulated.
\item
  \texttt{SimRange} {[} numeric {]} - Date range on which the shock will
  be simulated.
\item
  \texttt{PlotList} {[} cellstr {]} - List of variables that will be
  reported; you can use the syntax of
  \href{dbase/dbplot}{\texttt{dbase/dbplot}}.
\end{itemize}

\paragraph{Output arguments}\label{output-arguments}

\begin{itemize}
\item
  \texttt{S} {[} struct {]} - Database with simulation results.
\item
  \texttt{FF} {[} numeric {]} - Handles of figure windows created.
\item
  \texttt{AA} {[} numeric {]} - Handles of axes objects created.
\end{itemize}

\paragraph{Options affecting the
simulation}\label{options-affecting-the-simulation}

\begin{itemize}
\item
  \texttt{'deviation='} {[} \emph{\texttt{true}} \textbar{}
  \texttt{false} {]} - See the option \texttt{'deviation='} in
  \href{model/simulate}{\texttt{model/simulate}}.
\item
  \texttt{'dtrends='} {[} \emph{\texttt{@auto}} \textbar{} \texttt{true}
  \textbar{} \texttt{false} {]} - See the option \texttt{'dtrends='}
  option in \href{model/simulate}{\texttt{model/simulate}}.
\item
  \texttt{'shockSize='} {[} \emph{\texttt{'std'}} \textbar{} numeric {]}
  - Size of the shock that will be simulated; \texttt{'std'} means that
  one std dev of the shock will be simulated.
\end{itemize}

\paragraph{Options affecting the
graphs}\label{options-affecting-the-graphs}

See help on \href{dbase/dbplot}{\texttt{dbase/dbplot}} for other options
available.

\paragraph{Description}\label{description}

The simulated shock always occurs at time \texttt{t=1}. sStarting the
simulation range, \texttt{SimRange}, before \texttt{t=1} allows you to
simulate anticipated shocks.

The graphs automatically include one pre-sample period, i.e.~one period
prior to the start of the simulation.

\paragraph{Example}\label{example}


