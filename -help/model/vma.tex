

    \filetitle{vma}{Vector moving average representation of the model}{model/vma}

	\paragraph{Syntax}\label{syntax}

\begin{verbatim}
[Phi,List] = vma(M,P,...)
\end{verbatim}

\paragraph{Input arguments}\label{input-arguments}

\begin{itemize}
\item
  \texttt{M} {[} model {]} - Model object for which the VMA
  representation will be computed.
\item
  \texttt{P} {[} numeric {]} - Order up to which the VMA will be
  evaluated.
\end{itemize}

\paragraph{Output arguments}\label{output-arguments}

\begin{itemize}
\item
  \texttt{Phi} {[} namedmat \textbar{} numeric {]} - VMA matrices.
\item
  \texttt{List} {[} cell {]} - List of measurement and transition
  variables in the rows of the \texttt{Phi} matrix, and list of shocks
  in the columns of the \texttt{Phi} matrix.
\end{itemize}

\paragraph{Option}\label{option}

\begin{itemize}
\item
  \texttt{'matrixFmt='} {[} \emph{\texttt{'namedmat'}} \textbar{}
  \texttt{'plain'} {]} - Return matrix \texttt{Phi} as either a
  \href{namedmat/Contents}{\texttt{namedmat}} object (i.e.~matrix with
  named rows and columns) or a plain numeric array.
\item
  \texttt{'select='} {[} \emph{\texttt{@all}} \textbar{} char \textbar{}
  cellstr {]} - Return VMA for selected variables only; \texttt{@all}
  means all variables.
\end{itemize}

\paragraph{Description}\label{description}

\paragraph{Example}\label{example}


