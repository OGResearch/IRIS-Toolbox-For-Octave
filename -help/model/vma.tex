

    \filetitle{vma}{Vector moving average representation of the model}{model/vma}

	\paragraph{Syntax}

\begin{verbatim}
[Phi,List] = vma(M,P,...)
\end{verbatim}

\paragraph{Input arguments}

\begin{itemize}
\item
  \texttt{M} {[} model {]} - Model object for which the VMA
  representation will be computed.
\item
  \texttt{P} {[} numeric {]} - Order up to which the VMA will be
  evaluated.
\end{itemize}

\paragraph{Output arguments}

\begin{itemize}
\item
  \texttt{Phi} {[} namedmat \textbar{} numeric {]} - VMA matrices.
\item
  \texttt{List} {[} cell {]} - List of measurement and transition
  variables in the rows of the \texttt{Phi} matrix, and list of shocks
  in the columns of the \texttt{Phi} matrix.
\end{itemize}

\paragraph{Option}

\begin{itemize}
\item
  \texttt{'output='} {[} \emph{\texttt{'namedmat'}} \textbar{}
  \texttt{'numeric'} {]} - Output matrix \texttt{Phi} will be either a
  namedmat object or a plain numeric array; if the option
  \texttt{'select='} is used, \texttt{'output='} is always
  \texttt{'namedmat'}.
\item
  \texttt{'select='} {[} cellstr \textbar{} \emph{\texttt{Inf}} {]} -
  Return the VMA matrices for selected variabes and/or shocks only;
  \texttt{Inf} means all variables.
\end{itemize}

\paragraph{Description}

\paragraph{Example}


