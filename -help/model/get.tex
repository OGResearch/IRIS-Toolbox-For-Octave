

    \filetitle{get}{Query model object properties}{model/get}

	\paragraph{Syntax}\label{syntax}

\begin{verbatim}
Ans = get(M,Query)
[Ans,Ans,...] = get(M,Query,Query,...)
\end{verbatim}

\paragraph{Input arguments}\label{input-arguments}

\begin{itemize}
\item
  \texttt{M} {[} model {]} - Model object.
\item
  \texttt{Query} {[} char {]} - Query to the model object.
\end{itemize}

\paragraph{Output arguments}\label{output-arguments}

\begin{itemize}
\itemsep1pt\parskip0pt\parsep0pt
\item
  \texttt{Ans} {[} \ldots{} {]} - Answer to the query.
\end{itemize}

\paragraph{Valid queries to model
objects}\label{valid-queries-to-model-objects}

Below is the categorised list of model properties and attributes that
can be queried/accessed by the \texttt{get} function. Note that letter
\texttt{'y'} is used in various contexts to denote measurement variables
or equations, \texttt{'x'} transition variables or equations,
\texttt{'e'} shocks, \texttt{'p'} parameters, \texttt{'g'} exogenous
variables, \texttt{'d'} deterministic trend equations, \texttt{'l'}
dynamic links, and \texttt{'r'} reporting equations. The property names
are case insensitive.

\subparagraph{Steady state}\label{steady-state}

\begin{itemize}
\item
  \texttt{'sstate'} -- Returns {[} struct {]} a database with the steady
  states for all model variables. The steady states are described by
  complex numbers in which the real part is the level and the imaginary
  part is the growth rate.
\item
  \texttt{'sstateLevel'} -- Returns {[} struct {]} a database with the
  steady-state levels for all model variables.
\item
  \texttt{'sstateGrowth'} -- Returns {[} struct {]} a database with
  steady-state growth (first difference for linearised variables, gross
  rate of growth for log-linearised variables) for all model variables.
\item
  \texttt{'dtrends'} -- Returns {[} struct {]} a database with the
  effect of the deterministic trends on the measurement variables. The
  effect is described by complex numbers the same way as the steady
  state.
\item
  \texttt{'dtrendsLevel'} -- Returns {[} struct {]} a database with the
  effect of the deterministic trends on the steady-state levels of the
  measurement variables.
\item
  \texttt{'dtrendsGrowth'} -- Returns {[} struct {]} a database with the
  effect of deterministic trends on steady-state growth of the
  measurement variables.
\item
  \texttt{'sstate+dtrends'} -- Returns {[} struct {]} the same as
  `sstate' except that the measurement variables are corrected for the
  effect of the deterministic trends.
\item
  \texttt{'sstateLevel+dtrendsLevel'} -- Returns {[} struct {]} the same
  as `sstateLevel' except that the measurement variables are corrected
  for the effect of the deterministic trends.
\item
  \texttt{'sstateGrowth+dtrendsGrowth'} -- Returns {[} struct {]} the
  same as \texttt{'sstateGrowth'} except that the measurement variables
  are corrected for the effect of the deterministic trends.
\end{itemize}

\subparagraph{Variables, shocks, and
parameters}\label{variables-shocks-and-parameters}

\begin{itemize}
\item
  \texttt{'yList'}, \texttt{'xList'}, \texttt{'eList'},
  \texttt{'pList'}, \texttt{'gList'} - Return {[} cellstr {]} the lists
  of, respectively, measurement variables (\texttt{y}), transition
  variables (\texttt{x}), shocks (\texttt{e}), parameters (\texttt{p}),
  and exogenous variables (\texttt{g}), each in order of appearance of
  the names in declaration sections of the original model file. Note
  that the list of parameters, \texttt{'pList'}, does not include the
  names of std deviations or cross-correlations.
\item
  \texttt{'eyList'} -- Returns {[} cellstr {]} the list of measurement
  shocks in order of their appearance in the model code declarations;
  only those shocks that actually occur in at least one measurement
  equation are returned.
\item
  \texttt{'exList'} -- Returns {[} cellstr {]} the list of transition
  shocks in order of their appearance in the model code declarations;
  only those shocks that actually occur in at least one transition
  equation are returned.
\item
  \texttt{'stdList'} -- Returns {[} cellstr {]} the list of the names of
  the standard deviations for the shocks in order of the appearance of
  the corresponding shocks in the model code.
\item
  \texttt{'corrList'} -- Returns {[} cellstr {]} the list of the names
  of cross-correlation coefficients for the shocks in order of the
  appearance of the corresponding shocks in the model code.
\item
  \texttt{'stdCorrList'} -- Returns {[} cellstr {]} the list of the
  names of std deviations and cross-correlation coefficients for the
  shocks in order of the appearance of the corresponding shocks in the
  model code.
\end{itemize}

\subparagraph{Equations}\label{equations}

\begin{itemize}
\item
  \texttt{'yEqtn'}, \texttt{'xEqtn'}, \texttt{'dEqtn'},
  \texttt{'lEqtn'}, \texttt{'rEqtn'} - Return {[} cellstr {]} the lists
  of, respectively, to measurement equations (\texttt{y}), transition
  equations (\texttt{x}), deterministic trends (\texttt{d}), dynamic
  links (\texttt{l}), and reporting equations (\texttt{r}), each in
  order of appearance in the original model file.
\item
  \texttt{'links'} -- Returns {[} struct {]} a database with the dynamic
  links with fields names after the LHS name.
\end{itemize}

\subparagraph{First-order Taylor expansion of
equations}\label{first-order-taylor-expansion-of-equations}

\begin{itemize}
\item
  \texttt{'derivatives'} -- Returns {[} cellstr {]} the
  symbolic/automatic derivatives for each model equation; in each
  equation, the derivatives w.r.t. all variables present in that
  equation are evaluated at once and returned as a vector of numbers;
  see also \texttt{'wrt'}.
\item
  \texttt{'wrt'} - Returns {[} cellstr {]} the list of the variables
  (and their auxiliary lags or leads) with respect to which the
  corresponding equation in \texttt{'derivatives'} is differentiated.
\end{itemize}

\subparagraph{Descriptions and aliases of variables, parameters, and
shocks}\label{descriptions-and-aliases-of-variables-parameters-and-shocks}

\begin{itemize}
\item
  \texttt{'descript'} -- Returns {[} struct {]} a database with user
  descriptions of model variables, shocks, and parameters.
\item
  \texttt{'yDescript'}, \texttt{'xDescript'}, \texttt{'eDescript'},
  \texttt{'pDescript'}, \texttt{'gDescript'} - Return {[} cellstr {]}
  user descriptions of, respectively, measurement variables
  (\texttt{y}), transition variables (\texttt{x}), shocks (\texttt{e}),
  parameters (\texttt{p}), and exogenous variables (\texttt{g}).
\item
  \texttt{'alias'} -- Returns {[} struct {]} a database with all aliases
  of model variables, shocks, and parameters.
\item
  \texttt{'yAlias'}, \texttt{'xAlias'}, \texttt{'eAlias'},
  \texttt{'pAlias'}, \texttt{'gAlias'} - Return {[} cellstr {]} the
  aliases of, respectively, measurement variables (\texttt{y}),
  transition variables (\texttt{x}), shocks (\texttt{e}), parameters
  (\texttt{p}), and exogenous variables (\texttt{g}).
\end{itemize}

\subparagraph{Equation labels and
aliases}\label{equation-labels-and-aliases}

\begin{itemize}
\item
  \texttt{'labels'} -- Returns {[} cellstr {]} the list of all user
  labels added to equations.
\item
  \texttt{'yLabels'}, \texttt{'xLabels'}, \texttt{'dLabels'},
  \texttt{'lLabels'}, \texttt{'rLabels'} - Return {[} cellstr {]} user
  labels added, respectively, to measurement equations (\texttt{y}),
  transition equations (\texttt{x}), deterministic trends (\texttt{d}),
  dynamic links (\texttt{l}), and reporting equations (\texttt{r}).
\item
  \texttt{'eqtnAlias'} -- Returns {[} cellstr {]} the list of all
  aliases added to equations.
\item
  \texttt{'yEqtnAlias'}, \texttt{'xEqtnAlias'}, \texttt{'dEqtnAlias'},
  \texttt{'lEqtnAlias'}, \texttt{'rEqtnAlias'} - Return {[} cellstr {]}
  the aliases of, respectively, measurement equations (\texttt{y}),
  transition equations (\texttt{x}), deterministic trends (\texttt{d}),
  dynamic links (\texttt{l}), and reporting equations (\texttt{r}).
\end{itemize}

\subparagraph{Parameter values}\label{parameter-values}

\begin{itemize}
\item
  \texttt{'corr'} -- Returns {[} struct {]} a database with current
  cross-correlation coefficients of shocks.
\item
  \texttt{'nonzeroCorr'} -- Returns {[} struct {]} a database with
  current nonzero cross-correlation coefficients of shocks.
\item
  \texttt{'parameters'} -- Returns {[} struct {]} a database with
  current parameter values, including the std devs and non-zero corr
  coefficients.
\item
  \texttt{'std'} -- Returns {[} struct {]} a database with current std
  deviations of shocks.
\end{itemize}

\subparagraph{Eigenvalues}\label{eigenvalues}

\begin{itemize}
\item
  \texttt{'stableRoots'} -- Returns {[} cell of numeric {]} a vector of
  the model eigenvalues that are smaller than one in magnitude (allowing
  for rounding errors around one).
\item
  \texttt{'unitRoots'} -- Returns {[} cell of numeric {]} a vector of
  the model eigenvalues that equal one in magnitude (allowing for
  rounding errors around one).
\item
  \texttt{'unstableRoots'} {[} cell of numeric {]} A vector of the model
  eigenvalues that are greater than one in magnitude (allowing for
  rounding errors around one).
\end{itemize}

\subparagraph{Model structure, solution,
build}\label{model-structure-solution-build}

\begin{itemize}
\item
  \texttt{'build'} -- Returns {[} numeric {]} IRIS version number under
  which the model object has been built.
\item
  \texttt{'log'} -- Returns {[} struct {]} a database with \texttt{true}
  for each log-linearised variables, and \texttt{false} for each
  linearised variable.
\item
  \texttt{'maxLag'} -- Returns {[} numeric {]} the maximum lag in the
  model.
\item
  \texttt{'maxLead'} -- Returns {[} numeric {]} the maximum lead in the
  model.
\item
  \texttt{'stationary'} -- Returns {[} struct {]} a database with
  \texttt{true} for each stationary variables, and \texttt{false} for
  each unit-root (non-stationary) variables (under current solution).
\item
  \texttt{'nonStationary'} -- Returns {[} struct {]} a database with
  \texttt{true} for each unit-root (non-stationary) varible, and
  \texttt{false} for each stationary variable (under current solution).
\item
  \texttt{'stationaryList'} -- Returns {[} cellstr {]} the list of
  stationary variables (under current solution).
\item
  \texttt{'nonStationaryList'} -- Returns {[} cellstr {]} cell with the
  list of unit-root (non-stationary) variables (under current solution).
\item
  \texttt{'initCond'} -- Returns {[} cellstr {]} the list of the lagged
  transition variables that need to be supplied as initial conditions in
  simulations and forecasts. The list of the initial conditions is
  solution-specific as the state-spece coefficients at some of the lags
  may evaluate to zero depending on the current parameters.
\item
  \texttt{'yVector'} -- Returns {[} cellstr {]} the list of measurement
  variables in order of their appearance in the rows and columns of
  state-space matrices (effectively identical to \texttt{'yList'}) from
  the \href{model/sspace}{\texttt{model/sspace}} function.
\item
  \texttt{'xVector'} -- Returns {[} cellstr {]} the list of transition
  variables, and their auxiliary lags and leads, in order of their
  appearance in the rows and columns of state-space matrices from the
  \href{model/sspace}{\texttt{model/sspace}} function.
\item
  \texttt{'xfVector'} -- Returns {[} cellstr {]} the list of
  forward-looking (i.e.~non-predetermined) transition variables, and
  their auxiliary lags and leads, in order of their appearance in the
  rows and columns of state-space matrices from the
  \href{model/sspace}{\texttt{model/sspace}} function.
\item
  \texttt{'xbVector'} -- Returns {[} cellstr {]} the list of
  backward-looking (i.e.~predetermined) transition variables, and their
  auxiliary lags and leads, in order of their appearance in the rows and
  columns of state-space matrices from the
  \href{model/sspace}{\texttt{model/sspace}} function.
\item
  \texttt{'eVector'} -- Returns {[} cellstr {]} the list of the shocks
  in order of their appearance in the rows and columns of state-space
  matrices (effectively identical to \texttt{'eList'}) from the
  \href{model/sspace}{\texttt{model/sspace}} function.
\end{itemize}

\paragraph{Description}\label{description}

\subparagraph{First-order Taylor expansion of
equations}\label{first-order-taylor-expansion-of-equations-1}

The expressions for symbolic/automatic derivatives of individual model
equations returned by \texttt{'derivatives'} are expressions that
evaluate the derivatives with respect to all variables present in that
equation at once. The list of variables with respect to which each
equation is differentiated is returned by \texttt{'wrt'}.

The expressions returned by the query \texttt{'derivatives'} can refer
to

\begin{itemize}
\itemsep1pt\parskip0pt\parsep0pt
\item
  the names of model parameters, such as \texttt{alpha};
\item
  the names of transition or measurement variables, such as \texttt{X};
\item
  the lags or leads of variables, such as \texttt{X\{-1\}} or
  \texttt{X\{2\}}.
\end{itemize}

Note that the lags and leads of variables must be, in general, preserved
in the derivatives for non-stationary (unit-root) models. For stationary
models, the lags and leads can be removed and each simply replaced with
the current date of the respective variable.

\paragraph{Example}\label{example}

\begin{verbatim}
d = get(m,'derivatives');
w = get(m,'wrt');
\end{verbatim}

The 1-by-N cell array \texttt{d} (where N is the total number of
equations in the model) will contain expressions that evaluate to the
vector of derivatives of the individual equations w.r.t. to the
variables present in that equation:

\begin{verbatim}
d{k}
\end{verbatim}

is an expression that returns, in general, a vector of M numbers. These
M numbers are the derivatives of the k-th equation w.r.t to M variables
whose list is in

\begin{verbatim}
w{k}
\end{verbatim}


