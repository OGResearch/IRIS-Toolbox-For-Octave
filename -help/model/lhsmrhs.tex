

    \filetitle{lhsmrhs}{Evaluate the discrepancy between the LHS and RHS for each model equation and given data}{model/lhsmrhs}

	\paragraph{Syntax for casual
evaluation}\label{syntax-for-casual-evaluation}

\begin{verbatim}
Q = lhsmrhs(M,D,Range)
\end{verbatim}

\paragraph{Syntax for fast evaluation}\label{syntax-for-fast-evaluation}

\begin{verbatim}
Q = lhsmrhs(M,YXE)
\end{verbatim}

\paragraph{Input arguments}\label{input-arguments}

\texttt{M} {[} model {]} - Model object whose equations and currently
assigned parameters will be evaluated.

\texttt{YXE} {[} numeric {]} - Numeric array created from an input
database by calling the function
\href{model/data4lhsmrhs}{\texttt{data4lhsmrhs}}; \texttt{YXE} contains
the observations on the measurement variables, transition variables, and
shocks organised row-wise.

\begin{itemize}
\item
  \texttt{D} {[} struct {]} - Input database with observations on
  measurement variables, transition variables, and shocks on which the
  discrepancies will be evaluated.
\item
  \texttt{Range} {[} numeric {]} - Date range on which the discrepancies
  will be evaluated.
\end{itemize}

\paragraph{Output arguments}\label{output-arguments}

\texttt{Q} {[} numeric {]} - Numeric array with discrepancies between
the LHS and RHS for each model equation.

\paragraph{Description}\label{description}

The function \texttt{lhsmrhs} evaluates the discrepancy between the LHS
and the RHS in each model equation; each lead is replaced with the
actual observation supplied in the input data. The function
\texttt{lhsmrhs} does not work for models with
\href{modellang/sstateref}{references to steady state values}.

The first syntax, with the array \texttt{YXE} pre-built in a call to
\href{model/data4lhsmrhs}{\texttt{data4lhsmrhs}} is computationally much
more efficient if you need to evaluate the LHS-RHS discrepancies
repeatedly for different parameterisations.

The output argument \texttt{D} is an \texttt{nEqtn} by \texttt{nPer} by
\texttt{nAlt} array, where \texttt{nEqnt} is the number of measurement
and transition equations, \texttt{nPer} is the number of periods used to
create \texttt{X} in a prior call to
\href{model/data4lhsmrhs}{\texttt{data4lhsmrhs}}, and \texttt{nAlt} is
the greater of the number of alternative parameterisations in
\texttt{M}, and the number of alternative datasets in the input data.

\paragraph{Example}\label{example}

\begin{verbatim}
YXE = data4lhsmrhs(M,d,range);
Q = lhsmrhs(M,YXE);
\end{verbatim}


