

    \filetitle{sstatefile}{Create a steady-state file based on the model object's steady-state equations}{model/sstatefile}

	\paragraph{Syntax}

\begin{verbatim}
sstatefile(m,filename,...)
\end{verbatim}

\paragraph{Input arguments}

\begin{itemize}
\item
  \texttt{m} {[} model {]} - Model object.
\item
  \texttt{file} {[} char {]} - Filename under which the steady-state
  file will be saved.
\end{itemize}

\paragraph{Options}

\begin{itemize}
\item
  \texttt{'endogenise='} {[} cellstr \textbar{} char \textbar{}
  \emph{empty} {]} - List of parameters that will be endogenised when
  computing the steady state; the number of endogenised parameters must
  match the number of transtion variables exogenised in the
  \texttt{'exogenised='} option.
\item
  \texttt{'endogenise='} {[} cellstr \textbar{} char \textbar{}
  \emph{empty} {]} - List of transition variables that will be
  exogenised when computing the steady state; the number of exogenised
  variables must match the number of parameters exogenised in the
  \texttt{'exogenise='} option.
\item
  \texttt{'growthNames='} {[} char \textbar{} \emph{\texttt{'d?'}} {]} -
  Template for growth names used in evaluating lags and leads.
\item
  \texttt{'time='} {[} \emph{\texttt{true}} \textbar{} \texttt{false}
  {]} - Keep or remove time subscripts (curly braces) in the
  steady-state file.
\end{itemize}

\paragraph{Description}

\paragraph{Example}


