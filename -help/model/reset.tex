

    \filetitle{reset}{Reset specific values within model object}{model/reset}

	\paragraph{Syntax}

\begin{verbatim}
M = reset(M)
M = reset(M,Req1,Req2,...)
\end{verbatim}

\paragraph{Input arguments}

\begin{itemize}
\item
  \texttt{M} {[} model {]} - Model object in which the requested type(s)
  of values will be reset.
\item
  \texttt{Req1}, \texttt{Req2}, \ldots{} {[} \texttt{'corr'} \textbar{}
  \texttt{'parameters'} \textbar{} \texttt{'sstate'} \textbar{}
  \texttt{'std'} \textbar{} \texttt{'stdcorr'} {]} - Requested type(s)
  of values that will be reset; if omitted, everything will be reset.
\end{itemize}

\paragraph{Output arguments}

\begin{itemize}
\itemsep1pt\parskip0pt\parsep0pt
\item
  \texttt{M} {[} model {]} - Model object with the requested values
  reset.
\end{itemize}

\paragraph{Description}

\begin{itemize}
\item
  \texttt{'corr'} - All cross-correlation coefficients will be reset to
  \texttt{0}.
\item
  \texttt{'parameters'} - All parameters will be reset to \texttt{NaN}.
\item
  \texttt{'sstate'} - All steady state values will be reset to
  \texttt{NaN}.
\item
  \texttt{'std'} - All std deviations will be reset to \texttt{1} (in
  linear models) or \texttt{log(1.01)} (in non-linear models).
\item
  \texttt{'stdcorr'} - Equivalent to \texttt{'std'} and \texttt{'corr'}.
\end{itemize}

\paragraph{Example}

-IRIS Toolbox. -Copyright (c) 2007-2014 IRIS Solutions Team.


