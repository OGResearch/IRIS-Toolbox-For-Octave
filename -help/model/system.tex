

    \filetitle{system}{System matrices for unsolved model}{model/system}

	\paragraph{Syntax}

\begin{verbatim}
[A,B,C,D,F,G,H,J,List,Nf] = system(M)
\end{verbatim}

\paragraph{Input arguments}

\begin{itemize}
\itemsep1pt\parskip0pt\parsep0pt
\item
  \texttt{M} {[} model {]} - Model object whose system matrices will be
  returned.
\end{itemize}

\paragraph{Output arguments}

\begin{itemize}
\item
  \texttt{A}, \texttt{B}, \texttt{C}, \texttt{D}, \texttt{F},
  \texttt{G}, \texttt{H} ,\texttt{J} {[} numeric {]} - Matrices
  describing the unsolved system, see Description.
\item
  \texttt{List} {[} cell {]} - Lists of measurement variables,
  transition variables includint their auxiliary lags and leads, and
  shocks as they appear in the rows and columns of the system matrices.
\item
  \texttt{Nf} {[} numeric {]} - Number of non-predetermined
  (forward-looking) transition variables (multiplied by the first
  \texttt{Nf} columns of matrices \texttt{A} and \texttt{B}).
\end{itemize}

\paragraph{Options}

\begin{itemize}
\item
  \texttt{'linear='} {[} \emph{\texttt{'auto'}} \textbar{} \texttt{true}
  \textbar{} \texttt{false} {]} - Compute the model using a linear
  approach, i.e.~differentiating around zero and not the currently
  assigned steady state.
\item
  \texttt{'select='} {[} \emph{\texttt{true}} \textbar{} \texttt{false}
  {]} - Automatically detect which equations need to be
  re-differentiated based on parameter changes from the last time the
  system matrices were calculated.
\item
  \texttt{'sparse='} {[} \texttt{true} \textbar{} \emph{\texttt{false}}
  {]} - Return matrices \texttt{A}, \texttt{B}, \texttt{D}, \texttt{F},
  \texttt{G}, and \texttt{J} as sparse matrices; can be set to
  \texttt{true} only in models with one parameterization.
\end{itemize}

\paragraph{Description}

The system before the model is solved has the following form:

\begin{verbatim}
A E[xf;xb] + B [xf(-1);xb(-1)] + C + D e = 0

F y + G xb + H + J e = 0
\end{verbatim}

where \texttt{E} is a conditional expectations operator, \texttt{xf} is
a vector of non-predetermined (forward-looking) transition variables,
\texttt{xb} is a vector of predetermined (backward-looking) transition
variables, \texttt{y} is a vector of measurement variables, and
\texttt{e} is a vector of transition and measurement shocks.

\paragraph{Example}


