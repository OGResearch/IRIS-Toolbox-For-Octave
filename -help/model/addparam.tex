

    \filetitle{addparam}{Add model parameters to a database (struct)}{model/addparam}

	\paragraph{Syntax}
 
 \begin{verbatim}
 D = addparam(M,D)
 \end{verbatim}
 
 \paragraph{Input arguments}
 
 \begin{itemize}
 \item
   \texttt{M} {[} model {]} - Model object whose parameters will be added
   to database (struct) \texttt{D}.
 \item
   \texttt{D} {[} struct {]} - Database to which the model parameters
   will be added.
 \end{itemize}
 
 \paragraph{Output arguments}
 
 \begin{itemize}
 \item
   `D {[} struct {]} - Database with the model parameters added.
 \end{itemize}
 
 \paragraph{Description}
 
 If there are database entries in \texttt{D} whose names conincide with
 the model parameters, they will be overwritten.
 
 \paragraph{Example}
 
 \begin{verbatim}
 D = struct();
 D = addparam(M,D);
 \end{verbatim}


