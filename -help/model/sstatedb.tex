

    \filetitle{sstatedb}{Create model-specific steady-state or balanced-growth-path database}{model/sstatedb}

	\paragraph{Syntax}
 
 \begin{verbatim}
 [D,IsDev] = sstatedb(M,Range)
 [D,IsDev] = sstatedb(M,Range,NCol)
 \end{verbatim}
 
 \paragraph{Input arguments}
 
 \begin{itemize}
 \item
   \texttt{M} {[} model {]} - Model object for which the sstate database
   will be created.
 \item
   \texttt{Range} {[} numeric {]} - Intended simulation range; the
   steady-state or balanced-growth-path database will be created on a
   range that also automatically includes all the necessary lags.
 \item
   \texttt{NCol} {[} numeric {]} - Number of columns for each variable;
   the input argument \texttt{NCol} can be only used with
   single-parameterisation models.
 \end{itemize}
 
 \paragraph{Output arguments}
 
 \begin{itemize}
 \item
   \texttt{D} {[} struct {]} - Database with a steady-state or
   balanced-growth path tseries object for each model variable, and a
   scalar or vector of the currently assigned values for each model
   parameter.
 \item
   \texttt{IsDev} {[} \texttt{false} {]} - The second output argument is
   always \texttt{false}, and can be used to set the option
   \texttt{'deviation='} in \url{model/simulate}.
 \end{itemize}
 
 \paragraph{Description}
 
 \paragraph{Example}


