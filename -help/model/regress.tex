

    \filetitle{regress}{Centred population regression for selected model variables}{model/regress}

	\paragraph{Syntax}
 
 \begin{verbatim}
 [B,CovRes,R2] = regress(M,Lhs,Rhs,...)
 \end{verbatim}
 
 \paragraph{Input arguments}
 
 \begin{itemize}
 \item
   \texttt{M} {[} model {]} - Model on whose covariance matrices the
   popolation regression will be based.
 \item
   \texttt{Lhs} {[} char \textbar{} cellstr {]} - Lhs variables in the
   regression; each of the variables must be part of the state-space
   vector.
 \item
   \texttt{Rhs} {[} char \textbar{} cellstr {]} - Rhs variables in the
   regression; each of the variables must be part of the state-space
   vector, or must refer to a larger lag of a transition variable present
   in the state-space vector.
 \end{itemize}
 
 \paragraph{Output arguments}
 
 \begin{itemize}
 \item
   \texttt{B} {[} namedmat \textbar{} numeric {]} - Population regression
   coefficients.
 \item
   \texttt{CovRes} {[} namedmat \textbar{} numeric {]} - Covariance
   matrix of residuals from the population regression.
 \item
   \texttt{R2} {[} numeric {]} - Coefficient of determination
   (R-squared).
 \end{itemize}
 
 \paragraph{Options}
 
 \begin{itemize}
 \item
   \texttt{'output='} {[} \emph{\texttt{'namedmat'}} \textbar{}
   \texttt{'numeric'} {]} - Output matrices will be either namedmat
   objects or plain numeric arrays.
 \end{itemize}
 
 \paragraph{Description}
 
 Population regressions calculated by this function are always centred.
 This means the regressions are always calculated as if estimated on
 observations with their uncondional means (the steady-state levels)
 removed from them.
 
 The Lhs and Rhs variables that are log-variables must include
 \texttt{log(...)} explicitly in their names. For instance, if \texttt{X}
 is declared to be a log variable, then you must refer to \texttt{log(X)}
 or \texttt{log(X\{-1\})}.
 
 \paragraph{Example}
 
 \begin{verbatim}
 [B,C] = regress('log(R)',{'log(R{-1})','log(dP)'});
 \end{verbatim}


