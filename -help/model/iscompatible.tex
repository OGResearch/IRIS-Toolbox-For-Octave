

    \filetitle{iscompatible}{True if two models can occur together on the LHS and RHS in an assignment}{model/iscompatible}

	\paragraph{Syntax}

\begin{verbatim}
Flag = iscompatible(M1,M2)
\end{verbatim}

\paragraph{Input arguments}

\begin{itemize}
\itemsep1pt\parskip0pt\parsep0pt
\item
  \texttt{M1}, \texttt{M2} {[} model {]} - Two model objects that will
  be tested for compatibility.
\end{itemize}

\paragraph{Output arguments}

\begin{itemize}
\itemsep1pt\parskip0pt\parsep0pt
\item
  \texttt{Flag} {[} \texttt{true} \textbar{} \texttt{false} {]} - True
  if \texttt{M1} and \texttt{M1} can occur in an assignment,
  \texttt{M1(...) = M2(...)} or horziontal concatenation,
  \texttt{{[}M1,M2{]}}.
\end{itemize}

\paragraph{Description}

The function compares the names of all variables, shocks, and
parameters, and the composition of the state-space vectors.

\paragraph{Example}


