

    \filetitle{solve}{Calculate first-order accurate solution of the model}{model/solve}

	\paragraph{Syntax}

\begin{verbatim}
M = solve(M,...)
\end{verbatim}

\paragraph{Input arguments}

\begin{itemize}
\itemsep1pt\parskip0pt\parsep0pt
\item
  \texttt{M} {[} model {]} - Paramterised model object. Non-linear
  models must also have a steady state values assigned.
\end{itemize}

\paragraph{Output arguments}

\begin{itemize}
\itemsep1pt\parskip0pt\parsep0pt
\item
  \texttt{M} {[} model {]} - Model with newly computed solution.
\end{itemize}

\paragraph{Options}

\begin{itemize}
\item
  \texttt{'expand='} {[} numeric \textbar{} \emph{\texttt{0}} \textbar{}
  \texttt{NaN} {]} - Number of periods ahead up to which the model
  solution will be expanded; if \texttt{NaN} the matrices needed to
  support solution expansion are not calculated and stored at all and
  the model cannot be used later in simulations or forecasts with
  anticipated shocks or plans.
\item
  \texttt{'error='} {[} \texttt{true} \textbar{} \emph{\texttt{false}}
  {]} - Throw an error if no unique stable solution exists; if
  \texttt{false}, a warning message only will be displayed.
\item
  \texttt{'linear='} {[} \emph{\texttt{'auto'}} \textbar{} \texttt{true}
  \textbar{} \texttt{false} {]} - Solve the model using a linear
  approach, i.e.~differentiating around zero and not the currently
  assigned steady state.
\item
  \texttt{'progress='} {[} \texttt{true} \textbar{}
  \emph{\texttt{false}} {]} - Display progress bar in the command
  window.
\item
  \texttt{'refresh='} {[} \emph{\texttt{true}} \textbar{} \texttt{false}
  {]} - Refresh dynamic links before computing the solution.
\item
  \texttt{'select='} {[} \emph{\texttt{true}} \textbar{} \texttt{false}
  {]} - Automatically detect which equations need to be
  re-differentiated based on parameter changes from the last time the
  system matrices were calculated.
\item
  \texttt{'warning='} {[} \emph{\texttt{true}} \textbar{} \texttt{false}
  {]} - Display warnings produced by this function.
\end{itemize}

\paragraph{Description}

The IRIS solver uses an ordered QZ (or generalised Schur) decomposition
to integrate out future expectations. The QZ may (very rarely) fail for
numerical reasons. IRIS includes two patches to handle the some of the
QZ failures: a SEVN2 patch (Sum-of-EigenValues-Near-Two), and an E2C2S
patch (Eigenvalues-Too-Close-To-Swap).

\begin{itemize}
\item
  The SEVN2 patch: The model contains two or more unit roots, and the QZ
  algorithm interprets some of them incorrectly as pairs of eigenvalues
  that sum up accurately to 2, but with one of them significantly below
  1 and the other significantly above 1. IRIS replaces the entries on
  the diagonal of one of the QZ factor matrices with numbers that
  evaluate to two unit roots.
\item
  The E2C2S patch: The re-ordering of thq QZ matrices fails with a
  warning
  \texttt{'Reordering failed because some eigenvalues are too close to swap.'}
  IRIS attempts to re-order the equations until QZ works. The number of
  attempts is limited to \texttt{N-1} at most where \texttt{N} is the
  total number of equations.
\end{itemize}

\paragraph{Example}


