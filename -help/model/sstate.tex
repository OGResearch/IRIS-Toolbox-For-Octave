

    \filetitle{sstate}{Compute steady state or balance-growth path of the model}{model/sstate}

	\paragraph{Syntax}
 
 \begin{verbatim}
 M = sstate(M,...)
 \end{verbatim}
 
 \paragraph{Input arguments}
 
 \begin{itemize}
 \item
   \texttt{M} {[} model {]} - Parameterised model object.
 \end{itemize}
 
 \paragraph{Output arguments}
 
 \begin{itemize}
 \item
   \texttt{M} {[} model {]} - Model object with newly computed steady
   state assigned.
 \end{itemize}
 
 \paragraph{Options}
 
 \begin{itemize}
 \item
   \texttt{'linear='} {[} \emph{\texttt{'auto'}} \textbar{} \texttt{true}
   \textbar{} \texttt{false} {]} - Solve for steady state using a linear
   approach, i.e.~based on the first-order solution matrices and the
   vector of constants.
 \item
   \texttt{'warning='} {[} \emph{\texttt{true}} \textbar{} \texttt{false}
   {]} - Display IRIS warning produced by this function.
 \end{itemize}
 
 \subparagraph{Options for non-linear models}
 
 \begin{itemize}
 \item
   \texttt{'blocks='} {[} \texttt{true} \textbar{} \emph{\texttt{false}}
   {]} - Re-arrarnge steady-state equations in recursive blocks before
   computing steady state.
 \item
   \texttt{'display='} {[} \emph{\texttt{'iter'}} \textbar{}
   \texttt{'final'} \textbar{} \texttt{'notify'} \textbar{}
   \texttt{'off'} {]} - Level of screen output, see Optim Tbx.
 \item
   \texttt{'endogenise='} {[} cellstr \textbar{} char \textbar{}
   \emph{empty} {]} - List of parameters that will be endogenised when
   computing the steady state; the number of endogenised parameters must
   match the number of transtion variables exogenised in the
   \texttt{'exogenised='} option.
 \item
   \texttt{'exogenise='} {[} cellstr \textbar{} char \textbar{}
   \emph{empty} {]} - List of transition variables that will be
   exogenised when computing the steady state; the number of exogenised
   variables must match the number of parameters exogenised in the
   \texttt{'exogenise='} option.
 \item
   \texttt{'fix='} {[} cellstr \textbar{} \emph{empty} {]} - List of
   variables whose steady state will not be computed and kept fixed to
   the currently assigned values.
 \item
   \texttt{'fixAllBut='} {[} cellstr \textbar{} \emph{empty} {]} -
   Inverse list of variables whose steady state will not be computed and
   kept fixed to the currently assigned values.
 \item
   \texttt{'fixGrowth='} {[} cellstr \textbar{} \emph{empty} {]} - List
   of variables whose steady-state growth will not be computed and kept
   fixed to the currently assigned values.
 \item
   \texttt{'fixGrowthAllBut='} {[} cellstr \textbar{} \emph{empty} {]} -
   Inverse list of variables whose steady-state growth will not be
   computed and kept fixed to the currently assigned values.
 \item
   \texttt{'fixLevel='} {[} cellstr \textbar{} \emph{empty} {]} - List of
   variables whose steady-state levels will not be computed and kept
   fixed to the currently assigned values.
 \item
   \texttt{'fixLevelAllBut='} {[} cellstr \textbar{} \emph{empty} {]} -
   Inverse list of variables whose steady-state levels will not be
   computed and kept fixed to the currently assigned values.
 \item
   \texttt{'growth='} {[} \texttt{true} \textbar{} \emph{\texttt{false}}
   {]} - If \texttt{true}, both the steady-state levels and growth rates
   will be computed; if \texttt{false}, only the levels will be computed
   assuming that the model is either stationary or that the correct
   steady-state growth rates are already assigned in the model object.
 \item
   \texttt{'optimSet='} {[} cell \textbar{} \emph{empty} {]} - Name-value
   pairs with Optim Tbx settings; see \texttt{help optimset} for details
   on these settings.
 \item
   \texttt{'refresh='} {[} \emph{\texttt{true}} \textbar{} \texttt{false}
   {]} - Refresh dynamic links after steady state is computed.
 \item
   \texttt{'reuse='} {[} \texttt{true} \textbar{} \emph{\texttt{false}}
   {]} - Reuse the steady-state values calculated for a parameterisation
   to initialise the next parameterisation.
 \item
   \texttt{'solver='} {[} \texttt{'fsolve'} \textbar{}
   \emph{\texttt{'lsqnonlin'}} {]} - Solver function used to solve for
   the steady state of non-linear models; it can be either of the two
   Optimization Tbx functions, or a user-supplied solver.
 \item
   \texttt{'sstate='} {[} \texttt{true} \textbar{} \emph{\texttt{false}}
   \textbar{} cell {]} - If \texttt{true} or a cell array, the steady
   state is re-computed in each iteration; the cell array can be used to
   modify the default options with which the \texttt{sstate} function is
   called.
 \end{itemize}
 
 \subparagraph{Options for linear models}
 
 \begin{itemize}
 \item
   \texttt{'refresh='} {[} \emph{\texttt{true}} \textbar{} \texttt{false}
   {]} - Refresh dynamic links before steady state is computed.
 \item
   \texttt{'solve='} {[} \texttt{true} \textbar{} \emph{\texttt{false}}
   {]} - Solve model before computing steady state.
 \end{itemize}
 
 \paragraph{Description}
 
 Note that for backward compatibility, the option \texttt{'growth='} is
 set to \texttt{false} by default so that either the model is assumed
 stationary or the steady-state growth rates have been already
 pre-assigned to the model object. To use the \texttt{sstate} function
 for computing both the steady-state levels and steady-state growth rates
 in a balanced-growth model, you need to set the option
 \texttt{'growth=' true}.
 
 \paragraph{Example}


