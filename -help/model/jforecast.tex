

    \filetitle{jforecast}{Forecast with judgmental adjustments (conditional forecasts)}{model/jforecast}

	\paragraph{Syntax}\label{syntax}

\begin{verbatim}
F = jforecast(M,D,Range,...)
\end{verbatim}

\paragraph{Input arguments}\label{input-arguments}

\begin{itemize}
\item
  \texttt{M} {[} model {]} - Solved model object.
\item
  \texttt{D} {[} struct {]} - Input data from which the initial
  condition is taken.
\item
  \texttt{Range} {[} numeric {]} - Forecast range.
\end{itemize}

\paragraph{Output arguments}\label{output-arguments}

\begin{itemize}
\itemsep1pt\parskip0pt\parsep0pt
\item
  \texttt{F} {[} struct {]} - Output struct with the judgmentally
  adjusted forecast.
\end{itemize}

\paragraph{Options}\label{options}

\begin{itemize}
\item
  \texttt{'anticipate='} {[} \emph{\texttt{true}} \textbar{}
  \texttt{false} {]} - If true, real future shocks are anticipated,
  imaginary are unanticipated; vice versa if false.
\item
  \texttt{'currentOnly='} {[} \emph{\texttt{true}} \textbar{}
  \texttt{false} {]} - If \texttt{true}, MSE matrices will be computed
  only for the current-dated variables, not for their lags or leads.
\item
  \texttt{'deviation='} {[} \texttt{true} \textbar{}
  \emph{\texttt{false}} {]} - Treat input and output data as deviations
  from balanced-growth path.
\item
  \texttt{'dtrends='} {[} \emph{\texttt{'auto'}} \textbar{}
  \texttt{true} \textbar{} \texttt{false} {]} - Measurement data contain
  deterministic trends.
\item
  \texttt{'initCond='} {[} \emph{\texttt{'data'}} \textbar{}
  \texttt{'fixed'} {]} - Use the MSE for the initial conditions if found
  in the input data or treat the initical conditions as fixed.
\item
  \texttt{'meanOnly='} {[} \texttt{true} \textbar{}
  \emph{\texttt{false}} {]} - Return only mean data, i.e.~point
  estimates.
\item
  \texttt{'plan='} {[} plan {]} - Simulation plan specifying the
  exogenised variables and endogenised shocks.
\item
  \texttt{'vary='} {[} struct \textbar{} \emph{empty} {]} - Database
  with time-varying std deviations or cross-correlations of shocks.
\end{itemize}

\paragraph{Description}\label{description}

When adjusting the mean and/or std devs of shocks, you can use real and
imaginary numbers ot distinguish between anticipated and unanticipated
shocks:

\begin{itemize}
\item
  any shock entered as an imaginary number is treated as an anticipated
  change in the mean of the shock distribution;
\item
  any std dev of a shock entered as an imaginary number indicates that
  the shock will be treated as anticipated when conditioning the
  forecast on the reduced-form tunes.
\item
  the same shock or its std dev can have both the real and the imaginary
  part.
\end{itemize}

\paragraph{Description}\label{description-1}

\paragraph{Example}\label{example}


