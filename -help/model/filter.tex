

    \filetitle{filter}{Kalman smoother and estimator of out-of-likelihood parameters}{model/filter}

	\paragraph{Syntax}

\begin{verbatim}
[M,Outp,V,Delta,PE,SCov] = filter(M,Inp,Range,...)
\end{verbatim}

\paragraph{Input arguments}

\begin{itemize}
\item
  \texttt{M} {[} model {]} - Solved model object.
\item
  \texttt{Inp} {[} struct \textbar{} cell {]} - Input database or
  datapack from which the measurement variables will be taken.
\item
  \texttt{Range} {[} numeric {]} - Filter date range.
\end{itemize}

\paragraph{Output arguments}

\begin{itemize}
\item
  \texttt{M} {[} model {]} - Model object with updates of std devs (if
  \texttt{'relative='} is true) and/or updates of out-of-likelihood
  parameters (if \texttt{'outoflik='} is non-empty).
\item
  \texttt{Outp} {[} struct \textbar{} cell {]} - Output struct with
  smoother or prediction data.
\item
  \texttt{V} {[} numeric {]} - Estimated variance scale factor if the
  \texttt{'relative='} options is true; otherwise \texttt{V} is 1.
\item
  \texttt{Delta} {[} struct {]} - Database with estimates of
  out-of-likelihood parameters.
\item
  \texttt{PE} {[} struct {]} - Database with prediction errors for
  measurement variables.
\item
  \texttt{SCov} {[} numeric {]} - Sample covariance matrix of smoothed
  shocks; the covariance matrix is computed using shock estimates in
  periods that are included in the option \texttt{'objrange='} and, at
  the same time, contain at least one observation of measurement
  variables.
\end{itemize}

\paragraph{Options}

\begin{itemize}
\item
  \texttt{'ahead='} {[} numeric \textbar{} \emph{\texttt{1}} {]} -
  Predictions will be computed this number of period ahead.
\item
  \texttt{'chkFmse='} {[} \texttt{true} \textbar{} \emph{\texttt{false}}
  {]} - Check the condition number of the forecast MSE matrix in each
  step of the Kalman filter, and return immediately if the matrix is
  ill-conditioned; see also the option \texttt{'fmseCondTol='}.
\item
  \texttt{'condition='} {[} char \textbar{} cellstr \textbar{}
  \emph{empty} {]} - List of conditioning measurement variables.
  Condition time t\textbar{}t-1 prediction errors (that enter the
  likelihood function) on time t observations of these measurement
  variables.
\item
  \texttt{'deviation='} {[} \texttt{true} \textbar{}
  \emph{\texttt{false}} {]} - Treat input and output data as deviations
  from balanced-growth path.
\item
  \texttt{'dtrends='} {[} \emph{`auto'} \textbar{} \texttt{true}
  \textbar{} \texttt{false} {]} - Measurement data contain deterministic
  trends.
\item
  \texttt{'data='} {[} \texttt{'predict'} \textbar{}
  \emph{\texttt{'smooth'}} \textbar{} \texttt{'predict,smooth'} {]} -
  Return smoother data or prediction data or both.
\item
  \texttt{'fmseCondTol='} {[} \emph{\texttt{eps()}} \textbar{} numeric
  {]} - Tolerance for the FMSE condition number test; not used unless
  \texttt{'chkFmse=' true}.
\item
  \texttt{'initCond='} {[} \texttt{'fixed'} \textbar{}
  \texttt{'optimal'} \textbar{} \emph{\texttt{'stochastic'}} \textbar{}
  struct {]} - Method or data to initialise the Kalman filter;
  user-supplied initial condition must be a mean database or a mean-MSE
  struct.
\item
  \texttt{'lastSmooth='} {[} numeric \textbar{} \emph{\texttt{Inf}} {]}
  - Last date up to which to smooth data backward from the end of the
  range; if \texttt{Inf} smoother will run on the entire range.
\item
  \texttt{'meanOnly='} {[} \texttt{true} \textbar{}
  \emph{\texttt{false}} {]} - Return a plain database with mean data
  only; this option overrides the \texttt{'return*='} options, i.e.
  \texttt{'returnCont='}, \texttt{'returnMse='}, \texttt{'returnStd='}.
\item
  \texttt{'outOfLik='} {[} cellstr \textbar{} empty {]} - List of
  parameters in deterministic trends that will be estimated by
  concentrating them out of the likelihood function.
\item
  \texttt{'objFunc='} {[} \emph{\texttt{'-loglik'}} \textbar{}
  \texttt{'prederr'} {]} - Objective function computed; can be either
  minus the log likelihood function or weighted sum of prediction
  errors.
\item
  \texttt{'objRange='} {[} numeric \textbar{} \emph{\texttt{Inf}} {]} -
  The objective function will be computed on the specified range only;
  \texttt{Inf} means the entire filter range.
\item
  \texttt{'precision='} {[} \emph{\texttt{'double'}} \textbar{}
  \texttt{'single'} {]} - Numeric precision to which output data will be
  stored; all calculations themselves always run to double precision.
\item
  \texttt{'relative='} {[} \emph{\texttt{true}} \textbar{}
  \texttt{false} {]} - Std devs of shocks assigned in the model object
  will be treated as relative std devs, and a common variance scale
  factor will be estimated.
\item
  \texttt{'returnCont='} {[} \texttt{true} \textbar{}
  \emph{\texttt{false}} {]} - Return contributions of prediction errors
  in measurement variables to the estimates of all variables and shocks.
\item
  \texttt{'returnMse='} {[} \emph{\texttt{true}} \textbar{}
  \texttt{false} {]} - Return MSE matrices for predetermined state
  variables; these can be used for settin up initial condition in
  subsequent call to another \texttt{filter} or \texttt{jforecast}.
\item
  \texttt{'returnStd='} {[} \emph{\texttt{true}} \textbar{}
  \texttt{false} {]} - Return database with std devs of model variables.
\item
  \texttt{'weighting='} {[} numeric \textbar{} \emph{empty} {]} -
  Weighting vector or matrix for prediction errors when
  \texttt{'objective=' 'prederr'}; empty means prediction errors are
  weighted equally.
\end{itemize}

\paragraph{Options for models with non-linearised
equations}

\begin{itemize}
\item
  \texttt{'nonlinear='} {[} numeric \textbar{} \emph{\texttt{0}} {]} -
  If non-zero the prediction step in the Kalman filter will be run in an
  exact non-linear mode using the same technique as
  \href{model/simulate}{\texttt{model/simulate}}.
\item
  \texttt{'simulate='} {[} cell \textbar{} empty {]} - Options passed in
  to \texttt{simulate} when invoking the non-linear simulation in the
  prediction step; only used when \texttt{nonlinear=} is greater than
  \texttt{0}.
\end{itemize}

\paragraph{Description}

The \texttt{'ahead='} and \texttt{'rollback='} options cannot be
combined with one another, or with multiple data sets, or with multiple
parameterisations.

\subparagraph{Initial conditions in time
domain}

By default (with \texttt{'initCond=' 'stochastic'}), the Kalman filter
starts from the model-implied asymptotic distribution. You can change
this behaviour by setting the option \texttt{'initCond='} to one of the
following four different values:

\begin{itemize}
\item
  \texttt{'fixed'} -- the filter starts from the model-implied
  asymptotic mean (steady state) but with no initial uncertainty. The
  initial condition is treated as a vector of fixed, non-stochastic,
  numbers.
\item
  \texttt{'optimal'} -- the filter starts from a vector of fixed numbers
  that is estimated optimally (likelihood maximising).
\item
  database (i.e.~struct with fields for individual model variables) -- a
  database through which you supply the mean for all the required
  initial conditions, see help on \href{model/get}{\texttt{model/get}}
  for how to view the list of required initial conditions.
\item
  mean-mse struct (i.e.~struct with fields \texttt{.mean} and
  \texttt{.mse}) -- a struct through which you supply the mean and MSE
  for all the required initial conditions.
\end{itemize}

\subparagraph{Contributions of measurement variables to the estimates of
all
variables}

Use the option \texttt{'returnCont=' true} to request the decomposition
of measurement variables, transition variables, and shocks into the
contributions of each individual measurement variable. The resulting
output database will include one extra subdatabase called
\texttt{.cont}. In the \texttt{.cont} subdatabase, each time series will
have Ny columns where Ny is the number of measurement variables in the
model. The k-th column will be the contribution of the observations on
the k-th measurement variable.

The contributions are additive for linearised variables, and
multiplicative for log-linearised variables (log-variables). The
difference between the actual path for a particular variable and the sum
of the contributions (or their product in the case of log-varibles) is
due to the effect of constant terms and deterministic trends.

\paragraph{Example}


