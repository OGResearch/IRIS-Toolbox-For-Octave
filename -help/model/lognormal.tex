

    \filetitle{lognormal}{Characteristics of log-normal distributions returned by filter of forecast}{model/lognormal}

	\paragraph{Syntax}
 
 \begin{verbatim}
 D = lognormal(M,D,...)
 \end{verbatim}
 
 \paragraph{Input arguments}
 
 \begin{itemize}
 \item
   \texttt{M} {[} model {]} - Model on which the \texttt{filter} or
   \texttt{forecast} function has been run.
 \item
   \texttt{D} {[} struct {]} - Struct or database returned from the
   \texttt{filter} or \texttt{forecast} function.
 \end{itemize}
 
 \paragraph{Output arguments}
 
 \begin{itemize}
 \item
   \texttt{D} {[} struct {]} - Struct including new sub-databases with
   requested log-normal statistics.
 \end{itemize}
 
 \paragraph{Options}
 
 \begin{itemize}
 \item
   \texttt{'fresh='} {[} \texttt{true} \textbar{} \emph{\texttt{false}}
   {]} - Output structure will include only the newly computed databases.
 \item
   \texttt{'mean='} {[} \emph{\texttt{true}} \textbar{} \texttt{false}
   {]} - Compute the mean of the log-normal distributions.
 \item
   \texttt{'median='} {[} \emph{\texttt{true}} \textbar{} \texttt{false}
   {]} - Compute the median of the log-normal distributions.
 \item
   \texttt{'mode='} {[} \emph{\texttt{true}} \textbar{} \texttt{false}
   {]} - Compute the mode of the log-normal distributions.
 \item
   \texttt{'prctile='} {[} numeric \textbar{} \emph{{[}5,95{]}} {]} -
   Compute the selected percentiles of the log-normal distributions.
 \item
   \texttt{'prefix='} {[} char \textbar{} \emph{`lognormal'} {]} - Prefix
   used in the names of the newly created databases.
 \item
   \texttt{'std='} {[} \emph{\texttt{true}} \textbar{} \texttt{false} {]}
   - Compute the std deviations of the log-normal distributions.
 \end{itemize}
 
 \paragraph{Description}


