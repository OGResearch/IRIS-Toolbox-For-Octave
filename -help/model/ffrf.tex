

    \filetitle{ffrf}{Filter frequency response function of transition variables to measurement variables}{model/ffrf}

	\paragraph{Syntax}
 
 \begin{verbatim}
 [F,List] = ffrf(M,Freq,...)
 \end{verbatim}
 
 \paragraph{Input arguments}
 
 \begin{itemize}
 \item
   \texttt{M} {[} model {]} - Model object for which the frequency
   response function will be computed.
 \item
   \texttt{Freq} {[} numeric {]} - Vector of frequencies for which the
   response function will be computed.
 \end{itemize}
 
 \paragraph{Output arguments}
 
 \begin{itemize}
 \item
   \texttt{F} {[} numeric {]} - Array with frequency responses of
   transition variables (in rows) to measurement variables (in columns).
 \item
   \texttt{List} {[} cell {]} - List of transition variables in rows of
   the \texttt{F} matrix, and list of measurement variables in columns of
   the \texttt{F} matrix.
 \end{itemize}
 
 \paragraph{Options}
 
 \begin{itemize}
 \item
   \texttt{'exclude='} {[} char \textbar{} cellstr \textbar{}
   \emph{empty} {]} - Remove the effect of these measurement variables
   from the FFRF.
 \item
   \texttt{'maxIter='} {[} numeric \textbar{} \emph{500} {]} - Maximum
   number of iteration when computing the steady-state Kalman filter.
 \item
   \texttt{'output='} {[} \emph{`namedmat'} \textbar{} numeric {]} -
   Output matrix \texttt{F} will be either a namedmat object or a plain
   numeric array; if the option \texttt{'select='} is used,
   \texttt{'output='} is always \texttt{'namedmat'}.
 \item
   \texttt{'select='} {[} char \textbar{} cellstr \textbar{}
   \emph{\texttt{Inf}} {]} - Return the frequency response function for
   selected variables only; \texttt{Inf} means all variables.
 \item
   \texttt{'tolerance='} {[} numeric \textbar{} \emph{1e-7} {]} -
   Convergence tolerance when computing the steady-state Kalman filter.
 \end{itemize}
 
 \paragraph{Description}
 
 \paragraph{Example}


