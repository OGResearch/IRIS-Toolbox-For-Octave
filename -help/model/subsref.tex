

    \filetitle{subsref}{Subscripted reference for model and systemfit objects}{model/subsref}

	\paragraph{Syntax for retrieving object with subset of
parameterisations}\label{syntax-for-retrieving-object-with-subset-of-parameterisations}

\begin{verbatim}
M(Inx)
\end{verbatim}

\paragraph{Syntax for retrieving parameters or steady-state
values}\label{syntax-for-retrieving-parameters-or-steady-state-values}

\begin{verbatim}
M.Name
\end{verbatim}

\paragraph{Syntax to retrieve a std deviation or a cross-correlation of
shocks}\label{syntax-to-retrieve-a-std-deviation-or-a-cross-correlation-of-shocks}

\begin{verbatim}
M.std_ShockName
M.corr_ShockName1__ShockName2
\end{verbatim}

Note that a double underscore is used to separate the names of shocks in
correlation coefficients.

\paragraph{Input arguments}\label{input-arguments}

\begin{itemize}
\item
  \texttt{M} {[} model \textbar{} systemfit {]} - Model or systemfit
  object.
\item
  \texttt{Inx} {[} numeric \textbar{} logical {]} - Inx of requested
  parameterisations.
\item
  \texttt{Name} - Name of a variable, shock, or parameter.
\item
  \texttt{ShockName1}, \texttt{ShockName2} - Name of a shock.
\end{itemize}

\paragraph{Description}\label{description}

\paragraph{Example}\label{example}


