

    \filetitle{fisher}{Approximate Fisher information matrix in frequency domain}{model/fisher}

	\paragraph{Syntax}
 
 \begin{verbatim}
 [F,FF,Delta,Freq] = fisher(M,NPer,List,...)
 \end{verbatim}
 
 \paragraph{Input arguments}
 
 \begin{itemize}
 \item
   \texttt{M} {[} model {]} - Solved model object.
 \item
   \texttt{NPer} {[} numeric {]} - Length of the hypothetical range for
   which the Fisher information will be computed.
 \item
   \texttt{List} {[} cellstr {]} - List of parameters with respect to
   which the likelihood function will be differentiated.
 \end{itemize}
 
 \paragraph{Output arguments}
 
 \begin{itemize}
 \item
   \texttt{F} {[} numeric {]} - Approximation of the Fisher information
   matrix.
 \item
   \texttt{FF} {[} numeric {]} - Contributions of individual frequencies
   to the total Fisher information matrix.
 \item
   \texttt{Delta} {[} numeric {]} - Kronecker delta by which the
   contributions in \texttt{Fi} need to be multiplied to sum up to
   \texttt{F}.
 \item
   \texttt{Freq} {[} numeric {]} - Vector of frequencies at which the
   Fisher information matrix is evaluated.
 \end{itemize}
 
 \paragraph{Options}
 
 \begin{itemize}
 \item
   \texttt{'chkSstate='} {[} \texttt{true} \textbar{}
   \emph{\texttt{false}} \textbar{} cell {]} - Check steady state in each
   iteration; works only in non-linear models.
 \item
   \texttt{'deviation='} {[} \emph{\texttt{true}} \textbar{}
   \texttt{false} {]} - Exclude the steady state effect at zero
   frequency.
 \item
   \texttt{'exclude='} {[} char \textbar{} cellstr \textbar{}
   \emph{empty} {]} - List of measurement variables that will be excluded
   from the likelihood function.
 \item
   \texttt{'percent='} {[} \texttt{true} \textbar{} \emph{\texttt{false}}
   {]} - Report the overall Fisher matrix \texttt{F} as Hessian w.r.t.
   the log of variables; the interpretation for this is that the Fisher
   matrix describes the changes in the log-likelihood function in reponse
   to percent, not absolute, changes in parameters.
 \item
   \texttt{'progress='} {[} \texttt{true} \textbar{}
   \emph{\texttt{false}} {]} - Display progress bar in the command
   window.
 \item
   \texttt{'refresh='} {[} \emph{\texttt{true}} \textbar{} \texttt{false}
   {]} - Refresh dynamic links in each differentiation step.
 \item
   \texttt{'solve='} {[} \emph{\texttt{true}} \textbar{} \texttt{false}
   \textbar{} cellstr {]} - Re-compute solution in each differentiation
   step; you can specify a cell array with options for the \texttt{solve}
   function.
 \item
   \texttt{'sstate='} {[} \texttt{true} \textbar{} \emph{\texttt{false}}
   \textbar{} cell {]} - Re-compute steady state in each differentiation
   step; if the model is non-linear, you can pass in a cell array with
   opt used in the \texttt{sstate} function.
 \end{itemize}
 
 \paragraph{Description}
 
 \paragraph{Example}


