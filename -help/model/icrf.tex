

    \filetitle{icrf}{Initial-condition response functions}{model/icrf}

	\paragraph{Syntax}\label{syntax}

\begin{verbatim}
S = icrf(M,NPer,...)
S = icrf(M,Range,...)
\end{verbatim}

\paragraph{Input arguments}\label{input-arguments}

\begin{itemize}
\item
  \texttt{M} {[} model {]} - Model object for which the initial
  condition responses will be simulated.
\item
  \texttt{Range} {[} numeric {]} - Date range with the first date being
  the shock date.
\item
  \texttt{NPer} {[} numeric {]} - Number of periods.
\end{itemize}

\paragraph{Output arguments}\label{output-arguments}

\begin{itemize}
\itemsep1pt\parskip0pt\parsep0pt
\item
  \texttt{S} {[} struct {]} - Database with initial condition response
  series.
\end{itemize}

\paragraph{Options}\label{options}

\begin{itemize}
\item
  \texttt{'delog='} {[} \emph{\texttt{true}} \textbar{} \texttt{false}
  {]} - Delogarithmise the responses for variables declared as
  \texttt{!variables:log}.
\item
  \texttt{'size='} {[} numeric \textbar{} \emph{\texttt{1}} for linear
  models \textbar{} \emph{\texttt{log(1.01)}} for non-linear models {]}
  - Size of the deviation in initial conditions.
\end{itemize}

\paragraph{Description}\label{description}

\paragraph{Example}\label{example}


