
    \foldertitle{namedmat}{Matrices with named rows and columns}{namedmat/Contents}

	Matrices with named rows and columns are returned by several IRIS
functions, such as \url{model/acf}, \url{model/xsf}, or
\url{model/fmse}, to facilitate easy selection of submatrices by
referrring to variable names in rows and columns.

Namedmat methods:

\paragraph{Constructor}\label{constructor}

\begin{itemize}
\itemsep1pt\parskip0pt\parsep0pt
\item
  \href{namedmat/namedmat}{\texttt{namedmat}} - Create a new matrix with
  named rows and columns.
\end{itemize}

\paragraph{Manipulating named
matrices}\label{manipulating-named-matrices}

\begin{itemize}
\itemsep1pt\parskip0pt\parsep0pt
\item
  \href{namedmat/select}{\texttt{select}} - Select submatrix by
  referring to row names and column names.
\item
  \href{namedmat/transpose}{\texttt{transpose}} - Transpose each page of
  matrix with names rows and columns.
\end{itemize}

\paragraph{Getting row and column
names}\label{getting-row-and-column-names}

\begin{itemize}
\itemsep1pt\parskip0pt\parsep0pt
\item
  \href{namedmat/rownames}{\texttt{rownames}} - Names of rows in
  namedmat object.
\item
  \href{namedmat/colnames}{\texttt{colnames}} - Names of columns in
  namedmat object.
\end{itemize}

\paragraph{Sample characteristics}\label{sample-characteristics}

\begin{itemize}
\itemsep1pt\parskip0pt\parsep0pt
\item
  {[}\texttt{cutoff}{]}(namedmat/cutoff{]} -
\end{itemize}

All operators and functions available for standard Matlab matrices and
arrays (i.e.~double objects) are also available for namedmat objects.



