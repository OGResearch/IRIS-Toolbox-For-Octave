

    \filetitle{namedmat}{Create a new matrix with named rows and columns}{namedmat/namedmat}

	\paragraph{Syntax}

\begin{verbatim}
X = namedmat(X,ROWNAMES,COLNAMES)
\end{verbatim}

\paragraph{Input arguments}

\begin{itemize}
\item
  \texttt{X} {[} numeric {]} - Matrix or multidimensional array.
\item
  \texttt{ROWNAMES} {[} cellstr {]} - Names for individual rows of
  \texttt{X}.
\item
  \texttt{COLNAMES} {[} cellstr {]} - Names for individual columns of
  \texttt{X}.
\end{itemize}

\paragraph{Output arguments}

\begin{itemize}
\itemsep1pt\parskip0pt\parsep0pt
\item
  \texttt{X} {[} namedmat {]} - Matrix with named rows and columns.
\end{itemize}

\paragraph{Description}

Namedmat objects are used by some of the IRIS functions to preserve the
names of variables that relate to individual rows and columns, such as
in

\begin{itemize}
\itemsep1pt\parskip0pt\parsep0pt
\item
  \texttt{acf}, the autocovariance and autocorrelation functions,
\item
  \texttt{xsf}, the power spectrum and spectral density functions,
\item
  \texttt{fmse}, the forecast mean square error fuctions,
\item
  etc.
\end{itemize}

You can use the function \href{namedmat/select}{\texttt{select}} to
extract submatrices by referring to a selection of names.

Namedmat matrices derives from the built-in double class of objects, and
hence you can use any operators and functions on them that are available
for double objects.

\paragraph{Example}


